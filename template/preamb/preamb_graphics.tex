\usepackage{graphicx}

\usepackage{pgfplots}
\pgfplotsset{compat=newest}

\usepackage{tikz}
\usetikzlibrary{calc, external}
%\tikzexternalize[prefix=figures/pgf/]



%\pgfplotscreateplotcyclelist{mycolorlist}{%
%blue,every mark/.append style={fill=blue!80!black},mark=*\\%
%red,every mark/.append style={fill=red!80!black},mark=square*\\%
%brown!60!black,every mark/.append style={fill=brown!80!black},mark=otimes*\\%
%black,mark=star\\%
%blue,every mark/.append style={fill=blue!80!black},mark=diamond*\\%
%red,densely dashed,every mark/.append style={solid,fill=red!80!black},mark=*\\%
%brown!60!black,densely dashed,every mark/.append style={
%solid,fill=brown!80!black},mark=square*\\%
%black,densely dashed,every mark/.append style={solid,fill=gray},mark=otimes*\\%
%blue,densely dashed,mark=star,every mark/.append style=solid\\%
%red,densely dashed,every mark/.append style={solid,fill=red!80!black},mark=diamond*\\%
%}


\usetikzlibrary{pgfplots.colorbrewer}


%\definecolor{s1}{RGB}{228, 26, 28}
%\definecolor{s2}{RGB}{55, 126, 184}
%\definecolor{s3}{RGB}{77, 175, 74}
%\definecolor{s4}{RGB}{152, 78, 163}
%\definecolor{s5}{RGB}{255, 127, 0}

\definecolor{s1}{HTML}{68ABD9}
\definecolor{s2}{HTML}{FA6655}%{FAA43A}%
\definecolor{s3}{HTML}{AADA57}%{60BD68}%
\definecolor{s4}{HTML}{897EDA}
\definecolor{s5}{HTML}{FAA43A}%{FA6655}%
\definecolor{s6}{HTML}{60BD68}%{AADA57}%
\definecolor{s7}{HTML}{F094C3}
\definecolor{s8}{HTML}{A3A3A3}
\newcommand{\lwid}{0.0pt}

\pgfplotscreateplotcyclelist{set1}{
  	s1, mark=square*,   mark size=1.2pt\\%, line width=\lwid\\
	s2, mark=*,         mark size=1.5pt\\%, line width=\lwid\\
	s3, mark=triangle*, mark size=1.7pt\\%, line width=\lwid\\
	s4, mark=diamond*,  mark size=1.7pt\\%, line width=\lwid\\
	s5, mark=square*,   mark size=1.2pt\\%, line width=\lwid\\
	s6, mark=*,         mark size=1.5pt\\%, line width=\lwid\\
	s7, mark=triangle*, mark size=1.7pt\\%, line width=\lwid\\
	s8, mark=diamond*,  mark size=1.5pt\\%, line width=\lwid\\
}


\pgfplotsset{
        % define a `cycle list' for marker
        cycle list/.define={my marks}{
            every mark/.append style={line width=\lwid, fill=\pgfkeysvalueof{/pgfplots/mark list fill}}\\
            every mark/.append style={line width=\lwid, fill=\pgfkeysvalueof{/pgfplots/mark list fill}}\\
            every mark/.append style={line width=\lwid, fill=\pgfkeysvalueof{/pgfplots/mark list fill}}\\
            every mark/.append style={line width=\lwid, fill=\pgfkeysvalueof{/pgfplots/mark list fill}}\\
        }
}

\pgfmathsetlengthmacro\MajorTickLength{
	\pgfkeysvalueof{/pgfplots/major tick length} * 0.75
}
\pgfmathsetlengthmacro\MinorTickLength{
	\MajorTickLength * 0.5
	%\pgfkeysvalueof{/pgfplots/minor tick length} * 0.5
}
\pgfplotsset{
	cycle multiindex* list={
		set1
		\nextlist
		thick
		\nextlist
		mark options={scale=.85}%
	},
	clip mode=individual,
	axis line style={line width=0.5pt},
	grid style={line width=0.3pt, draw=black!13},
	major grid style={line width=0.4pt,draw=black!25},
%	grid style={line width=0.1pt, draw=gray!16},
%	major grid style={line width=0.2pt,draw=gray!40},
	every tick/.style={
        black,
        line width=0.5pt
      },
    major tick length=\MajorTickLength,
    minor tick length=\MinorTickLength,
	legend cell align={left},
	legend style={line width=0.5pt}
}







\makeatletter
\newcommand{\gettikzxy}[3]{%
  \tikz@scan@one@point\pgfutil@firstofone#1\relax
  \edef#2{\the\pgf@x}%
  \edef#3{\the\pgf@y}%
}

\newcommand{\globalgettikzxy}[3]{%
  \tikz@scan@one@point\pgfutil@firstofone#1\relax
  \edef\@tempdima{\the\pgf@x}%
  \edef\@tempdimb{\the\pgf@y}%
  \global#2=\@tempdima%
  \global#3=\@tempdimb%
}

\newcommand\getwidthofnode[2]{%
    \pgfextractx{\pgf@xb}{\pgfpointanchor{#2}{east}}%
    \pgfextractx{\pgf@xa}{\pgfpointanchor{#2}{west}}% 
    \pgfmathsetlength{\pgf@xb}{\pgf@xb - \pgf@xa}%
	\global#1=\pgf@xb%
}

\newcommand\getheightofnode[2]{%
    \pgfextracty{\pgf@yb}{\pgfpointanchor{#2}{north}}%
    \pgfextracty{\pgf@ya}{\pgfpointanchor{#2}{south}}% 
    \pgfmathsetlength{\pgf@yb}{\pgf@yb - \pgf@ya}%
	\global#1=\pgf@yb%
}
\makeatother




% chemin(s) des figures
\graphicspath{{./figures/}}%,{../fig2/}}


% controle des objets flottants (figures, tables)
\renewcommand{\topfraction}{0.7}     % autorise 70% page de graphique en haut
\renewcommand{\bottomfraction}{0.5}  % autorise 50% page de graphique en bas
\renewcommand{\floatpagefraction}{0.7}
\renewcommand{\textfraction}{0.1}