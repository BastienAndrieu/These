\setlength{\imagewidth}{50mm}%
\setlength{\imageheight}{\imagewidth}%
\begin{figure}
	\centering
	\colorlet{colorContourEdSnappe0}{mycolor_2}
	\colorlet{colorContourEdSnappe1}{mycolor_3}
	\colorlet{colorContourEdSarete}{mycolor_1}
	\colorlet{colorInterieurEdSnappe0}{colorContourEdSnappe0}%!50!white}
	\colorlet{colorInterieurEdSnappe1}{colorContourEdSnappe1}%!50!white}
	\colorlet{colorInterieurarete}{colorContourEdSarete!30!white}
	\colorlet{colorPseudoEdSarete}{mycolor_4}%colorContourEdSarete!70!black}
	\begin{tikzpicture}[
		x = \imagewidth,
		y = \imageheight,
		styleEdS/.style = {
			thick, fill opacity=0.2
		},
		stylePseudoEdS/.style = {
			line width=1.1pt, colorPseudoEdSarete, line cap=round%, dash pattern=on 4pt off 3pt
		},
		styleNappe/.style = {
			draw=black, thick, line cap=round
		}
	]
		\begin{scope}
			%\clip (-1,-0.55) rectangle (1,0.8);
			\begin{scope}[blend group = overlay]
\draw[styleEdS, draw=colorContourEdSnappe0, fill=colorInterieurEdSnappe0] 
(1.0398871880695557, -0.05029176412270403) -- 
(1.0183167534800655, -0.0449025730216131) -- 
(0.9970249326809821, -0.03889063165462239) -- 
(0.9760087070839042, -0.03227375493086215) -- 
(0.9552635663797575, -0.025069985986572035) -- 
(0.9347833891410695, -0.017297348539330626) -- 
(0.9145603833626653, -0.008973612328943836) -- 
(0.8945850834066453, -0.00011607984793962633) -- 
(0.8748463978997224, 0.009258598781243438) -- 
(0.8553317017006214, 0.019134577567845562) -- 
(0.83602696412914, 0.029496929643137104) -- 
(0.816916905210558, 0.04033172390652845) -- 
(0.7979851716958212, 0.05162606350851576) -- 
(0.7792145250047978, 0.06336808792773979) -- 
(0.7605870339290941, 0.07554694172716503) -- 
(0.742084265838364, 0.08815271410883804) -- 
(0.7236874711762205, 0.10117635410812229) -- 
(0.7053777571313378, 0.11460956667949797) -- 
(0.68713624745879, 0.12844469504848804) -- 
(0.6689442264517612, 0.1426745945736954) -- 
(0.650783265984124, 0.1572925030243364) -- 
(0.6326353353336946, 0.1722919116818077) -- 
(0.6144828941409042, 0.18766644106904284) -- 
(0.5963089693558062, 0.20340972444626615) -- 
(0.5780972173844605, 0.21951530152848314) -- 
(0.5598319728767698, 0.2359765242139565) -- 
(0.5414982857190578, 0.2527864754915883) -- 
(0.5230819478253321, 0.26993790213828667) -- 
(0.5045695112809123, 0.28742316133741974) -- 
(0.4859482992996188, 0.3052341809521778) -- 
(0.4672064113277807, 0.3233624328735336) -- 
(0.4483327234793507, 0.34179891862769246) -- 
(0.4293168853280881, 0.36053416626560775) -- 
(0.4101493139242216, 0.37955823745850226) -- 
(0.3908211857508439, 0.3988607436785092) -- 
(0.3713244271940277, 0.4184308703424265) -- 
(0.35165170397301176, 0.4382574078293584) -- 
(0.33179640986407155, 0.4583287883406225) -- 
(0.3117526549541225, 0.4786331276446601) -- 
(0.291515253577205, 0.4991582708338544) -- 
(0.28440988145616547, 0.5063087863339162) -- 
(0.27710254029824977, 0.5132527729480235) -- 
(0.2695992307813853, 0.5199845283794992) -- 
(0.26190611450974904, 0.5264985246127) -- 
(0.25402950895395104, 0.5327894124525396) -- 
(0.24597588226322395, 0.5388520259171669) -- 
(0.23775184795387602, 0.5446813864801907) -- 
(0.2293641594783722, 0.5502727071589723) -- 
(0.22081970467950082, 0.5556213964456214) -- 
(0.2121255001341817, 0.5607230620774737) -- 
(0.20328868539155925, 0.5655735146439503) -- 
(0.19431651711011275, 0.5701687710268386) -- 
(0.18521636309859835, 0.5745050576711679) -- 
(0.17599569626571607, 0.5785788136839961) -- 
(0.1666620884834703, 0.5823866937585616) -- 
(0.1572232043692633, 0.5859255709213973) -- 
(0.147686794991827, 0.5891925391001548) -- 
(0.13806069150616307, 0.592184915510027) -- 
(0.1283527987227171, 0.59490024285681) -- 
(0.1185710886160675, 0.597336291354796) -- 
(0.10872359377846114, 0.5994910605578396) -- 
(0.09881840082356977, 0.6013627810020942) -- 
(0.0888636437458856, 0.6029499156590681) -- 
(0.07886749724120823, 0.6042511611978101) -- 
(0.06883816999370787, 0.6052654490551841) -- 
(0.058783897935078405, 0.6059919463133578) -- 
(0.048712937481315166, 0.606430056383782) -- 
(0.038633558752670744, 0.6065794194970995) -- 
(0.028554038782357544, 0.6064399129985829) -- 
(0.018482654719573693, 0.6060116514488569) -- 
(0.008427677032433106, 0.6052949865298223) -- 
(-0.0016026372836175917, 0.6042905067558604) -- 
(-0.011600051486318735, 0.602999036990554) -- 
(-0.02155635585048546, 0.601421637769322) -- 
(-0.031463374409710254, 0.5995596044285232) -- 
(-0.04131297167034209, 0.5974144660417463) -- 
(-0.051097059292231184, 0.5949879841641577) -- 
(-0.06080760273075139, 0.592282151385939) -- 
(-0.07043662783464624, 0.5892991896960018) -- 
(-0.07997622739428173, 0.5860415486573246) -- 
(-0.08941856763492664, 0.5825119033954085) -- 
(-0.09875589464973036, 0.5787131524015046) -- 
(-0.10798054076711328, 0.5746484151524166) -- 
(-0.11708493084734278, 0.5703210295488347) -- 
(-0.12606158850312335, 0.5657345491743005) -- 
(-0.1349031422390926, 0.5608927403770597) -- 
(-0.14360233150518156, 0.5557995791771925) -- 
(-0.15215201265886868, 0.550459248001567) -- 
(-0.16054516483143072, 0.5448761322492933) -- 
(-0.16877489569337395, 0.5390548166905) -- 
(-0.17683444711431084, 0.5330000817013907) -- 
(-0.18471720071263423, 0.526716899338671) -- 
(-0.19241668329043216, 0.5202104292565717) -- 
(-0.19992657214918022, 0.5134860144698171) -- 
(-0.2072407002818455, 0.5065491769660245) -- 
(-0.2143530614371397, 0.49940561317112975) -- 
(-0.22125781505176162, 0.4920611892715692) -- 
(-0.22794929104657935, 0.48452193639705643) -- 
(-0.23442199448281317, 0.4767940456679094) -- 
(-0.24067061007439672, 0.4688838631109964) -- 
(-0.24669000655280857, 0.4607978844484749) -- 
(-0.252475240880793, 0.45254274976360276) -- 
(-0.25802156231150697, 0.44412523804800214) -- 
(-0.2633244162897616, 0.43555226163485494) -- 
(-0.2683794481921539, 0.42683086052259867) -- 
(-0.2731825069030178, 0.4179681965937866) -- 
(-0.2777296482232569, 0.4089715477338585) -- 
(-0.2820171381092618, 0.39984830185465114) -- 
(-0.2860414557392498, 0.39060595082755645) -- 
(-0.28979929640451024, 0.38125208433131086) -- 
(-0.2932875742231817, 0.3717943836194658) -- 
(-0.2965034246743312, 0.3622406152126584) -- 
(-0.29944420695025553, 0.35259862452086255) -- 
(-0.3021075061250724, 0.3428763294008567) -- 
(-0.30449113513782133, 0.3330817136542001) -- 
(-0.30659313658844445, 0.3232228204710559) -- 
(-0.3084117843451744, 0.31330774582524457) -- 
(-0.30994558496200736, 0.30334463182595317) -- 
(-0.31119327890509796, 0.29334166003155904) -- 
(-0.31215384158707016, 0.2833070447310568) -- 
(-0.312826484208392, 0.2732490261986096) -- 
(-0.3132106544051267, 0.2631758639267602) -- 
(-0.3133060367025248, 0.2530958298438609) -- 
(-0.31311255277408784, 0.24301720152129108) -- 
(-0.3126303615058885, 0.23294825537604125) -- 
(-0.311859858866096, 0.22289725987424416) -- 
(-0.31080167757981286, 0.21287246874123647) -- 
(-0.3094566866094909, 0.20288211418372343) -- 
(-0.30782599044135184, 0.19293440012961588) -- 
(-0.3059109281784007, 0.18303749549108922) -- 
(-0.30371307244077433, 0.1731995274563957) -- 
(-0.3012342280743293, 0.16342857481594059) -- 
(-0.29847643066853036, 0.15373266132810265) -- 
(-0.2954419448848546, 0.14411974913024406) -- 
(-0.29213326259708616, 0.13459773220032423) -- 
(-0.28855310084502717, 0.12517442987448518) -- 
(-0.2847043996033055, 0.11585758042593017) -- 
(-0.2805903193671125, 0.10665483471037143) -- 
(-0.27621423855685184, 0.09757374988326305) -- 
(-0.2715797507438321, 0.08862178319397956) -- 
(-0.26669066169928013, 0.07980628586203543) -- 
(-0.26155098626909984, 0.07113449704037526) -- 
(-0.25616494507694093, 0.0626135378706904) -- 
(-0.2505369610582869, 0.0542504056356455) -- 
(-0.2446716558284075, 0.0460519680128164) -- 
(-0.23857384588715833, 0.03802495743505638) -- 
(-0.23224853866374337, 0.030175965561924676) -- 
(-0.2257009284046912, 0.022511437866715802) -- 
(-0.218936391908417, 0.015037668343533813) -- 
(-0.2119604841098781, 0.0077607943387599816) -- 
(-0.20477893351894494, 0.0006867915111568382) -- 
(-0.19739763751623587, -0.006178531075252337) -- 
(-0.18982265751027697, -0.0128295357214991) -- 
(-0.18206021395996608, -0.019260760723262417) -- 
(-0.17411668126642477, -0.025466924855938287) -- 
(-0.1659985825384379, -0.0314429317115007) -- 
(-0.13814675776047186, -0.051389728187050454) -- 
(-0.11010149732607256, -0.07084782564879044) -- 
(-0.08187628874232598, -0.08980691529833429) -- 
(-0.05348514223118164, -0.10825785303989452) -- 
(-0.024942488161001697, -0.1261927311153278) -- 
(0.0037369331581589837, -0.14360494415097552) -- 
(0.032538188799990014, -0.16048924853161478) -- 
(0.0614462769517472, -0.17684181400939064) -- 
(0.09044625611895998, -0.19266026647310083) -- 
(0.11952337865593937, -0.20794372084870297) -- 
(0.14866322725437772, -0.22269280317764836) -- 
(0.17785185281340019, -0.23690966102675073) -- 
(0.2070759119251168, -0.25059796152153513) -- 
(0.2363228020507566, -0.26376287646259583) -- 
(0.265580792345058, -0.2764110541778851) -- 
(0.29483914802179645, -0.2885505779777789) -- 
(0.32408824615147613, -0.30019091130722525) -- 
(0.35331968085209736, -0.3113428299218043) -- 
(0.38252635598214596, -0.32201834164250964) -- 
(0.4117025636751192, -0.33223059445727054) -- 
(0.4408440473670057, -0.3419937739254134) -- 
(0.46994804835801024, -0.3513229909948436) -- 
(0.49901333540890813, -0.3602341614525258) -- 
(0.5280402173876334, -0.36874387829068567) -- 
(0.5570305395358062, -0.37686927828051764) -- 
(0.58598766449683, -0.38462790400140734) -- 
(0.6149164398129897, -0.39203756247917254) -- 
(0.6438231541327407, -0.39911618144689354) -- 
(0.6727154848442609, -0.40588166406421605) -- 
(0.7016024402409139, -0.41235174272506525) -- 
(0.7304942996033684, -0.4185438323598505) -- 
(0.7594025547290846, -0.4244748834068749) -- 
(0.788339856433129, -0.4301612343982981) -- 
(0.8173199693690699, -0.43561846388665537) -- 
(0.8463577381635287, -0.44086124123470705) -- 
(0.8754690673157085, -0.44590317560849646) -- 
(0.9046709165817106, -0.4507566623537582) -- 
(0.9339813126454302, -0.4554327258016044) -- 
(0.9634193767817649, -0.4599408574440896) -- 
(0.9693155652265435, -0.4609052976013954) -- 
(0.9752369803674806, -0.4617002980555111) -- 
(0.9811787540119019, -0.4623252052100757) -- 
(0.9871360012297284, -0.462779505308097) -- 
(0.9931038243695415, -0.46306282485432837) -- 
(0.9990773170851042, -0.4631749309223312) -- 
(1.0050515683690329, -0.463115731345972) -- 
(1.0110216665902994, -0.4628852747951948) -- 
(1.0169827035322463, -0.46248375073600834) -- 
(1.022929778427797, -0.46191148927471937) -- 
(1.0288580019885372, -0.46116896088654163) -- 
(1.0347625004243635, -0.4602567760288024) -- 
(1.0406384194503882, -0.45917568463906483) -- 
(1.0464809282778087, -0.45792657551857996) -- 
(1.0522852235854594, -0.45651047560157293) -- 
(1.0580465334687832, -0.45492854911096664) -- 
(1.0637601213629715, -0.453182096601235) -- 
(1.0694212899370539, -0.4512725538891737) -- 
(1.0750253849557292, -0.4492014908734677) -- 
(1.0805677991057672, -0.4469706102440253) -- 
(1.0860439757838356, -0.4445817460821393) -- 
(1.0914494128426344, -0.4420368623526284) -- 
(1.0967796662922613, -0.439338051289196) -- 
(1.1020303539537641, -0.4364875316743353) -- 
(1.1071971590618763, -0.43348764701519416) -- 
(1.1122758338139729, -0.4303408636168996) -- 
(1.1172622028623307, -0.42704976855492677) -- 
(1.1221521667468217, -0.42361706754817735) -- 
(1.1269417052652146, -0.4200455827345184) -- 
(1.1316268807783172, -0.41633825035060845) -- 
(1.1362038414472417, -0.41249811831791966) -- 
(1.1406688244001288, -0.4085283437369402) -- 
(1.1450181588257307, -0.40443219029161614) -- 
(1.1492482689913066, -0.40021302556616906) -- 
(1.153355677182352, -0.3958743182764921) -- 
(1.1573370065617445, -0.3914196354184033) -- 
(1.161188983945953, -0.3868526393350996) -- 
(1.1649084424960312, -0.3821770847062224) -- 
(1.1684923243211818, -0.37739681546101034) -- 
(1.1719376829927497, -0.372515761618077) -- 
(1.1752416859665806, -0.36753793605441143) -- 
(1.1784016169117477, -0.3624674312062588) -- 
(1.18141487794374, -0.35730841570459176) -- 
(1.1842789917602683, -0.3520651309479401) -- 
(1.186991603677939, -0.346741887615396) -- 
(1.1895504835681168, -0.3413430621226607) -- 
(1.1919535276903879, -0.33587309302404755) -- 
(1.1941987604221163, -0.33033647736339883) -- 
(1.196284335882667, -0.32473776697691586) -- 
(1.1982085394509678, -0.31908156475094274) -- 
(1.1999697891751553, -0.3133725208377806) -- 
(1.2015666370731508, -0.30761532883264203) -- 
(1.202997770323094, -0.30181472191489017) -- 
(1.2042620123426586, -0.2959754689567343) -- 
(1.2053583237563583, -0.2901023706025812) -- 
(1.206285803250054, -0.2842002553222653) -- 
(1.207043688311952, -0.27827397544140275) -- 
(1.2076313558594922, -0.272328403152133) -- 
(1.2080483227516046, -0.2663684265075271) -- 
(1.2082942461859139, -0.2603989454029569) -- 
(1.2083689239805722, -0.25442486754772686) -- 
(1.2082722947404756, -0.24845110443028384) -- 
(1.2080044379077421, -0.24248256728031795) -- 
(1.2075655736963986, -0.2365241630310776) -- 
(1.206956062911335, -0.2305807902852156) -- 
(1.2061764066516745, -0.22465733528748522) -- 
(1.2052272458988051, -0.2187586679075953) -- 
(1.2041093609894056, -0.21288963763652835) -- 
(1.2028236709739044, -0.20705506959961203) -- 
(1.2013712328609003, -0.20125976058962325) -- 
(1.1997532407481573, -0.19550847512318478) -- 
(1.1979710248408986, -0.18980594152369779) -- 
(1.1960260503582012, -0.18415684803402882) -- 
(1.1939199163283891, -0.17856583896215006) -- 
(1.1916543542744191, -0.1730375108628982) -- 
(1.1892312267903393, -0.16757640875899452) -- 
(1.186652526009987, -0.1621870224044293) -- 
(1.1839203719691895, -0.15687378259328583) -- 
(1.1810370108628132, -0.15164105751703733) -- 
(1.1780048131980896, -0.14649314917331077) -- 
(1.1748262718457447, -0.14143428982907139) -- 
(1.1715039999905257, -0.13646863854113506) -- 
(1.1680407289828147, -0.13160027773686905) -- 
(1.164439306093095, -0.12683320985789287) -- 
(1.1607026921711152, -0.12217135406953673) -- 
(1.156833959211675, -0.11761854303876616) -- 
(1.1528362878290375, -0.11317851978321818) -- 
(1.1487129646420393, -0.10885493459394213) -- 
(1.1444673795720515, -0.10465134203437362) -- 
(1.1401030230560127, -0.1005711980180092) -- 
(1.1356234831768246, -0.09661785696718544) -- 
(1.1310324427134695, -0.09279456905529601) -- 
(1.1263336761132745, -0.08910447753471601) -- 
(1.1215310463888133, -0.08555061615262943) -- 
(1.1166285019419941, -0.0821359066568845) -- 
(1.1116300733179485, -0.07886315639392778) -- 
(1.1065398698913833, -0.0757350560007904) -- 
(1.1013620764881287, -0.07275417719302638) -- 
(1.0961009499446528, -0.06992297065041925) -- 
(1.090760815608376, -0.06724376400219671) -- 
(1.0853460637816557, -0.06471875991340899) -- 
(1.0798611461123757, -0.06235003427404442) -- 
(1.0743105719340957, -0.060139534492371444) -- 
(1.0686989045587796, -0.058089077893909054) -- 
(1.0630307575251439, -0.05620035022734317) -- 
(1.057310790805714, -0.05447490427861628) -- 
(1.0515437069757052, -0.052914158594330525) -- 
(1.0457342473468798, -0.05151939631551329) -- 
cycle; 
\coordinate (EdS0) at (0.5780972173844605, 0.21951530152848314);
\draw[styleEdS, draw=colorContourEdSnappe1, fill=colorInterieurEdSnappe1] 
(-0.15169069257764084, 0.5507561630088252) -- 
(-0.17504146443852367, 0.5360596714869722) -- 
(-0.1980509650850808, 0.521674771969309) -- 
(-0.22073415534445154, 0.5076006935349764) -- 
(-0.24310719678293558, 0.49383615486428445) -- 
(-0.2651874165214464, 0.48037937081071636) -- 
(-0.2869932690927314, 0.46722807665123334) -- 
(-0.3085442956789244, 0.4543795691229797) -- 
(-0.32986108091481187, 0.44183076319743253) -- 
(-0.35096520730549785, 0.42957826339985106) -- 
(-0.37187920718936585, 0.41761844835234146) -- 
(-0.3926265120789917, 0.40594756710319946) -- 
(-0.4132313991330905, 0.39456184570400504) -- 
(-0.4337189344495226, 0.3834576024096734) -- 
(-0.45411491281992367, 0.3726313698053583) -- 
(-0.47444579354746064, 0.3620800221069333) -- 
(-0.49473863189786793, 0.35180090583672363) -- 
(-0.5150210057287882, 0.34179197203978184) -- 
(-0.5353209368240557, 0.33205190817334773) -- 
(-0.5556668064512493, 0.3225802677669589) -- 
(-0.5760872646692548, 0.313377595905848) -- 
(-0.5966111329481912, 0.3044455485284712) -- 
(-0.6172672997411849, 0.29578700344381215) -- 
(-0.6380846087839275, 0.2874061608610278) -- 
(-0.6590917401143961, 0.27930863108194903) -- 
(-0.6803170841236117, 0.2715015068394799) -- 
(-0.7017886093907949, 0.2639934175814647) -- 
(-0.7235337256421194, 0.25679456281626867) -- 
(-0.7455791439158018, 0.24991672147641975) -- 
(-0.7679507369237274, 0.2433732341503716) -- 
(-0.790673403666179, 0.23717895501566963) -- 
(-0.8137709435620123, 0.23135017041918954) -- 
(-0.837265946664578, 0.22590448133181817) -- 
(-0.8611797078861451, 0.22086064739230282) -- 
(-0.8855321744698602, 0.21623839097508424) -- 
(-0.9103419371234759, 0.21205816068103941) -- 
(-0.9356262761334505, 0.20834085484647802) -- 
(-0.961401274258467, 0.20510750705228062) -- 
(-0.9876820080846747, 0.20237893711210087) -- 
(-1.0144828286234011, 0.20017537250377648) -- 
(-1.0212111055321056, 0.19966127297052094) -- 
(-1.0279217854342695, 0.19895361221776484) -- 
(-1.034609301063352, 0.19805297732998992) -- 
(-1.041268104370191, 0.1969601154855073) -- 
(-1.0478926711257381, 0.19567593333658903) -- 
(-1.0544775055040283, 0.194201496257298) -- 
(-1.0610171446415886, 0.19253802745963902) -- 
(-1.0675061631695015, 0.19068690697876728) -- 
(-1.0739391777143585, 0.18864967052809295) -- 
(-1.080310851364378, 0.18642800822523345) -- 
(-1.086615898096975, 0.18402376318987038) -- 
(-1.092849087164114, 0.18143893001467354) -- 
(-1.0990052474318048, 0.1786756531105611) -- 
(-1.105079271670143, 0.17573622492766872) -- 
(-1.1110661207903312, 0.17262308405350335) -- 
(-1.116960828025175, 0.1693388131898597) -- 
(-1.122758503049575, 0.16588613701017735) -- 
(-1.1284543360376047, 0.1622679198991166) -- 
(-1.134043601652803, 0.15848716357622827) -- 
(-1.1395216629683755, 0.15454700460568818) -- 
(-1.1448839753140478, 0.1504507117941632) -- 
(-1.1501260900463823, 0.146201683478967) -- 
(-1.1552436582394316, 0.14180344470875594) -- 
(-1.1602324342926624, 0.13725964431910287) -- 
(-1.1650882794531596, 0.13257405190537624) -- 
(-1.1698071652491915, 0.12775055469543542) -- 
(-1.1743851768322795, 0.12279315432473595) -- 
(-1.1788185162250082, 0.11770596351652171) -- 
(-1.1831035054718757, 0.11249320266985613) -- 
(-1.1872365896905732, 0.10715919635832517) -- 
(-1.1912143400211617, 0.1017083697423152) -- 
(-1.195033456470697, 0.09614524489784283) -- 
(-1.1986907706509484, 0.09047443706498239) -- 
(-1.2021832484069326, 0.08470065081900341) -- 
(-1.2055079923340877, 0.07882867616739407) -- 
(-1.2086622441819999, 0.07286338457600965) -- 
(-1.2116433871426804, 0.06680972492764123) -- 
(-1.2144489480215048, 0.06067271941635902) -- 
(-1.2170765992890076, 0.054457459381035425) -- 
(-1.2195241610118286, 0.048169101081504476) -- 
(-1.2217896026612163, 0.04181286142086249) -- 
(-1.223871044797579, 0.03539401361745781) -- 
(-1.225766760629692, 0.028917882830160894) -- 
(-1.2274751774472679, 0.022389841740544255) -- 
(-1.228994877925694, 0.01581530609563598) -- 
(-1.2303246013018665, 0.00919973021494658) -- 
(-1.2314632444201337, 0.002548602465495202) -- 
(-1.2324098626474882, -0.004132559291410428) -- 
(-1.2331636706572482, -0.010838212277863062) -- 
(-1.2337240430805723, -0.017562793397715527) -- 
(-1.2340905150252757, -0.024300723851797427) -- 
(-1.2342627824615098, -0.03104641376615843) -- 
(-1.2342407024739888, -0.03779426682950015) -- 
(-1.2340242933805552, -0.04453868493594776) -- 
(-1.2336137347169822, -0.05127407282931096) -- 
(-1.2330093670880287, -0.05799484274498075) -- 
(-1.2322116918848693, -0.06469541904561071) -- 
(-1.2312213708691342, -0.07137024284673765) -- 
(-1.2300392256239019, -0.0780137766285043) -- 
(-1.2286662368721049, -0.08462050882965623) -- 
(-1.2271035436629079, -0.09118495842000482) -- 
(-1.2253524424267392, -0.09770167944756064) -- 
(-1.2234143858997542, -0.10416526555656458) -- 
(-1.2212909819186275, -0.11057035447267138) -- 
(-1.2189839920866703, -0.11691163245156061) -- 
(-1.2164953303123824, -0.1231838386872879) -- 
(-1.213827061221648, -0.12938176967671738) -- 
(-1.2109813984448963, -0.13550028353641472) -- 
(-1.2079607027806447, -0.14153430426842012) -- 
(-1.2047674802369503, -0.14747882597136147) -- 
(-1.201404379952393, -0.15332891699341478) -- 
(-1.1978741919983158, -0.15907972402366605) -- 
(-1.1941798450641463, -0.1647264761184806) -- 
(-1.1903244040277179, -0.17026448865953914) -- 
(-1.1863110674126072, -0.17568916724025793) -- 
(-1.182143164734598, -0.18099601147736766) -- 
(-1.1778241537394727, -0.1861806187444887) -- 
(-1.1733576175344205, -0.1912386878246079) -- 
(-1.1687472616154468, -0.19616602247842244) -- 
(-1.1639969107932449, -0.20095853492559496) -- 
(-1.1591105060200841, -0.2056122492360284) -- 
(-1.154092101120346, -0.2101233046283487) -- 
(-1.1489458594274173, -0.21448795867285936) -- 
(-1.1436760503297374, -0.21870259039630863) -- 
(-1.1382870457288554, -0.22276370328589631) -- 
(-1.1327833164124483, -0.22666792819002524) -- 
(-1.1271694283452947, -0.23041202611339306) -- 
(-1.1214500388812925, -0.23399289090410488) -- 
(-1.1156298928996586, -0.23740755183057632) -- 
(-1.1097138188685158, -0.2406531760460909) -- 
(-1.1037067248391326, -0.2437270709389667) -- 
(-1.0976135943741416, -0.24662668636638074) -- 
(-1.0914394824131113, -0.24934961677000123) -- 
(-1.085189511078903, -0.2518936031716694) -- 
(-1.0788688654282916, -0.2542565350474768) -- 
(-1.072482789150376, -0.2564364520786833) -- 
(-1.0660365802163472, -0.25843154577802285) -- 
(-1.059535586484222, -0.26024016099004743) -- 
(-1.052985201262192, -0.261860797264266) -- 
(-1.0463908588342665, -0.26329211009993686) -- 
(-1.0397580299519174, -0.2645329120614833) -- 
(-1.0330922172954773, -0.26558217376360543) -- 
(-1.0263989509090414, -0.2664390247252709) -- 
(-1.0196837836126742, -0.26710275409187817) -- 
(-1.0129522863957163, -0.2675728112249897) -- 
(-1.0062100437950219, -0.2678488061591494) -- 
(-0.9994626492619538, -0.2679305099254024) -- 
(-0.9927157005219838, -0.2678178547412513) -- 
(-0.9618600215907757, -0.26718218439906705) -- 
(-0.9312831146272719, -0.2666713947766457) -- 
(-0.9009530059793759, -0.26625843546062694) -- 
(-0.8708379447198517, -0.2659148207309584) -- 
(-0.8409063796084735, -0.26561086314225923) -- 
(-0.8111270158339738, -0.2653158722753026) -- 
(-0.781468939498758, -0.2649983223125783) -- 
(-0.7519017969860202, -0.26462599327495656) -- 
(-0.7223960161469252, -0.26416609165852906) -- 
(-0.6929230564929207, -0.2635853567615401) -- 
(-0.6634556761362759, -0.26285015915721244) -- 
(-0.6339682039861398, -0.26192659754746617) -- 
(-0.6044368066058047, -0.2607805996498903) -- 
(-0.5748397401256944, -0.2593780318721067) -- 
(-0.5451575786645209, -0.25768482137646576) -- 
(-0.5153734118323504, -0.2556670928082199) -- 
(-0.48547300507696894, -0.25329132053369574) -- 
(-0.4554449178937359, -0.25052449579649216) -- 
(-0.4252805762499636, -0.24733430683318938) -- 
(-0.3949742969699929, -0.24368932877385877) -- 
(-0.36452326326741585, -0.23955921915536818) -- 
(-0.33392745206468954, -0.23491491415179303) -- 
(-0.30318951516469345, -0.2297288202133098) -- 
(-0.2723146176820055, -0.2239749957199047) -- 
(-0.241310238347933, -0.21762931749468523) -- 
(-0.21018593731769367, -0.2106696275580442) -- 
(-0.17895309788205088, -0.203075856293629) -- 
(-0.14762464898207767, -0.1948301191796247) -- 
(-0.1162147756228702, -0.18591678534310513) -- 
(-0.0847386241761405, -0.17632251734497642) -- 
(-0.05321200916732435, -0.16603628272319448) -- 
(-0.02165112749206638, -0.1550493388439899) -- 
(0.009927714853966221, -0.14335519347778738) -- 
(0.041508359462259836, -0.130949544186216) -- 
(0.07307503265108445, -0.11783020005372988) -- 
(0.1046125612256801, -0.1039969895135833) -- 
(0.13610656612738772, -0.08945165801084692) -- 
(0.16754363425536342, -0.07419775903601292) -- 
(0.19891146963183234, -0.058240541683307556) -- 
(0.2078176233741046, -0.053513397115755326) -- 
(0.21658463702996975, -0.04853294113247408) -- 
(0.22520530776875866, -0.04330326559216415) -- 
(0.23367255299267653, -0.03782866710811363) -- 
(0.2419794161557458, -0.032113643518173626) -- 
(0.25011907247918735, -0.026162890189410426) -- 
(0.25808483455854336, -0.01998129616047086) -- 
(0.2658701578579365, -0.013573940124830638) -- 
(0.27346864608695, -0.006946086258225556) -- 
(0.2808740564557121, -0.00010317989369318381) -- 
(0.28808030480386737, 0.006949156952220904) -- 
(0.29508147059921985, 0.014205130198417112) -- 
(0.3018718018019427, 0.021658778459561226) -- 
(0.3084457195903571, 0.029303977943858264) -- 
(0.3147978229443984, 0.03713444748425651) -- 
(0.32092289308300304, 0.04514375369894848) -- 
(0.3268158977517707, 0.053325316276928963) -- 
(0.33247199535738053, 0.06167241338426771) -- 
(0.3378865389453618, 0.070178187186655) -- 
(0.34305508001795404, 0.07883564948368263) -- 
(0.34797337218891744, 0.08763768745023151) -- 
(0.3526373746722923, 0.09657706948024913) -- 
(0.3570432556022406, 0.10564645112811495) -- 
(0.36118739518124277, 0.11483838114271346) -- 
(0.3650663886540617, 0.12414530758925614) -- 
(0.36867704910503296, 0.13355958405382418) -- 
(0.3720164100763803, 0.1430734759255331) -- 
(0.3750817280054082, 0.15267916675115878) -- 
(0.3778704844785657, 0.16236876465700348) -- 
(0.38038038830053345, 0.17213430883272657) -- 
(0.3826093773766308, 0.18196777607181205) -- 
(0.3845556204069979, 0.1918610873632996) -- 
(0.38621751839116114, 0.2018061145293642) -- 
(0.38759370594174447, 0.2117946869032897) -- 
(0.3886830524062495, 0.221818598042351) -- 
(0.38948466279597954, 0.23186961247008944) -- 
(0.38999787852134793, 0.2419394724424407) -- 
(0.39022227793296416, 0.2520199047321589) -- 
(0.3901576766680531, 0.2621026274259604) -- 
(0.38980412780192564, 0.2721793567288041) -- 
(0.3891619218043716, 0.2822418137697178) -- 
(0.3882315863010155, 0.2922817314035794) -- 
(0.38701388563982797, 0.30229086100326336) -- 
(0.3855098202631504, 0.3122609792365748) -- 
(0.3837206258857481, 0.3221838948223994) -- 
(0.3816477724795685, 0.3320514552605226) -- 
(0.3792929630660361, 0.34185555352958685) -- 
(0.3766581323168795, 0.35158813474768247) -- 
(0.37374544496463746, 0.36124120279010385) -- 
(0.3705572940241511, 0.3708068268588306) -- 
(0.3670962988265031, 0.38027714799833723) -- 
(0.3633653028670198, 0.3896443855523788) -- 
(0.3593673714691029, 0.398900843556447) -- 
(0.35510578926581204, 0.40803891706064477) -- 
(0.35058405750126476, 0.4170510983777851) -- 
(0.3458058911540741, 0.42592998325158) -- 
(0.34077521588518467, 0.434668276939853) -- 
(0.33549616481261524, 0.4432588002077763) -- 
(0.3299730751157592, 0.4516944952262091) -- 
(0.32421048447203105, 0.45996843137029075) -- 
(0.3182131273287865, 0.46807381091352634) -- 
(0.31198593101358096, 0.47600397461268296) -- 
(0.3055340116859604, 0.4837524071789122) -- 
(0.29886267013410983, 0.49131274263060226) -- 
(0.2919773874198163, 0.49867876952356016) -- 
(0.2848838203753203, 0.5058444360542305) -- 
(0.2775877969557575, 0.5128038550317544) -- 
(0.27009531145100973, 0.5195513087147872) -- 
(0.26241251956089684, 0.5260812535090974) -- 
(0.2545457333377575, 0.5323883245220917) -- 
(0.24650141600057368, 0.5384673399705187) -- 
(0.23828617662489873, 0.5443133054377355) -- 
(0.22990676471295257, 0.5499214179770362) -- 
(0.2213700646483436, 0.5552870700576722) -- 
(0.21268309003997596, 0.5604058533503213) -- 
(0.2038529779597865, 0.5652735623488971) -- 
(0.194886983079046, 0.569886197825722) -- 
(0.1857924717080448, 0.5742399701172248) -- 
(0.1765769157440554, 0.5783313022374652) -- 
(0.16724788653254827, 0.5821568328169241) -- 
(0.1578130486467014, 0.5857134188641493) -- 
(0.14828015359031646, 0.5889981383479833) -- 
(0.1386570334293139, 0.5920082925982559) -- 
(0.1289515943570395, 0.594741408522965) -- 
(0.11917181019866954, 0.597195240640128) -- 
(0.10932571586004947, 0.599367772922632) -- 
(0.0994214007263517, 0.6012572204545673) -- 
(0.08946700201597302, 0.6028620308976845) -- 
(0.0794706980951321, 0.6041808857667695) -- 
(0.06944070175866277, 0.6052127015128886) -- 
(0.059385253482519385, 0.6059566304136137) -- 
(0.049312614653541476, 0.6064120612694976) -- 
(0.03923106078203639, 0.6065786199062245) -- 
(0.02914887470276048, 0.6064561694820254) -- 
(0.019074339769881793, 0.6060448106001051) -- 
(0.009015733051514832, 0.6053448812259884) -- 
(-0.0010186814705782643, 0.6043569564098533) -- 
(-0.011020659690536047, 0.6030818478140791) -- 
(-0.020981984151616465, 0.6015206030463986) -- 
(-0.030894470797520404, 0.5996745047992016) -- 
(-0.04074997569627459, 0.5975450697956954) -- 
(-0.050540401731149445, 0.5951340475437916) -- 
(-0.06025770525311238, 0.5924434188987395) -- 
(-0.06989390268935552, 0.5894753944356885) -- 
(-0.07944107710246415, 0.5862324126335174) -- 
(-0.08889138469483893, 0.5827171378714212) -- 
(-0.09823706125302803, 0.5789324582399011) -- 
(-0.10747042852667375, 0.5748814831679595) -- 
(-0.11658390053683444, 0.570567540868444) -- 
(-0.12556998980849549, 0.5659941756036453) -- 
(-0.13442131352215392, 0.5611651447733902) -- 
(-0.14313059957941804, 0.5560844158280247) -- 
cycle; 
\coordinate (EdS1) at (-0.4337189344495226, 0.3834576024096734);
\end{scope}
\draw[styleEdS, draw=colorContourEdSarete, fill=colorInterieurarete] (0.03846156134313788, 0.25480769814250004) circle (0.3464951869522779);
\draw[stylePseudoEdS] (0.29657632742188633, 0.5040452822876815) arc (43.99759375651435:122.72148004933781:0.3588071986713943) node[above, pos=0.5, inner sep=2pt] {$\pseudoEdS{\crete}{\rho}$};
\path[decoration={text along path, raise={1ex}, text color=colorContourEdSarete, text={{$\implicitEdB{\crete}$} {$=$} {$0$}{}}, text align={center}}, decorate] (-0.21027863798930502, 0.0060674988100571925) arc (225:310:0.3517717634033278);
\draw[
    styleNappe,
    postaction={
        decoration={
            text along path, raise={1ex}, text={{$\Right{\nappe}$}{}}, text align=center, reverse path
        },
        decorate
    }
]
(1.0, -0.2548076981425002) .. controls (0.6394231211989716, -0.18750003123338313) and (0.33205293517056217, 0.00165690420180032) .. (0.03846156134313788, 0.25480769814250004) node[pos=0.35, above] (labelNappe0) {};
\node[colorContourEdSnappe0, inner sep=0.15\imagewidth, below] at (labelNappe0) {$\implicitEdB{\Right{\nappe}} < 0$};
\path[decoration={text along path, raise={1ex}, text color=colorContourEdSnappe0, text={{$\implicitEdB{\Right{\nappe}}$} {$=$} {$0$}{}}, text align={center}}, decorate] 
(0.291515253577205, 0.4991582708338544) -- 
(0.3117526549541225, 0.4786331276446601) -- 
(0.33179640986407155, 0.4583287883406225) -- 
(0.35165170397301176, 0.4382574078293584) -- 
(0.3713244271940277, 0.4184308703424265) -- 
(0.3908211857508439, 0.3988607436785092) -- 
(0.4101493139242216, 0.37955823745850226) -- 
(0.4293168853280881, 0.36053416626560775) -- 
(0.4483327234793507, 0.34179891862769246) -- 
(0.4672064113277807, 0.3233624328735336) -- 
(0.4859482992996188, 0.3052341809521778) -- 
(0.5045695112809123, 0.28742316133741974) -- 
(0.5230819478253321, 0.26993790213828667) -- 
(0.5414982857190578, 0.2527864754915883) -- 
(0.5598319728767698, 0.2359765242139565) -- 
(0.5780972173844605, 0.21951530152848314) -- 
(0.5963089693558062, 0.20340972444626615) -- 
(0.6144828941409042, 0.18766644106904284) -- 
(0.6326353353336946, 0.1722919116818077) -- 
(0.650783265984124, 0.1572925030243364) -- 
(0.6689442264517612, 0.1426745945736954) -- 
(0.68713624745879, 0.12844469504848804) -- 
(0.7053777571313378, 0.11460956667949797) -- 
(0.7236874711762205, 0.10117635410812229) -- 
(0.742084265838364, 0.08815271410883804) -- 
(0.7605870339290941, 0.07554694172716503) -- 
(0.7792145250047978, 0.06336808792773979) -- 
(0.7979851716958212, 0.05162606350851576) -- 
(0.816916905210558, 0.04033172390652845) -- 
(0.83602696412914, 0.029496929643137104) -- 
(0.8553317017006214, 0.019134577567845562) -- 
(0.8748463978997224, 0.009258598781243438) -- 
(0.8945850834066453, -0.00011607984793962633) -- 
(0.9145603833626653, -0.008973612328943836) -- 
(0.9347833891410695, -0.017297348539330626) -- 
(0.9552635663797575, -0.025069985986572035) -- 
(0.9760087070839042, -0.03227375493086215) -- 
(0.9970249326809821, -0.03889063165462239) -- 
(1.0183167534800655, -0.0449025730216131) -- 
(1.0398871880695557, -0.05029176412270403);
\draw[
    styleNappe,
    postaction={
        decoration={
            text along path, raise={1ex}, text={{$\Left{\nappe}$}{}}, text align=center, reverse path
        },
        decorate
    }
]
(0.03846156134313788, 0.25480769814250004) .. controls (-0.28894796328001215, 0.06631655134905) and (-0.5865384951744124, -0.014423064681421358) .. (-0.9999999999999998, -0.0336538334545584) node[pos=0.65, above] (labelNappe1) {};
\node[colorContourEdSnappe1, inner sep=0.15\imagewidth, below] at (labelNappe1) {$\implicitEdB{\Left{\nappe}} < 0$};
\path[decoration={text along path, raise={1ex}, text color=colorContourEdSnappe1, text={{$\implicitEdB{\Left{\nappe}}$} {$=$} {$0$}{}}, text align={center}}, decorate] 
(-1.0144828286234011, 0.20017537250377648) -- 
(-0.9876820080846747, 0.20237893711210087) -- 
(-0.961401274258467, 0.20510750705228062) -- 
(-0.9356262761334505, 0.20834085484647802) -- 
(-0.9103419371234759, 0.21205816068103941) -- 
(-0.8855321744698602, 0.21623839097508424) -- 
(-0.8611797078861451, 0.22086064739230282) -- 
(-0.837265946664578, 0.22590448133181817) -- 
(-0.8137709435620123, 0.23135017041918954) -- 
(-0.790673403666179, 0.23717895501566963) -- 
(-0.7679507369237274, 0.2433732341503716) -- 
(-0.7455791439158018, 0.24991672147641975) -- 
(-0.7235337256421194, 0.25679456281626867) -- 
(-0.7017886093907949, 0.2639934175814647) -- 
(-0.6803170841236117, 0.2715015068394799) -- 
(-0.6590917401143961, 0.27930863108194903) -- 
(-0.6380846087839275, 0.2874061608610278) -- 
(-0.6172672997411849, 0.29578700344381215) -- 
(-0.5966111329481912, 0.3044455485284712) -- 
(-0.5760872646692548, 0.313377595905848) -- 
(-0.5556668064512493, 0.3225802677669589) -- 
(-0.5353209368240557, 0.33205190817334773) -- 
(-0.5150210057287882, 0.34179197203978184) -- 
(-0.49473863189786793, 0.35180090583672363) -- 
(-0.47444579354746064, 0.3620800221069333) -- 
(-0.45411491281992367, 0.3726313698053583) -- 
(-0.4337189344495226, 0.3834576024096734) -- 
(-0.4132313991330905, 0.39456184570400504) -- 
(-0.3926265120789917, 0.40594756710319946) -- 
(-0.37187920718936585, 0.41761844835234146) -- 
(-0.35096520730549785, 0.42957826339985106) -- 
(-0.32986108091481187, 0.44183076319743253) -- 
(-0.3085442956789244, 0.4543795691229797) -- 
(-0.2869932690927314, 0.46722807665123334) -- 
(-0.2651874165214464, 0.48037937081071636) -- 
(-0.24310719678293558, 0.49383615486428445) -- 
(-0.22073415534445154, 0.5076006935349764) -- 
(-0.1980509650850808, 0.521674771969309) -- 
(-0.17504146443852367, 0.5360596714869722) -- 
(-0.15169069257764084, 0.5507561630088252);
\coordinate (Gamma) at (0.03846156134313788, 0.25480769814250004);
\fill[black] (Gamma) circle (1.2pt);
\node[above] at (Gamma) {$\crete$};

			\node[colorContourEdSarete] at ([shift={(120:0.2\imagewidth)}]Gamma) {$\implicitEdB{\crete} < 0$};
		\end{scope}
	\end{tikzpicture}
	\caption{Illustration de la notion de ``pseudo-EdS'' d'une crête $\crete$ adhérente aux nappes $\Left{\nappe}$ et $\Right{\nappe}$ (vue en coupe).}
	\label{fig:pseudo_EdS_arete_vue_coupe}
\end{figure}