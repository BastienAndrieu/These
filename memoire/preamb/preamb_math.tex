% Math packages, notations
\usepackage{amssymb}
\usepackage{amsmath}    % amsmath equation env has funny spacing with hyperref :-( so...
\let\equation\gather \let\endequation\endgather
\usepackage{amsfonts}
\usepackage{mathtools}
\usepackage{stmaryrd}
\usepackage{cases}
\usepackage{esint}


% Notations
\newcommand*{\determinant}[1]{\left\lvert {#1} \right\rvert}
\newcommand*{\inv}[1]{#1\raisebox{1.15ex}{$\scriptscriptstyle-\!1$}}
\newcommand*{\inverse}[1]{\ensuremath{{#1} ^{-1}}} % inverse matrix
\newcommand*{\transpose}[1]{\ensuremath{{#1} ^\mathsf{T}}} % transpose matrix
\newcommand*{\dotprod}[2]{ \transpose{#1} {#2} } % dot product
\newcommand*{\crossprod}[2]{ {#1} \times {#2} } % cross product
\newcommand*{\normtwo}[1]{ \left\| {#1} \right\| } % Euclidean norm
\newcommand*{\vit}[1]{\boldsymbol{#1}} % greek letter vector
\newcommand*{\vrm}[1]{\mathbf{#1}} % latin letter vector
\newcommand*{\unitized}[1]{ \frac{#1}{\normtwo{#1}} }

\DeclareMathOperator*{\argmin}{\arg\!\min} % argmin

\newcommand*{\bigO}[1]{O\!\left( {#1} \right)} % big-O
\newcommand*{\littleo}[1]{o\!\left( {#1} \right)} % little-o


\newcommand*{\bx}{\vit{x}}
\newcommand*{\bd}{\vit{\delta}}
\newcommand*{\bg}{\vit{\gamma}}
\newcommand*{\bl}{\vit{\lambda}}
\renewcommand*{\bs}{\vit{\sigma}} % originally backslash in typewriter font

\newcommand*{\unv}{\vrm{n}} % unit normal vector
\newcommand*{\nv}{\hat{\unv}} % normal vector

\newcommand{\colvec}[1]{\begin{pmatrix} #1 \end{pmatrix}} % vector printed in column
\newcommand{\rowvec}[1]{ \transpose{\begin{pmatrix} #1 \end{pmatrix}} } % vector printed in row (transposed column)

% Differential operators
\newcommand*{\jacobian}[1]{\mathbf{J}_{#1}}
\newcommand*{\gradient}[1]{\nabla \! {#1}}
\newcommand*{\hessian}[1]{\mathbf{H}_{#1}}
\newcommand*{\divergence}[1]{\nabla \cdot {#1}}%\mathrm{div}{#1}}



% Condition number
\newcommand*{\cond}[1]{\mathrm{cond} \! \left( {#1} \right)}

% Differential geometry
\newcommand{\fff}{\mathbf{I}}    % first fundamental form
\newcommand{\sff}{\mathbf{I\!I}} % second fundamental form




%% Chebyshev polynomials
% Reference interval
\newcommand*{\chebinterval}{\left[ -1, 1 \right]}
\newcommand*{\series}[1]{S{#1}}
\newcommand*{\truncseries}[2]{P_{#2}{#1}}
\newcommand*{\interpolant}[2]{I_{#2}{#1}}



%% System of equations
\newenvironment{eqsys}
{\left\lbrace\begin{array}{@{}l@{}}}
{\end{array}\right.}
%ex :
%\begin{equation}
%	\begin{eqsys}
%		y_1 = a_1 x + b_1 \\
%		y_2 = a_2 x + b_2
%	\end{eqsys}
%\end{equation}








% Encircled text
\newcommand{\pgftextcircled}[1]{                                                                    
    \setbox0=\hbox{#1}%
    \dimen0\wd0%
    \divide\dimen0 by 2%
    \begin{tikzpicture}[baseline=(a.base)]%
        \useasboundingbox (-\the\dimen0,0pt) rectangle (\the\dimen0,1pt);
        \node[circle,draw,outer sep=0pt,inner sep=0.1ex] (a) {#1};
    \end{tikzpicture}
}

