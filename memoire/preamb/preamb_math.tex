% Math packages, notations
\usepackage{amsthm}
\usepackage{amssymb}
\usepackage{amsmath}    % amsmath equation env has funny spacing with hyperref :-( so...
\let\equation\gather \let\endequation\endgather
\usepackage{amsfonts}
\usepackage{mathtools}
\usepackage{stmaryrd}
\usepackage{cases}
\usepackage{esint}
\usepackage[overload]{empheq}%https://tex.stackexchange.com/questions/165685/subequations-and-array-in-braces

\usepackage{mathrsfs}% script (mathscr}

% *** TWEAK (NEGATIVE) SPACES *** 
%https://tex.stackexchange.com/questions/67912/large-negative-spaces
%https://tex.stackexchange.com/questions/9091/what-is-the-right-way-to-use-the-spacing-command/9092

%https://tex.stackexchange.com/questions/2783/bold-calligraphic-typeface
\DeclareMathAlphabet\matholdcal{OMS}{cmsy}{m}{n}

%https://tex.stackexchange.com/questions/247531/how-to-use-boondox-calligraphic-font-in-latex-without-replacing-mathcal-command/247819
%https://ctan.org/pkg/mathalfa
\usepackage[
scr=boondoxo, scrscaled=1.05, 
cal=zapfc, calscaled=1.18%,
%frak=mma, frakscaled=1.,
]{mathalfa}

%https://tex.stackexchange.com/questions/234676/new-theorem-style
%https://en.wikibooks.org/wiki/LaTeX/Theorems
\newtheoremstyle%
   {mythm}% name of the style to be used
   {\topsep}% measure of space to leave above the theorem. E.g.: 3pt
   {\topsep}% measure of space to leave below the theorem. E.g.: 3pt
   {\itshape}% font to use in the body of the theorem
   {0pt}% measure of space to indent
   {\bfseries\scshape}% head font
   {~---}% punctuation between head and body
   { }%\newline}% space after theorem head; " " = normal interword space
   {}%{\thmname{#1}\thmnumber{ #2}\thmnote{ (#3)}}% Manually specify head
\theoremstyle{mythm}

\newtheorem{theoreme}{Théorème}[chapter]
\newtheorem{propriete}{Propriété}[chapter]
\newtheorem{definition}{Définition}[chapter]


% Notations
\newcommand*{\determinant}[1]{\left\lvert {#1} \right\rvert}
\newcommand*{\inv}[1]{#1\raisebox{1.15ex}{$\scriptscriptstyle-\!1$}}
\newcommand*{\inverse}[1]{\ensuremath{{#1} ^{-1}}} % inverse matrix
\newcommand*{\transpose}[1]{\ensuremath{{#1} ^\mathsf{T}}} % transpose matrix
\newcommand*{\dotprod}[2]{ \transpose{#1} {#2} } % dot product
\newcommand*{\crossprod}[2]{ {#1} \times {#2} } % cross product
\newcommand*{\normtwo}[1]{ \left\| {#1} \right\| } % Euclidean norm
\newcommand*{\norminf}[1]{ \left\| {#1} \right\|_{\infty} } % infinity norm
\newcommand*{\vit}[1]{\boldsymbol{#1}} % greek letter vector
\newcommand*{\vrm}[1]{\mathbf{#1}} % latin letter vector
\newcommand*{\unitized}[1]{ \frac{#1}{\normtwo{#1}} }
\newcommand*{\scalprod}[2]{\left \langle {#1}, {#2} \right \rangle}
\newcommand*{\modulo}[1]{\pmod{#1}}%{\left[ {#1} \right]}

\DeclareMathOperator*{\argmin}{\arg\!\min} % argmin

\newcommand*{\bigO}[1]{O\!\left( {#1} \right)} % big-O
\newcommand*{\littleo}[1]{o\!\left( {#1} \right)} % little-o

% sets
\newcommand*{\reals}{\ensuremath{\mathbb{R}}}
\newcommand*{\integers}{\ensuremath{\mathbb{N}}}
\newcommand{\family}[4]{\left\{{#1}_{#2}\right\}_{{#2}={#3},\ldots,{#4}}}
\newcommand{\ffamily}[7]{\left\{{#1}_{#2,#5}\right\}_{\substack{{#2}={#3},\ldots,{#4}\\ \substack{{#5}={#6},\ldots,{#7}}}}}
\newcommand{\polyspace}[1]{\Pi_{#1}}%\mathbb{R}_{#1}[x]}
\newcommand{\Ltwospace}{L^2\chebinterval}%\mathcal{L}_2\chebinterval}
\newcommand*{\contdiff}[1]{\ensuremath{C^{#1}}}%{\matholdcal{C}^{#1}\!}
\newcommand*{\contgeom}[1]{\ensuremath{G^{#1}}}%{\matholdcal{G}^{#1}\!}

\newcommand*{\bx}{\vit{x}}
\newcommand*{\bu}{\vit{u}}
\newcommand*{\bd}{\vit{\delta}}
\newcommand*{\bg}{\vit{\gamma}}
\newcommand*{\bgw}{\bg_{\mkern-2muw}}
\newcommand*{\bp}{\vit{\psi}}
\newcommand*{\bpw}{\bp_{\mkern-2muw}}
\newcommand*{\bl}{\vit{\lambda}}
\renewcommand*{\bs}{\vit{\sigma}} % originally backslash in typewriter font
\newcommand*{\bsu}{\bs_{\mkern-3muu}}
\newcommand*{\bsv}{\bs_{\mkern-2muv}}


\newcommand*{\unv}{\vrm{n}} % unit normal vector
\newcommand*{\nv}{\hat{\unv}} % normal vector

\newcommand*{\eos}{\vit{\phi}} % envelope of spheres

\newcommand{\colvec}[1]{\begin{pmatrix} #1 \end{pmatrix}} % vector printed in column
\newcommand{\rowvec}[1]{ \transpose{\begin{pmatrix} #1 \end{pmatrix}} } % vector printed in row (transposed column)

% Differential operators
\newcommand*{\dfdx}[2]{\frac{\mathrm{d}}{\mathrm{d}{#2}}{#1}}
\newcommand*{\dkfdxk}[3]{\frac{\mathrm{d}^{#3}}{\mathrm{d}{#2}^{#3}}{#1}}
\newcommand*{\deriv}[2]{{#1}^{({#2})}}
\newcommand*{\prim}[2]{{#1}^{(\raisebox{0.2ex}{$\scriptscriptstyle-\!$}{#2})}}
\newcommand*{\jacobian}[1]{\mathbf{J}_{#1}}
\newcommand*{\gradient}[1]{\nabla \! {#1}}
\newcommand*{\hessian}[1]{\mathbf{H}_{#1}}
\newcommand*{\divergence}[1]{\nabla \cdot {#1}}%\mathrm{div}{#1}}
\newcommand*{\dx}[1]{\mathrm{d}{#1}}



% Condition number
\newcommand*{\cond}[1]{\mathrm{cond} \! \left( {#1} \right)}

% Differential geometry
\newcommand{\fff}{\mathbf{I}}     % first fundamental form
\newcommand{\sffc}{I\mkern-4.8muI}          % second fundamental form coefs
\newcommand{\sff}{\mathbf{I\!I}} % second fundamental form




%% Chebyshev polynomials
% Reference interval
\newcommand*{\chebinterval}{\ensuremath{\left[ -1, 1 \right]}}
\newcommand*{\chebopeninterval}{\ensuremath{\left] -1, 1 \right[}}
\newcommand*{\series}[1]{S{#1}}
\newcommand*{\truncseries}[2]{P_{#2}{#1}}
\newcommand*{\interpolant}[2]{I_{#2}{#1}}
\newcommand*{\interpderiv}[3]{D_{#2}^{(#3)}{\!#1}}%\mathcal{D}
%\newcommand*{\interpprim}[2]{J_{#2}{#1}}
\newcommand*{\interpprim}[2]{D_{#2}^{(\raisebox{0.2ex}{$\scriptscriptstyle-\!$}1)}{\!#1}}

% Reference interval for Bernstein polynomials
\newcommand*{\berninterval}{\left[ 0, 1 \right]}

%% System of equations
\newenvironment{eqsys}
{\left\lbrace\begin{array}{@{}l@{}}}
{\end{array}\right.}
%ex :
%\begin{equation}
%	\begin{eqsys}
%		y_1 = a_1 x + b_1 \\
%		y_2 = a_2 x + b_2
%	\end{eqsys}
%\end{equation}






% Encircled text
\newcommand{\pgftextcircled}[1]{                                                                    
    \setbox0=\hbox{#1}%
    \dimen0\wd0%
    \divide\dimen0 by 2%
    \begin{tikzpicture}[baseline=(a.base)]%
        \useasboundingbox (-\the\dimen0,0pt) rectangle (\the\dimen0,1pt);
        \node[circle,draw,outer sep=0pt,inner sep=0.1ex] (a) {#1};
    \end{tikzpicture}
}

