% Pseudo-code

\usepackage{algorithm}% http://ctan.org/pkg/algorithms
\usepackage{algpseudocode}% http://ctan.org/pkg/algorithmicx

%https://tex.stackexchange.com/questions/1375/what-is-a-good-package-for-displaying-algorithms#1376
%https://tex.stackexchange.com/questions/74776/package-algorithmic-in-french
\algrenewcommand\alglinenumber[1]{{\color{gray}#1}}
\algrenewcommand{\algorithmiccomment}[1]{\ \textit{\color{gray}// #1}}%{\hskip3em$\rightarrow$ #1}

\floatname{algorithm}{Algorithme}
\renewcommand{\algorithmicreturn}{\textbf{retourner}}
\renewcommand{\algorithmicprocedure}{\textbf{procédure}}
\newcommand{\Not}{\textbf{non}\ }
\newcommand{\And}{\textbf{et}\ }
\newcommand{\Or}{\textbf{ou}\ }
\renewcommand{\algorithmicrequire}{\textbf{Entrée:}}
\renewcommand{\algorithmicensure}{\textbf{Sortie:}}
%\renewcommand{\algorithmiccomment}[1]{\{#1\}}
\renewcommand{\algorithmicend}{\textbf{fin}}
\renewcommand{\algorithmicif}{\textbf{si}}
\renewcommand{\algorithmicthen}{\textbf{alors}}
\renewcommand{\algorithmicelse}{\textbf{sinon}}
\renewcommand{\algorithmicfor}{\textbf{pour}}
\renewcommand{\algorithmicforall}{\textbf{pour tout}}
\renewcommand{\algorithmicdo}{}%\textbf{faire}}
\renewcommand{\algorithmicwhile}{\textbf{tant que}}
\renewcommand{\algorithmicrepeat}{\textbf{répéter}}
\renewcommand{\algorithmicuntil}{\textbf{jusqu'à}}
\newcommand{\algorithmicelsif}{\algorithmicelse, \algorithmicif}
\newcommand{\algorithmicendif}{\algorithmicend\ \algorithmicif}
\newcommand{\algorithmicendfor}{\algorithmicend\ \algorithmicfor}
    

