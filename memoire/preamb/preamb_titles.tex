\newboolean{titlevbar}
\setboolean{titlevbar}{true}


\def\titlefont{\rmfamily}%


% Table of contents style
\renewcommand*{\cftchapterfont}{\titlefont\large\scshape}%\bfseries}
\renewcommand*{\cftchapterformatpnum}{\titlefont\large\normalfont}%\bfseries}
%\renewcommand{\cftdot}{\ensuremath{\cdot}}
\renewcommand*{\cftdotsep}{\cftnodots} % no dotted lines in ToC
\renewcommand*{\cftparskip}{1pt}
\setlength{\cftbeforechapterskip}{\onelineskip}

\maxtocdepth{subsection}

% Titles
\colorlet{colchapttl}{black}%mydarkgray}%DodgerBlue3}%



\makeatletter
\makechapterstyle{mychapterstyle}{
\setlength{\beforechapskip}{0pt}%40pt}
\setlength{\midchapskip}{0pt}%25pt}
\newlength{\afterchapskipdef}
\newlength{\afterchapskipxtra}
\setlength{\afterchapskipdef}{8\onelineskip}%60pt}%

\newif\ifNoChapNumber
\newlength{\barheight}
\newlength{\barlength}
\newlength{\ttltopskip}
\setlength{\ttltopskip}{60mm}
\setlength{\barlength}{0.37\foremargin}
\setlength{\barheight}{14mm}

\newlength{\numberheight}
\setlength{\numberheight}{1.63\barheight}


\newlength{\yshiftchapname}
\newlength{\yshiftchapnum}
\setlength{\yshiftchapname}{-\ttltopskip + 10mm + 3.9mm}
\setlength{\yshiftchapnum}{-\ttltopskip - 0.1mm}

\newlength{\vbaryshift}
\setlength{\vbaryshift}{1.25mm}
\newlength{\vbarxshift}
\setlength{\vbarxshift}{2.5mm}
\newlength{\vbarthck}
\setlength{\vbarthck}{0.6pt}
\newlength{\vbartipthck}
\setlength{\vbartipthck}{0.15pt}

\renewcommand{\chapnamefont}{\scshape\titlefont}
\renewcommand{\chapnumfont}{\titlefont\normalfont\fontsize{\numberheight}{0mm}\selectfont}
\renewcommand{\chaptitlefont}{\titlefont\scshape\huge}%
\renewcommand{\printchaptername}{}%\chapnamefont\@chapapp}
\renewcommand{\chapternamenum}{}
\renewcommand{\printchapternum}{}


\renewcommand\printchaptertitle[1]{
	\begin{tikzpicture}[remember picture,overlay]
		\node[yshift=-\ttltopskip] at (current page.north east) {%
			\begin{tikzpicture}[remember picture, overlay]%
				\ifNoChapNumber%
                    \node[%
					anchor= east,%
					text width=\textwidth,%
					minimum height=\barheight,%
					align=flush right,%justify,%
					inner sep=0mm%
					]%
					(chapttl)%
					at ([xshift=-\foremargin, yshift=\yshiftchapname] current page.north east)%
					{{\chaptitlefont\color{colchapttl}##1\par}};
				\else%
                    \node[%
					anchor= east,%
					text width=\textwidth-\vbarxshift,%
					minimum height=\barheight,%
					align=flush right,%justify,%
					inner sep=0mm%
					]%
					(chapttl)%
					at ([xshift=-\foremargin-\vbarxshift, yshift=\yshiftchapname] current page.north east)%
					{{\chaptitlefont\color{colchapttl}##1\par}};%
					\node[%
						anchor= west,%west,%
						align=left,%
						xshift=2\vbarxshift,%
						inner sep=0mm%
						]%
						(chapnum)% 
						at (chapttl.east)%(chapttl.east)%
						{\color{colchapttl}\chapnumfont\thechapter};%
					%
\coordinate (a) at ([xshift=\vbarxshift+\vbartipthck,yshift=-\vbaryshift] chapttl.south east);
\coordinate (b) at ([xshift=\vbarxshift+\vbarthck]chapttl.east);
\coordinate (c) at ([xshift=\vbarxshift+\vbartipthck,yshift=+\vbaryshift] chapttl.north east);
\coordinate (d) at ([xshift=\vbarxshift-\vbartipthck,yshift=+\vbaryshift] chapttl.north east);
\coordinate (e) at ([xshift=\vbarxshift-\vbarthck]chapttl.east);
\coordinate (f) at ([xshift=\vbarxshift-\vbartipthck,yshift=-\vbaryshift] chapttl.south east);
%
\path [fill=colchapttl]%
(a) [in=0, out=90] to %
[in=270, out=90] (b) to [in=270, out=90] %
(c) -- %
(d) [in=180, out=270] to %
[in=90, out=270] (e) to [in=90, out=270] %
(f) -- cycle;
				\fi
                \gettikzxy{(chapttl.south west)}{\bx}{\by}%
				\global\afterchapskipxtra=-\by%
				\pgfmathsetlength{\afterchapskip}{\afterchapskipdef + \afterchapskipxtra}%
				\global\afterchapskip=\afterchapskip%
			\end{tikzpicture}%
		};%
	\end{tikzpicture}%
}
\renewcommand\printchapternonum{\NoChapNumbertrue}
}
\makeatother

% helper (check if proper space after chapter title)
\newcommand{\printskip}{%
%\textbf{\the\afterchapskipdef~$+$~\the\afterchapskipxtra~$=$~\the\afterchapskip}%
}

\makeatletter
\newcommand{\printchapapp}{}%This is a \MakeLowercase{\@chapapp}.}%
\makeatother


\chapterstyle{mychapterstyle}%veelo}%











%% Format chapter abstracts
\def\abstractname{}
\def\abstitleskip{0pt}



%% Sections
%\setsecheadstyle{\color{colchapttl}\titlefont\scshape\Large}%\bfseries}%\raggedright}
\setsecheadstyle{\color{colchapttl}\titlefont\scshape\Large\bfseries}%\raggedright}
\setbeforesecskip{-2\onelineskip}
\setaftersecskip{\onelineskip}
\setsecindent{0pt}

%% Subsections
%\setsubsecheadstyle{\color{colchapttl}\titlefont\large\bfseries\sethangfrom{\noindent ##1}\raggedright}
\setsubsecheadstyle{\color{colchapttl}\titlefont\large\scshape\sethangfrom{\noindent ##1}\raggedright}
\setbeforesubsecskip{-\onelineskip}
\setaftersubsecskip{\onelineskip}
\setsubsecindent{0pt}

%% Subsubsections
\setsubsubsecheadstyle{\color{colchapttl}\titlefont\bfseries\sethangfrom{\noindent ##1}\raggedright}
%\setbeforesubsubsecskip{-\onelineskip}
%\setaftersubsubsecskip{\onelineskip}
%\setsubsubsecindent{0pt}

%% Paragraphs
\setparaheadstyle{\bfseries\sethangfrom{\noindent ##1}\raggedright}
%\setbeforeparaskip{-\onelineskip}
%\setafterparaskip{-1em}
%\setparaindent{0pt}


%https://tex.stackexchange.com/questions/261711/how-to-indent-first-paragraph-after-section-using-texstudio
\usepackage{indentfirst}