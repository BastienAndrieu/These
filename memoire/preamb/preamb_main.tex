\usepackage[utf8]{inputenc}
\usepackage[T1]{fontenc}


% POLICES
\usepackage{lmodern} % lmodern pour le mode math...
\edef\oldtt{\ttdefault} % et le mode typewriter
\usepackage[oldstyle, proportional]{libertine} % libertine comme police principale
\renewcommand{\ttdefault}{\oldtt}
\usepackage[scaled=0.875]{helvet} % helvetica comme police sans serif
\usepackage{slantsc} % slanted small caps

\usepackage{hyphenat}
\usepackage{enumitem}

% preambules
% Page layout (margins)
%\usepackage[margin=28mm,includeheadfoot,bindingoffset=5mm]{geometry}

\settrimmedsize{297mm}{210mm}{*}
\setlength{\trimtop}{0pt} 
\setlength{\trimedge}{\stockwidth} 
\addtolength{\trimedge}{-\paperwidth} 

% si no de page en bas
% \setlrmarginsandblock{3.25cm}{*}{*} % spine, edge, edge-to-spine ratio
% \setulmarginsandblock{3.0cm}{*}{1.25} % upper, lower, lower-to-upper ratio
% \setheadfoot{\onelineskip}{2\onelineskip}
% \checkandfixthelayout 


% si no de page en haut
\setlrmarginsandblock{3.2cm}{*}{*} % spine, edge, edge-to-spine ratio
\setulmarginsandblock{3.0cm}{*}{} % upper, lower, lower-to-upper ratio
\setheadfoot{\onelineskip}{\onelineskip}
\checkandfixthelayout 


% Show page margins, \usepackage[noframe]{showframe} to disable
\usepackage[noframe]{showframe}
%https://tex.stackexchange.com/questions/328662/showframe-how-to-make-colored-lines-of-page-layout
\renewcommand*\ShowFrameColor{\color{red}} % marges
\usepackage{color,xcolor}

%\colorlet{mydarkgray}{black!84}
\definecolor{mydarkgray}{HTML}{212131}%111122}

\definecolor{DodgerBlue3}{rgb}{0.094, 0.455, 0.804}
\definecolor{bleuONERA1}{rgb}{0.09,0.40,1.00} % Bleu du logo Onera
\definecolor{bleuONERA2}{rgb}{0.0,0.33,0.66}  % Bleu de la gauche du bandeau haut
\definecolor{bleuONERA3}{rgb}{0.85,0.90,0.95} % Bleu de la droite du bandeau haut

\definecolor{deepblue}{rgb}{0.0, 0.212, 0.596}

\definecolor{myred}{rgb}{0.8431,0.1882,0.1529}
\definecolor{mygreen}{rgb}{0.2118,0.6118,0.0235}


\definecolor{redlink}{HTML}{CD463C}
\definecolor{blulink}{HTML}{3C5FCC}

\definecolor{greenlink}{HTML}{198C32}



%%%%%%%%%
\definecolor{sunset01}{HTML}{364B9A}
\definecolor{sunset02}{HTML}{4A7BB7}
\definecolor{sunset03}{HTML}{6EA6CD}
\definecolor{sunset04}{HTML}{98CAE1}
\definecolor{sunset05}{HTML}{C2E4EF}
\definecolor{sunset06}{HTML}{EAECCC}
\definecolor{sunset07}{HTML}{FEDA8B}
\definecolor{sunset08}{HTML}{FDB366}
\definecolor{sunset09}{HTML}{F67E4B}
\definecolor{sunset10}{HTML}{DD3D2D}
\definecolor{sunset11}{HTML}{A50026} % couleurs prédéfinies
% Pseudo-code

\usepackage{algorithm}% http://ctan.org/pkg/algorithms
\usepackage{algpseudocode}% http://ctan.org/pkg/algorithmicx

%https://tex.stackexchange.com/questions/1375/what-is-a-good-package-for-displaying-algorithms#1376
%https://tex.stackexchange.com/questions/74776/package-algorithmic-in-french
\algrenewcommand\alglinenumber[1]{{\color{gray}#1}}
\algrenewcommand{\algorithmiccomment}[1]{\ \textit{\color{gray}// #1}}%{\hskip3em$\rightarrow$ #1}

\floatname{algorithm}{Algorithme}
\renewcommand{\algorithmicreturn}{\textbf{retourner}}
\renewcommand{\algorithmicprocedure}{\textbf{procédure}}
\newcommand{\Not}{\textbf{non}\ }
\newcommand{\And}{\textbf{et}\ }
\newcommand{\Or}{\textbf{ou}\ }
\renewcommand{\algorithmicrequire}{\textbf{Entrée:}}
\renewcommand{\algorithmicensure}{\textbf{Sortie:}}
%\renewcommand{\algorithmiccomment}[1]{\{#1\}}
\renewcommand{\algorithmicend}{\textbf{fin}}
\renewcommand{\algorithmicif}{\textbf{si}}
\renewcommand{\algorithmicthen}{\textbf{alors}}
\renewcommand{\algorithmicelse}{\textbf{sinon}}
\renewcommand{\algorithmicfor}{\textbf{pour}}
\renewcommand{\algorithmicforall}{\textbf{pour tout}}
\renewcommand{\algorithmicdo}{}%\textbf{faire}}
\renewcommand{\algorithmicwhile}{\textbf{tant que}}
\renewcommand{\algorithmicrepeat}{\textbf{répéter}}
\renewcommand{\algorithmicuntil}{\textbf{jusqu'à}}
\newcommand{\algorithmicelsif}{\algorithmicelse, \algorithmicif}
\newcommand{\algorithmicendif}{\algorithmicend\ \algorithmicif}
\newcommand{\algorithmicendfor}{\algorithmicend\ \algorithmicfor}
    


%https://tex.stackexchange.com/questions/50747/options-for-appearance-of-links-in-hyperref
\usepackage[pagebackref]{hyperref} % hyper link to page where reference is cited
\renewcommand*{\backrefsep}{, }
\renewcommand*{\backreflastsep}{ et }
\renewcommand*{\backreftwosep}{ et }
\renewcommand*{\backref}[1]{}
\renewcommand*{\backrefalt}[4]{%
    \ifcase #1 {\color{red}(Pas cité)}%\relax%
    \or        (\cf p.~#2)%
    \else      (\cf pp.~#2)%
    \fi}

\hypersetup{
	breaklinks,
	colorlinks=true,
	linkcolor=DodgerBlue3,
	anchorcolor=myred,%DarkRed,%
    citecolor=DodgerBlue3,%redlink,%black,%mygreen,
	pdfdisplaydoctitle=true,
	pdfpagemode=UseOutlines,%
	bookmarksnumbered=true,
	bookmarksopen=true,
	hypertexnames=true,
	linktoc=all
}
    
\usepackage{memhfixc} % Must be used on memoir document class after hyperref (?)



% Adding package bookmark improves bookmarks handling.
% More features and faster updated bookmarks.
%https://tex.stackexchange.com/questions/65544/how-to-link-table-of-contents-in-thesis-pdf
\usepackage{bookmark}


\renewcommand*{\sectionrefname}{Section~}
\renewcommand*{\chapterrefname}{Chapitre~}
\renewcommand*{\partrefname}{Partie~}
\renewcommand*{\appendixrefname}{Appendice~}
\newcommand{\algorithmautorefname}{Algorithme} % format des liens et références croisées
\usepackage{graphicx}


\usepackage{tikz}
\usetikzlibrary{calc}

\makeatletter
\newcommand{\gettikzxy}[3]{%
  \tikz@scan@one@point\pgfutil@firstofone#1\relax
  \edef#2{\the\pgf@x}%
  \edef#3{\the\pgf@y}%
}

\newcommand{\globalgettikzxy}[3]{%
  \tikz@scan@one@point\pgfutil@firstofone#1\relax
  \edef\@tempdima{\the\pgf@x}%
  \edef\@tempdimb{\the\pgf@y}%
  \global#2=\@tempdima%
  \global#3=\@tempdimb%
}

\newcommand\getwidthofnode[2]{%
    \pgfextractx{\pgf@xb}{\pgfpointanchor{#2}{east}}%
    \pgfextractx{\pgf@xa}{\pgfpointanchor{#2}{west}}% 
    \pgfmathsetlength{\pgf@xb}{\pgf@xb - \pgf@xa}%
	\global#1=\pgf@xb%
}

\newcommand\getheightofnode[2]{%
    \pgfextracty{\pgf@yb}{\pgfpointanchor{#2}{north}}%
    \pgfextracty{\pgf@ya}{\pgfpointanchor{#2}{south}}% 
    \pgfmathsetlength{\pgf@yb}{\pgf@yb - \pgf@ya}%
	\global#1=\pgf@yb%
}
\makeatother




% chemin(s) des figures
\graphicspath{{./figures/}}%,{../fig2/}}


% controle des objets flottants (figures, tables)
\renewcommand{\topfraction}{0.7}     % autorise 70% page de graphique en haut
\renewcommand{\bottomfraction}{0.5}  % autorise 50% page de graphique en bas
\renewcommand{\floatpagefraction}{0.7}
\renewcommand{\textfraction}{0.1} % TikZ, PGF, PStricks
% Math packages, notations
\usepackage{amsthm}
\usepackage{amssymb}
\usepackage{amsmath}    % amsmath equation env has funny spacing with hyperref :-( so...
\let\equation\gather \let\endequation\endgather
\usepackage{amsfonts}
\usepackage{bm}
\usepackage{mathtools}
\usepackage{stmaryrd}

%https://tex.stackexchange.com/questions/165685/subequations-and-array-in-braces
\usepackage[overload]{empheq}

\usepackage{cases}

\usepackage{esint}
\usepackage[overload]{empheq}%https://tex.stackexchange.com/questions/165685/subequations-and-array-in-braces

\usepackage{xifthen}% provides \isempty test
\usepackage{xparse}
\usepackage{xstring}

\usepackage{mathrsfs}% script (mathscr}

% *** TWEAK (NEGATIVE) SPACES *** 
%https://tex.stackexchange.com/questions/67912/large-negative-spaces
%https://tex.stackexchange.com/questions/9091/what-is-the-right-way-to-use-the-spacing-command/9092

%https://tex.stackexchange.com/questions/2783/bold-calligraphic-typeface
\DeclareMathAlphabet\matholdcal{OMS}{cmsy}{m}{n}

%https://tex.stackexchange.com/questions/247531/how-to-use-boondox-calligraphic-font-in-latex-without-replacing-mathcal-command/247819
%https://ctan.org/pkg/mathalfa
\usepackage[
scr=boondoxo, scrscaled=1.05, 
cal=zapfc, calscaled=1.18%,
%frak=mma, frakscaled=1.,
]{mathalfa}


%https://tex.stackexchange.com/questions/399620/how-can-i-get-a-bar-over-a-subscript
\usepackage{widebar}


%https://tex.stackexchange.com/questions/234676/new-theorem-style
%https://en.wikibooks.org/wiki/LaTeX/Theorems
\newtheoremstyle%
   {mythm}% name of the style to be used
   {\topsep}% measure of space to leave above the theorem. E.g.: 3pt
   {\topsep}% measure of space to leave below the theorem. E.g.: 3pt
   {\itshape}% font to use in the body of the theorem
   {0pt}% measure of space to indent
   {\bfseries\scshape}% head font
   {~---}% punctuation between head and body
   { }%\newline}% space after theorem head; " " = normal interword space
   {}%{\thmname{#1}\thmnumber{ #2}\thmnote{ (#3)}}% Manually specify head
\theoremstyle{mythm}

\newtheorem{theoreme}{Théorème}[chapter]
\newtheorem{propriete}{Propriété}[chapter]
\newtheorem{definition}{Définition}[chapter]
\renewcommand{\proofname}{Preuve}
\renewcommand\qedsymbol{$\blacksquare$}%{C.Q.F.D.}%

% Notations
\newcommand*{\rank}[1]{\mathrm{rang}{#1}}
\newcommand*{\determinant}[1]{\det{#1}}%{\left\lvert {#1} \right\rvert}
\newcommand*{\inv}[1]{#1\raisebox{1.15ex}{$\scriptscriptstyle-\!1$}}
\newcommand*{\inverse}[1]{\ensuremath{{#1} ^{-1}}} % inverse matrix
\newcommand*{\transpose}[1]{\ensuremath{{#1} ^\mathsf{T}}} % transpose matrix
\newcommand*{\dotprod}[2]{ \transpose{#1} {#2} } % dot product
\newcommand*{\crossprod}[2]{ {#1} \times {#2} } % cross product
\newcommand*{\normtwo}[1]{ \left\| {#1} \right\| } % Euclidean norm
\newcommand*{\norminf}[1]{ \left\| {#1} \right\|_{\infty} } % infinity norm
\newcommand*{\vit}[1]{\bm{#1}} % greek letter vector
\newcommand*{\vrm}[1]{\mathbf{#1}} % latin letter vector
\newcommand*{\unitized}[1]{ \frac{#1}{\normtwo{#1}} }
\newcommand*{\scalprod}[2]{\left \langle {#1}, {#2} \right \rangle}
\newcommand*{\modulo}[1]{\pmod{#1}}%{\left[ {#1} \right]}

\DeclareMathOperator*{\argmin}{\arg\!\min} % argmin
\DeclareMathOperator*{\sign}{sign} % argmin

\newcommand*{\bigO}[1]{O\!\left( {#1} \right)} % big-O
\newcommand*{\littleo}[1]{o\!\left( {#1} \right)} % little-o

% sets
\newcommand*{\reals}{\ensuremath{\mathbb{R}}}
\newcommand*{\integers}{\ensuremath{\mathbb{N}}}
\newcommand{\family}[4]{\left\{{#1}_{#2}\right\}_{{#2}={#3},\ldots,{#4}}}
\newcommand{\ffamily}[7]{\left\{{#1}_{#2,#5}\right\}_{\substack{{#2}={#3},\ldots,{#4}\\ \substack{{#5}={#6},\ldots,{#7}}}}}
\newcommand{\polyspace}[1]{\Pi_{#1}}%\mathbb{R}_{#1}[x]}
\newcommand{\Ltwospace}{L^2\chebinterval}%\mathcal{L}_2\chebinterval}
\newcommand*{\contdiff}[1]{\ensuremath{C^{#1}}}%{\matholdcal{C}^{#1}\!}
\newcommand*{\contgeom}[1]{\ensuremath{G^{#1}}}%{\matholdcal{G}^{#1}\!}

\newcommand*{\uvdomain}{\ensuremath{\mathcal{U}}}
\newcommand*{\wdomain}{\ensuremath{\mathcal{W}}}

\newcommand*{\bx}{\vit{x}}
\newcommand*{\bu}{\vit{u}}
\newcommand*{\bd}{\vit{\delta}}
\newcommand*{\bg}{\vit{\gamma}}
\newcommand*{\bgw}{\bg_{\mkern-2muw}}
\newcommand*{\bp}{\vit{\psi}}
\newcommand*{\bpw}{\bp_{\mkern-2muw}}
\newcommand*{\bl}{\vit{\lambda}}
\renewcommand*{\bs}{\vit{\sigma}} % originally backslash in typewriter font
\newcommand*{\bsu}{\bs_{\mkern-3muu}}
\newcommand*{\bsv}{\bs_{\mkern-2muv}}
\newcommand{\bt}{\vrm{t}}
\newcommand{\br}{\vrm{r}}
\newcommand{\bo}{\vit{o}}

\newcommand*{\unv}{\vrm{n}} % unit normal vector
\newcommand*{\nv}{\hat{\unv}} % normal vector



\newcommand{\colvec}[1]{\begin{pmatrix} #1 \end{pmatrix}} % vector printed in column
\newcommand{\rowvec}[1]{ \transpose{\begin{pmatrix} #1 \end{pmatrix}} } % vector printed in row (transposed column)

% Differential operators
\newcommand*{\dfdx}[2]{\frac{\mathrm{d}}{\mathrm{d}{#2}}{#1}}
\newcommand*{\partialdfdx}[2]{\frac{\partial}{\partial{#2}}{#1}}
\newcommand*{\dkfdxk}[3]{\frac{\mathrm{d}^{#3}}{\mathrm{d}{#2}^{#3}}{#1}}
\newcommand*{\deriv}[2]{{#1}^{({#2})}}
\newcommand*{\prim}[2]{{#1}^{(\raisebox{0.2ex}{$\scriptscriptstyle-\!$}{#2})}}
\newcommand*{\jacobian}[1]{\mathbf{J}_{#1}}
\newcommand*{\gradient}[1]{\nabla \! {#1}}
\newcommand*{\hessian}[1]{\mathbf{H}_{#1}}
\newcommand*{\divergence}[1]{\nabla \cdot {#1}}%\mathrm{div}{#1}}
\newcommand*{\dx}[1]{\mathrm{d}{#1}}



% Condition number
\newcommand*{\cond}[1]{\mathrm{cond} \! \left( {#1} \right)}

% Differential geometry
\newcommand{\fff}{\mathbf{I}}     % first fundamental form
\newcommand{\sffc}{I\mkern-4.8muI} % second fundamental form coefs
\newcommand{\sff}{\mathbf{I\!I}} % second fundamental form




%% Chebyshev polynomials
% Reference interval
\newcommand*{\chebinterval}{\ensuremath{\left[ -1, 1 \right]}}
\newcommand*{\chebopeninterval}{\ensuremath{\left] -1, 1 \right[}}
\newcommand*{\series}[1]{S{#1}}
\newcommand*{\truncseries}[2]{P_{#2}{#1}}
\newcommand*{\interpolant}[2]{I_{#2}{#1}}
\newcommand*{\interpderiv}[3]{D_{#2}^{(#3)}{\!#1}}%\mathcal{D}
%\newcommand*{\interpprim}[2]{J_{#2}{#1}}
\newcommand*{\interpprim}[2]{D_{#2}^{(\raisebox{0.2ex}{$\scriptscriptstyle-\!$}1)}{\!#1}}

% Reference interval for Bernstein polynomials
\newcommand*{\berninterval}{\left[ 0, 1 \right]}

%% System of equations
\newenvironment{eqsys}
{\left\lbrace\begin{array}{@{}l@{}}}
{\end{array}\right.}
%ex :
%\begin{equation}
%	\begin{eqsys}
%		y_1 = a_1 x + b_1 \\
%		y_2 = a_2 x + b_2
%	\end{eqsys}
%\end{equation}


% Intervals
\newcommand*{\lo}[1]{\underline{#1}}
\newcommand*{\hi}[1]{\widebar{#1}}%\overline{#1}}


% GRAPHS
\DeclareMathOperator{\orig}{orig}%ine}
\DeclareMathOperator{\dest}{dest}%ination}



% ENVELOPES OF SPHERES/BALLS
%\newcommand*{\sphere}[2]{\mathscr{S}(#1,#2)}                   % sphère
\newcommand*{\spherenotation}{\mathscr{S}}
\NewDocumentCommand{\sphere}{o o}{%
	\IfValueTF{#1}{%
		\IfValueTF{#2}{%
			\ensuremath{\spherenotation(#1,#2)}%
		}{%
			\ensuremath{\spherenotation(#1)}%
		}%
	}{%
		\ensuremath{\spherenotation}%
	}%
}

\newcommand*{\ball}[2]{\mathscr{B}(#1,#2)}                     % boule ouverte
\newcommand*{\implicitsphere}{S}                               % fonction définissant implicitement une sphère
\newcommand*{\implicitEdB}[1]{\varphi_{#1}}                    % fonction définissant implicitement une EdB
\newcommand*{\EdB}[2]{{#1}^{#2}}                               % EdB
%\newcommand*{\EoS}[2]{\widehat{\EoB{#1}{#2}}}                  % EdS
\newcommand*{\influ}{\mathscr{I}}                              % zone d'influence
\newcommand*{\influEdB}[2]{\influ^{\!#2}({#1})}                % zone d'influence sur l'EdB
%\newcommand*{\influEoS}[2]{\widehat{\influ}^{\!#2}({#1})}      % zone d'influence sur l'EdS
%\newcommand*{\properEoB}[2]{\influ_*^{\!#2}({#1})}             % EdB propre
%\newcommand*{\properEoS}[2]{\widehat{\influ}_{*}^{\!#2}({#1})} % EdS propre
\newcommand*{\eos}{\vit{e}}                                    % paramétrisation de l'EdS (propre)
\newcommand*{\dilation}[2]{{#1}_{\uparrow #2}}                 % "dilatation"

\newcommand*{\envelope}{\mathcal{E}}
\newcommand*{\EdS}[2]{{\envelope}^{#2}({#1})}                  % EdS
\newcommand*{\EdSpropre}[2]{{\envelope}_{*}^{#2}({#1})}        % EdS propre
\newcommand*{\EdSpropreplus}[2]{{\envelope}_{+}^{#2}({#1})}    % EdS propre "positive"
\newcommand*{\EdSpropremoins}[2]{{\envelope}_{-}^{#2}({#1})}   % EdS propre "négative"
\newcommand*{\pseudoEdS}[2]{\widehat{\envelope}^{#2}({#1})}    % pseudo-EdS

\newcommand*{\circlenotation}{\mathscr{C}}

% SET THEORY/TOPOLOGY
\newcommand*{\closure}[1]{\widebar{#1}}%\mathrm{cl}#1}%
\newcommand*{\interior}[1]{\mathrm{int}{#1}}%\mathring{#1}}%
\newcommand*{\boundary}[1]{\partial #1}
\renewcommand*{\complement}[1]{#1^\mathsf{C}}
\newcommand*{\neigborhood}{\mathscr{V}}



%\newcommand*{\proof}[1]{\textcolor{blue!60!green}{\textbf{Preuve : }#1$\,_\blacksquare$}}



% Encircled text
\newcommand{\pgftextcircled}[1]{                                                                    
    \setbox0=\hbox{#1}%
    \dimen0\wd0%
    \divide\dimen0 by 2%
    \begin{tikzpicture}[baseline=(a.base)]%
        \useasboundingbox (-\the\dimen0,0pt) rectangle (\the\dimen0,1pt);
        \node[circle,draw,outer sep=0pt,inner sep=0.1ex] (a) {#1};
    \end{tikzpicture}
}

 % notations mathématiques
\newboolean{titlerectangle}
\setboolean{titlerectangle}{true}

\newboolean{titlevbar}
\setboolean{titlevbar}{true}

\newboolean{titlehelpers}
\setboolean{titlehelpers}{true}

\newboolean{sansseriftitles}
\setboolean{sansseriftitles}{false}

\ifthenelse{\boolean{sansseriftitles}}{
	\def\titlefont{\sffamily}%
}
{
	\def\titlefont{\rmfamily}%
}

% Table of contents style
\renewcommand*{\contentsname}{Table of contents}
\renewcommand*{\cftchapterfont}{\titlefont\bfseries}



\maxtocdepth{subsection}

% Titles
\colorlet{colchapttl}{mydarkgray}%black}%DodgerBlue3}%



\makeatletter
\makechapterstyle{mychapterstyle}{
\setlength{\beforechapskip}{0pt}%40pt}
\setlength{\midchapskip}{0pt}%25pt}
\newlength{\afterchapskipdef}
\newlength{\afterchapskipxtra}
\setlength{\afterchapskipdef}{8\onelineskip}%60pt}%

\newif\ifNoChapNumber
\newlength{\barheight}
\newlength{\barlength}
\newlength{\ttltopskip}
\setlength{\ttltopskip}{60mm}
\setlength{\barlength}{0.37\foremargin}
\setlength{\barheight}{14mm}

\newlength{\numberheight}
\ifthenelse{\boolean{sansseriftitles}}{
	\setlength{\numberheight}{1.6\barheight}
}
{
	\setlength{\numberheight}{1.63\barheight}
}

\newlength{\yshiftchapname}
\newlength{\yshiftchapnum}
\setlength{\yshiftchapname}{-\ttltopskip + 10mm + 3.9mm}
\setlength{\yshiftchapnum}{-\ttltopskip - 0.1mm}

\newlength{\vbaryshift}
\setlength{\vbaryshift}{1.25mm}
\newlength{\vbarxshift}
\setlength{\vbarxshift}{2.5mm}
\newlength{\vbarthck}
\setlength{\vbarthck}{0.6pt}
\newlength{\vbartipthck}
\setlength{\vbartipthck}{0.15pt}

\renewcommand{\chapnamefont}{\scshape\titlefont}
\renewcommand{\chapnumfont}{\normalfont\titlefont\fontsize{\numberheight}{0mm}\selectfont}
\renewcommand{\chaptitlefont}{\normalfont\titlefont\bfseries\huge}%
\renewcommand{\printchaptername}{}%\chapnamefont\@chapapp}
\renewcommand{\chapternamenum}{}
\renewcommand{\printchapternum}{}


\renewcommand\printchaptertitle[1]{
	\begin{tikzpicture}[remember picture,overlay]
		\node[yshift=-\ttltopskip] at (current page.north east) {%
			\begin{tikzpicture}[remember picture, overlay]%
				%\node[%draw=red, dashed,%*****
				%	anchor=north east,%
				%	%yshift=\yshiftchapname,%
				%	%xshift=-\foremargin,%
				%	text width=\textwidth-\vbarxshift,%
				%	minimum height=\barheight,%
				%	align=flush right,%justify,%
				%	inner sep=0mm%
				%	]%
				%	(chapttl)%
				%	at ([xshift=-\foremargin-\vbarxshift, yshift=\yshiftchapname] current page.north east)%
				%	{\chaptitlefont\color{colchapttl}##1\par};%
				%
				%\draw[red, very thick, dotted] (chapttl.west) -- (chapttl.east);%*****
				%
				%\gettikzxy{(chapttl.south west)}{\bx}{\by}%
				%\global\afterchapskipxtra=-\by%
				%\pgfmathsetlength{\afterchapskip}{\afterchapskipdef + \afterchapskipxtra}%
				%\global\afterchapskip=\afterchapskip%
				%
				\ifNoChapNumber%
                                        \node[%draw=red, dashed,%*****
					anchor=north east,%
					%yshift=\yshiftchapname,%
					%xshift=-\foremargin,%
					text width=\textwidth,%
					minimum height=\barheight,%
					align=flush right,%justify,%
					inner sep=0mm%
					]%
					(chapttl)%
					at ([xshift=-\foremargin, yshift=\yshiftchapname] current page.north east)%
					{\chaptitlefont\color{colchapttl}##1\par};
					%\draw[black, thick]%
					%	([xshift=\vbarxshift, yshift= \vbaryshift] chapttl.north east)%
					%	--%
					%	([xshift=\vbarxshift, yshift=-\vbaryshift] chapttl.south east);%
					%\relax%
				\else%
                                        \node[%draw=red, dashed,%*****
					anchor=north east,%
					%yshift=\yshiftchapname,%
					%xshift=-\foremargin,%
					text width=\textwidth-\vbarxshift,%
					minimum height=\barheight,%
					align=flush right,%justify,%
					inner sep=0mm%
					]%
					(chapttl)%
					at ([xshift=-\foremargin-\vbarxshift, yshift=\yshiftchapname] current page.north east)%
					{\chaptitlefont\color{colchapttl}##1\par};%
					\node[%draw=blue, dashed,%*****
						anchor=north west,%west,%
						align=left,%
						xshift=2\vbarxshift,%
						inner sep=0mm%
						]%
						(chapnum)% 
						at (chapttl.north east)%(chapttl.east)%
						{\color{colchapttl}\chapnumfont\thechapter};%
					%
					%\draw[blue, very thick, dotted] (chapnum.west) -- (chapnum.east);%*****
					%
%\draw[colchapttl, line width=2\vbarthck]%thick]%
%([xshift=\vbarxshift, yshift= \vbaryshift] chapttl.north east)%
%--%
%([xshift=\vbarxshift, yshift=-\vbaryshift] chapttl.south east);%
%
\coordinate (a) at ([xshift=\vbarxshift+\vbartipthck,yshift=-\vbaryshift] chapttl.south east);
\coordinate (b) at ([xshift=\vbarxshift+\vbarthck]chapttl.east);
\coordinate (c) at ([xshift=\vbarxshift+\vbartipthck,yshift=+\vbaryshift] chapttl.north east);
\coordinate (d) at ([xshift=\vbarxshift-\vbartipthck,yshift=+\vbaryshift] chapttl.north east);
\coordinate (e) at ([xshift=\vbarxshift-\vbarthck]chapttl.east);
\coordinate (f) at ([xshift=\vbarxshift-\vbartipthck,yshift=-\vbaryshift] chapttl.south east);
%
\path [fill=colchapttl]%
(a) [in=0, out=90] to %
[in=270, out=90] (b) to [in=270, out=90] %
(c) -- %
(d) [in=180, out=270] to %
[in=90, out=270] (e) to [in=90, out=270] %
(f) -- cycle;
				\fi
                                \gettikzxy{(chapttl.south west)}{\bx}{\by}%
				\global\afterchapskipxtra=-\by%
				\pgfmathsetlength{\afterchapskip}{\afterchapskipdef + \afterchapskipxtra}%
				\global\afterchapskip=\afterchapskip%
			\end{tikzpicture}%
		};%
	\end{tikzpicture}%
}
\renewcommand\printchapternonum{\NoChapNumbertrue}
}
\makeatother

% helper (check if proper space after chapter title)
\newcommand{\printskip}{%
%\textbf{\the\afterchapskipdef~$+$~\the\afterchapskipxtra~$=$~\the\afterchapskip}%
}

\makeatletter
\newcommand{\printchapapp}{}%This is a \MakeLowercase{\@chapapp}.}%
\makeatother


\chapterstyle{mychapterstyle}%veelo}%











%% Format chapter abstracts
\def\abstractname{}
\def\abstitleskip{0pt}

\usepackage{lettrine}  % dropped capitals
\renewcommand{\LettrineTextFont}{\scshape}
\renewcommand{\DefaultLhang}{0.1}
\renewcommand{\DefaultNindent}{0pt}
% \renewcommand{\DefaultOptionsFile}{\lettrineconffile}



%% Sections
\setsecheadstyle{\color{colchapttl}\titlefont\Large\bfseries}%\raggedright}
\setbeforesecskip{-2\onelineskip}
\setaftersecskip{\onelineskip}
\setsecindent{0pt}

%% Subsections
\setsubsecheadstyle{\color{colchapttl}\titlefont\large\bfseries\sethangfrom{\noindent ##1}\raggedright}
\setbeforesubsecskip{-\onelineskip}
\setaftersubsecskip{\onelineskip}
\setsubsecindent{0pt}

%% Subsubsections
\setsubsubsecheadstyle{\color{colchapttl}\titlefont\bfseries\sethangfrom{\noindent ##1}\raggedright}
%\setbeforesubsubsecskip{-\onelineskip}
%\setaftersubsubsecskip{\onelineskip}
%\setsubsubsecindent{0pt}

%% Paragraphs
\setparaheadstyle{\bfseries\sethangfrom{\noindent ##1}\raggedright}
%\setbeforeparaskip{-\onelineskip}
%\setafterparaskip{-1em}
%\setparaindent{0pt} % style des titres
% Page style (header, footer) (memoir options)
\def\headingsfont{\titlefont}%\sffamily}%

% https://tex.stackexchange.com/questions/183973/preserve-memoir-headings-for-1-page
% \makeatletter
% \providecommand*{\righttopmark}{\expandafter\@rightmark\topmark\@empty\@empty}
\makepagestyle{mypagestylezero} 
\makeoddfoot{mypagestylezero}{}{}{} 
\makeevenfoot{mypagestylezero}{}{}{} 
\makeevenhead{mypagestylezero}{\thepage}{}{\headingsfont\small\leftmark}
\makeoddhead{mypagestylezero}{\itshape\headingsfont\small\rightmark}{}{\thepage}
\makepsmarks{mypagestylezero}{%
	\nouppercaseheads
	\createmark{chapter}{both}{nonumber}{}{ \ }%. \ }%
	\createmark{section}{right}{shownumber}{} { \ }
    
    \createplainmark{bib}{both}{\bibname}
    \createplainmark{toc}{both}{\contentsname}
}%
%\makeatother

\copypagestyle{mypagestyle}{mypagestylezero}
\makeheadrule{mypagestyle}{\textwidth}{\normalrulethickness} 
% \makepsmarks{mypagestyle}{%
% 	\nouppercaseheads
% 	\createmark{chapter}{left}{nonumber}{}{. \ }
% 	\createmark{section}{right}{nonumber}{} {. \ }
% }%


\copypagestyle{mypagestyleb}{mypagestyle}
\makeoddfoot{mypagestyleb}{}{}{\thepage} 
\makeevenfoot{mypagestyleb}{\thepage}{}{}
\makeevenhead{mypagestyleb}{\headingsfont\small\scshape\leftmark}{}{}
\makeoddhead{mypagestyleb}{}{}{\headingsfont\small\rightmark}



\copypagestyle{mypagestylebiblio}{mypagestyleb}
\makeoddhead{mypagestylebiblio}{}{}{\headingsfont\small\scshape\rightmark}




\makepagestyle{mypagestylenew}
\makeevenhead{mypagestylenew}{\thepage\hskip.5cm\vrule\hskip.5cm\leftmark}{}{} \makeoddhead{mypagestylenew}{}{}{\rightmark\hskip.5cm\vrule\hskip.5cm\thepage} 
\makeatletter 
\makepsmarks{mypagestylenew}{ 
  \def\chaptermark##1{\markboth{% 
  \ifnum \value{secnumdepth} < -1 %
  	\if@mainmatter \chaptername\ \thechapter\ --- %
  	\fi %
  \fi ##1}{}} 
  \def\sectionmark##1{\markright{% 
  \ifnum \value{secnumdepth} < 0 %
  	\thesection. \ %
  \fi ##1}} } 
\makeatother 




\makepagestyle{myruledpagestyle} 
%\makeevenhead{myruledpagestyle}{\liningnums{\thepage}}{}{\leftmark} \makeoddhead{myruledpagestyle}{\rightmark}{}{\liningnums{\thepage}}
\makeevenhead{myruledpagestyle}{\thepage}{}{\leftmark} \makeoddhead{myruledpagestyle}{\rightmark}{}{\thepage}
\makeatletter
\makepsmarks{myruledpagestyle}{ 
  \nouppercaseheads
  \def\chaptermark##1{\markboth{% 
    {\small\headingsfont%
    \ifnum \value{secnumdepth} > -1 %
      \if@mainmatter %
      	\ % \chaptername\ \thechapter\ --- % 
      \fi %
    \fi %
    \scshape ##1}}{}%
  } 
  \def\sectionmark##1{\markright{% 
    {\small\headingsfont\itshape%
    \ifnum \value{secnumdepth} > 0 %
    	\thesection \ %. \ % 
    \fi %
    ##1}}}
    \createplainmark{bib}{both}{\small\headingsfont\scshape\bibname}
    \createplainmark{toc}{both}{\small\headingsfont\scshape\contentsname}
}
\makeatother 
\makerunningwidth{myruledpagestyle}{\textwidth}%1.1\textwidth}%
\makeheadposition{myruledpagestyle}{flushright}{flushleft}{flushright}{flushleft}
%\makeheadrule{myruledpagestyle}{\textwidth}{\normalrulethickness} 




\copypagestyle{chapter}{empty}
% \copypagestyle{chapter}{plain}
% \makeoddfoot{chapter}{}{}{\thepage}



\setsecnumdepth{subsubsection}

%\mergepagefloatstyle{mergedstyle}{myruledpagestyle}{empty} % empty pagestyle on pages with only floats (figures, tables)

\pagestyle{myruledpagestyle}%mypagestyle}%



 % entetes et pieds de pages
% Format captions (figures, tables, ...) (memoir options)
\captiondelim{\space\textendash\space}%$\mid$\space}%
\captionnamefont{\small\scshape}%\bfseries}
\captiontitlefont{\small\itshape}
\hangcaption

%\renewcommand{\figurename}{Fig.}

%\usepackage{subfig}
\providecommand\subfigureautorefname{Figure}
\newsubfloat{figure}% Allow subfloats in figure environment (subfigures)

\subcaptionsize{\footnotesize}
\subcaptionlabelfont{\normalfont}
\subcaptionfont{\itshape}

% Blank footnotes 
%https://tex.stackexchange.com/questions/30720/footnote-without-a-marker
\newcommand\blfootnote[1]{%
  \begingroup
  \renewcommand\thefootnote{}\footnote{#1}%
  \addtocounter{footnote}{-1}%
  \endgroup
}


\tightsubcaptions
%\loosesubcaptions
 % format des légendes
%% Abbreviations
\newcommand*{\ie}{i.e.\ }
\newcommand*{\eg}{e.g.\ }
\newcommand*{\etc}{etc.}

%% Commonly used words
\newcommand{\brep}{BRep}


%% Hyphenation rules for some words
\hyphenation{res-pec-tively}
\hyphenation{mono-ti-ca-lly}
\hyphenation{hypo-the-sis}
\hyphenation{para-me-ters}
\hyphenation{sol-va-bi-li-ty}

\hyphenation{in-té-res-se}


% mots anglais
\newcommand{\eng}[1]{\textit{#1}}

% guillemets
\newcommand{\guill}[1]{«~{#1}~»} % abréviations, mots courants, fonctions sur texte
\renewcommand{\bibname}{Bibliographie}
\renewcommand{\contentsname}{Table des matières}

\renewcommand*\chapterautorefname{Chapitre}
\renewcommand*\sectionautorefname{Section}
\renewcommand*\subsectionautorefname{Section}
\renewcommand*\appendixautorefname{Annexe} % noms de sections et de flottants
\usepackage{silence}

%warnings due to minitoc set off (see I.5 of http://texdoc.net/texmf-dist/doc/latex/minitoc/minitoc.pdf)
\WarningFilter{minitoc(hints)}{W0023} %due to the loading of hyperref
\WarningFilter{minitoc(hints)}{W0024} %due to the loading of hyperref
\WarningFilter{minitoc(hints)}{W0028} %due to the loading of hyperref
\WarningFilter{minitoc(hints)}{W0030} %due to the loading of hyperref

\newcommand{\figbrepface}[3]{%#1 : indice de face, #2 : x ancrage, #3 :y ancrage
\DTLsetseparator{,}%
%
\DTLloaddb[noheader,keys={id,x,y,a,dx,dy}]{dbverts}{figures/data/BRep/faces/verts_dxy_#1.dat}%
%
\DTLloaddb[noheader,keys={id,x,y,dx,dy,a}]{dbedges}{figures/data/BRep/faces/edges_#1.dat}%
%
\DTLloaddb[noheader,keys={r,g,b}]{dbfacecolor}{figures/data/BRep/faces/facecolor_#1.dat}%
\DTLassign{dbfacecolor}{1}{\rfai=r,\gfai=g,\bfai=b}% 
\definecolor{facecolor}{RGB}{\rfai,\gfai,\bfai}
%
\DTLloaddb[noheader,keys={u,v,du,dv}]{dbwires}{figures/data/BRep/faces/contours_label_#1.dat}%
\pgfmathsetmacro\numberofwires{\DTLrowcount{dbwires}}%
%
\DTLloaddb[noheader,keys={u,v,du,dv,ied,ihe,iw}]{dbcurves}{figures/data/BRep/faces/curve_uvdata_#1.dat}%
%
\DTLloaddb[noheader,keys={id,x,y}]{dbfacexyzlabel}{figures/data/BRep/faces/face_xyzlabel_#1.dat}%
\DTLassign{dbfacexyzlabel}{1}{\idflab=id,\xflab=x,\yflab=y}%
%
\DTLloaddb[noheader,keys={u,v}]{dbfaceuvlabel}{figures/data/BRep/faces/face_uvlabel_#1.dat}%
%
\begin{scope}[shift={({#2},{#3})}]
	%% SURFACE & FACE
	\node[img] (face_#1) at (0,0) {\includegraphics[width=\imfacew]{BRep/faces/face_#1}};
	%
	{\transparent{0.25}%
		\node[img] (surface_#1) at (0,0) {\includegraphics[width=\imfacew]{BRep/faces/surface_#1}};}%
	%%% HIDDEN EDGES
		\node[img] at (0,0) {\includegraphics[width=\imfacew]{BRep/faces/edges_hid_#1}};%}%
	\DTLforeach*{dbedges}{\loci=id, \locx=x, \locy=y, \locdx=dx, \locdy=dy, \loca=a}%
	{%
		\pgfmathsetmacro\iloci{int(round(\loci))}%
		\pgfmathsetmacro\scl{\edglabsepxyz/veclen(\locdx,\locdy)}
		\node[label, anchor=center] at 
		({\locx+\scl*\locdx},{\locy+\scl*\locdy}) 
		{$\brepedge_{\iloci}$};%
	}%
	%%% HIDDEN VERTICES
	\DTLforeach*{dbverts}{\loci=id, \locx=x, \locy=y, \loca=a, \locdx=dx, \locdy=dy}%
	{%
		\pgfmathsetmacro\iloca{int(round(\loca))}%
		\ifnum \iloca = 0
			\fill[black] (\locx,\locy) circle (1pt);
			\node[label, anchor=center] at 
			({\locx+\vertsep*\locdx},{\locy+\vertsep*\locdy}) 
			{$\brepvertex_{\mkern-2mu\loci}$};%
		\fi
	}%
	%%% FACE (semi-transparent to mask hidden edges & verts)
	{\transparent{0.65}%
		\node[img] (facetr_#1) at (0,0) {\includegraphics[width=\imfacew]{BRep/faces/face_#1}};}%
	%%% VISIBLE EDGES
	\node[img] at (0,0) {\includegraphics[width=\imfacew]{BRep/faces/edges_vis_#1}};
	%%% VISIBLE VERTICES
	\DTLforeach*{dbverts}{\loci=id, \locx=x, \locy=y, \loca=a, \locdx=dx, \locdy=dy}%
	{%
		\pgfmathsetmacro\iloca{int(round(\loca))}%
		\ifnum \iloca = 1
			\fill[black] (\locx,\locy) circle (1pt);
			\node[label, anchor=center] at 
			({\locx+\vertsep*\locdx},{\locy+\vertsep*\locdy}) 
			{$\brepvertex_{\mkern-2mu\loci}$};%
		\fi
	}%
	% FACE LABEL
	\node[label, anchor=center] at (\xflab,\yflab) {$\brepface_{\idflab}$};
	%%% UV-SPACE
	\begin{scope}[shift={(0.5,-0.7)}, scale={\uvscale}]
		%% UV-GRID
		\foreach \loci in {0,...,\ngriduv}
		{%
			\pgfmathsetmacro\locxi{-1 + 2*\loci/\ngriduv}%
			\draw[uvgrid] (-1,\locxi) -- ++ (2,0);
			\draw[uvgrid] (\locxi,-1) -- ++ (0,2);
		}%		
		%% FILL UV DOMAIN
		\begin{scope}[every path/.style={draw=none,fill=facecolor}]
			\path 
				plot file {figures/data/BRep/faces/contour_ext_#1.dat} -- cycle
				\foreach \iwint in {1,...,\numberofwires}{
					plot file {figures/data/BRep/faces/contour_int_#1_\iwint.dat} -- cycle
				};
		\end{scope}
		%% DRAW EDGES
		\DTLforeach*{dbcurves}{\locu=u, \locv=v, \locdu=du, \locdv=dv, \locie=ied, \locih=ihe, \lociw=iw}{%
			\pgfmathsetmacro\ilocie{int(round(\locie))}%
			\pgfmathsetmacro\ilocih{int(round(\locih))}%
			\pgfmathsetmacro\ilociw{int(round(\lociw))}%
			% set wire color
			\pgfmathsetmacro\locfiw{(\iniwclr+\lociw*\decwclr)}%
			\pgfmathsetmacro\locriw{\locfiw*\rfai/255.}%
			\pgfmathsetmacro\locgiw{\locfiw*\gfai/255.}%
			\pgfmathsetmacro\locbiw{\locfiw*\bfai/255.}%
			\definecolor{clriw}{rgb}{\locriw, \locgiw, \locbiw}%
			%
			\node[label, anchor=center, clriw] at 
			({\locu + \edglabsepuv*\locdv},
			{\locv - \edglabsepuv*\locdu})
			{$\brepedge_{\ilocie}^{\ilocih}$};
			\draw[curv, 
				clriw,
				-{Triangle[left]}, 
				shorten <= 0.25pt, 
				shorten >= 0.25pt] plot file {figures/data/BRep/faces/curve_uv_#1_\DTLcurrentindex.dat};
		}%
		%% WIRE LABELS
		\DTLforeach*{dbwires}{\locu=u, \locv=v, \locdu=du, \locdv=dv}{%
			% set wire color
		 \pgfmathsetmacro\locfiw{(\iniwclr+(\DTLcurrentindex-1)*\decwclr)}%
			\pgfmathsetmacro\locriw{\locfiw*\rfai/255.}%
			\pgfmathsetmacro\locgiw{\locfiw*\gfai/255.}%
			\pgfmathsetmacro\locbiw{\locfiw*\bfai/255.}%
			\definecolor{clriw}{rgb}{\locriw, \locgiw, \locbiw}%
			\pgfmathsetmacro\locx{\locu-\wirlabsepuv*\locdv}%
			\pgfmathsetmacro\locy{\locv+\wirlabsepuv*\locdu}%
			\ifnum \DTLcurrentindex = 1%
				\node[label, anchor=center, clriw] at (\locx,\locy) {$\brepwire_{\idflab}^{\mathrm{ext}}$};
			\else%
				\pgfmathsetmacro\ilociw{int(\DTLcurrentindex - 1)}
				\node[label, anchor=center, clriw] at (\locx,\locy) {$\brepwire_{\idflab}^{\mathrm{int},\ilociw}$};
			\fi%
		}%
		%% FACE LABEL
		\DTLassign{dbfaceuvlabel}{1}{\locu=u,\locv=v}% 
		\node[label, anchor=center] at (\locu,\locv) {$\brepface_{\idflab}$};
		%
%		\draw[blue, dashed] (-1,-1) -- (1,-1) -- (1,1) -- (-1,1) -- cycle;
%		\draw[blue, dashed] (0,-1) -- (0,1);
%		\draw[blue, dashed] (-1,0) -- (1,0);
	\end{scope}
	%
%	\draw[red, dashed] (0,0) -- (1,0) -- (1,1) -- (0,1) -- cycle;
%	\draw[red, dashed] (.5,0) -- (.5,1);
%	\draw[red, dashed] (0,.5) -- (1,.5);
%	\foreach \ii in  {0,0.1,...,1.01}{
%		\draw[red, thin] (0,\ii) -- (1,\ii)
%		                 (\ii,0) -- (\ii,1);
%	}
\end{scope}
\DTLgdeletedb{dbfacexyzlabel}
\DTLgdeletedb{dbfaceuvlabel}
\DTLgdeletedb{dbfacecolor}
\DTLgdeletedb{dbcurves}
\DTLgdeletedb{dbwires}
\DTLgdeletedb{dbverts}
\DTLgdeletedb{dbedges}
}
\setlength{\imagewidth}{74mm}
\setlength{\imageheight}{\imagewidth}
\def\trmask{0.82}
\begin{figure}
  \centering
  \tikzset{x=\imagewidth, y=\imageheight,
  	img/.style={anchor=south west, inner sep=0}}
  %
  \hspace*{\fill}
  \subbottom[Modèle \brep\ de l'interface.]{
	\begin{tikzpicture}
		\figEoBBrep{1}{0}{0}
%		\draw[blue, dashed] 
%			(current bounding box.south west) --
%			(current bounding box.south east) --
%			(current bounding box.north east) --
%			(current bounding box.north west) -- cycle;
	\end{tikzpicture}
  }
  \hfill%
  \subbottom[Modèle \brep\ de l'enveloppe des boules centrées sur l'interface.]{
	\begin{tikzpicture}
		\figEoBBrep{2}{0}{0}
%		\draw[blue, dashed] 
%			(current bounding box.south west) --
%			(current bounding box.south east) --
%			(current bounding box.north east) --
%			(current bounding box.north west) -- cycle;
	\end{tikzpicture}
  }
  \hspace*{\fill}
  \caption{Entrée et sortie de l'algorithme présenté dans le chapitre (REFAIRE).}
  %
\end{figure} % template de figures
%https://www.overleaf.com/learn/latex/Nomenclatures
%https://www.xm1math.net/doculatex/nomenclature.html
\usepackage[intoc, french]{nomencl}
\makenomenclature

\renewcommand{\nomname}{Notations}

%% This code creates the groups
% -----------------------------------------
\usepackage{etoolbox}
\renewcommand\nomgroup[1]{%
  \item[\bfseries
  \ifstrequal{#1}{T}{Topologie}{%
  \ifstrequal{#1}{A}{Analyse}{%
  \ifstrequal{#1}{O}{Other Symbols}{}}}%
]}
% -----------------------------------------
 
 % création d'une liste des notations

%\usepackage[outer,final]{showlabels} %final to deactivate


%\usepackage{microtype} % Makes pdf look better.
\usepackage[
	activate={true,nocompatibility},% activate protrusion and expansion
	final, % enable microtype; use "draft" to disable
	tracking=true, % 
	kerning=true, % 
	spacing=true, % 
	factor=1100, % add 10% to the protrusion amount (default is 1000)
	stretch=10, % reduce stretchability (default is 20/20)
	shrink=10 % reduce shrinkability (default is 20/20)
]{microtype}
\SetTracking{encoding={*}, shape=sc}{40} % reduce space between small cap letters
%\microtypecontext{spacing=nonfrench} % si texte en anglais
\frenchspacing


%Reduce widows  (the last line of a paragraph at the start of a page) and orphans (the first line of paragraph at the end of a page)
\widowpenalty=1000
\clubpenalty=1000

%% Hyphenation
%https://sumanta679.wordpress.com/2009/05/20/latex-justify-without-hyphenation/
%\tolerance=1
%\emergencystretch=\maxdimen
%\hyphenpenalty=10000
%\hbadness=10000
%\hyphenchar\font=-1 % suppress hyphen character completely
%\sloppy % get rid of overfull boxes
%\fussy

\usepackage[pdftex]{changebar}
%\usepackage{versions}          % permet d'activer ou non certains environnements

