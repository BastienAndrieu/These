%% PACKAGES %%%%%%%%%%%%%%%%%%%%%%%%%%%%%
\usepackage{graphicx}

\usepackage{pgf, pgfplots, pgfplotstable}
\usepgfplotslibrary{external, colormaps, patchplots, groupplots}
\pgfplotsset{compat=newest}

%\usepackage{pst-solides3d,pstricks-add}

\usepackage{tikz, tikzscale}

\usetikzlibrary{
angles, 
arrows, 
arrows.meta, 
backgrounds,
bending,
calc,
chains,
decorations.markings, 
decorations.text, 
external,
fadings,
fit,
math,
patterns,
pgfplots.colorbrewer,
positioning,
quotes,
shadows,
shapes,
shapes.geometric, 
shapes.misc,
spy
}
%\tikzexternalize[prefix=figures/tikz_pgf/]

%https://axiomatic.neophilus.net/using-datatool-and-tikz-to-generate-figures-from-data/
\usepackage{datatool}

\usepackage{forloop}

\usepackage{transparent}

%%https://tex.stackexchange.com/questions/57418/crop-an-inserted-image
%\usepackage[export]{adjustbox}
%%%%%%%%%%%%%%%%%%%%%%%%%%%%%%%%%%%%%%%%%%%%%%%%%%%%%%

\graphicspath{{./figures/}{./figures/images/}}






%% MACROS %%%%%%%%%%%%%%%%%%%%%%%%%%%%%
\makeatletter
%%%%%%%%%%%%%%%%%%%%%%%%%%%
% Conversion pt -> mm
%https://tex.stackexchange.com/questions/8260/what-are-the-various-units-ex-em-in-pt-bp-dd-pc-expressed-in-mm
%(The syntax is \convertto{mm}{1pt} to convert 1pt in mm)
\def\convertto#1#2{\strip@pt\dimexpr #2*65536/\number\dimexpr 1#1}
%%%%%%%%%%%%%%%%%%%%%%%%%%%


%%%%%%%%%%%%%%%%%%%%%%%%%%%
\newcommand{\gettikzxy}[3]{%
  \tikz@scan@one@point\pgfutil@firstofone#1\relax
  \edef#2{\the\pgf@x}%
  \edef#3{\the\pgf@y}%
}
%%%%%%%%%%%%%%%%%%%%%%%%%%%
\newcommand{\globalgettikzxy}[3]{%
  \tikz@scan@one@point\pgfutil@firstofone#1\relax
  \edef\@tempdima{\the\pgf@x}%
  \edef\@tempdimb{\the\pgf@y}%
  \global#2=\@tempdima%
  \global#3=\@tempdimb%
}
%%%%%%%%%%%%%%%%%%%%%%%%%%%
\newcommand\getwidthofnode[2]{%
    \pgfextractx{\pgf@xb}{\pgfpointanchor{#2}{east}}%
    \pgfextractx{\pgf@xa}{\pgfpointanchor{#2}{west}}% 
    \pgfmathsetlength{\pgf@xb}{\pgf@xb - \pgf@xa}%
	\global#1=\pgf@xb%
}
%%%%%%%%%%%%%%%%%%%%%%%%%%%
\newcommand\getheightofnode[2]{%
    \pgfextracty{\pgf@yb}{\pgfpointanchor{#2}{north}}%
    \pgfextracty{\pgf@ya}{\pgfpointanchor{#2}{south}}% 
    \pgfmathsetlength{\pgf@yb}{\pgf@yb - \pgf@ya}%
	\global#1=\pgf@yb%
}
%%%%%%%%%%%%%%%%%%%%%%%%%%%

\newcommand{\getDistance}[3]{%
%	%https://tex.stackexchange.com/questions/39293/coordinates-a-b-compute-b-a-and-angle-between-x-and-b-a
%	\pgfpointdiff{\pgfpointanchor{#1}{center}}
%                 {\pgfpointanchor{#2}{center}}
%    \pgf@xa=\pgf@x % no need to use a new dimen
%    \pgf@ya=\pgf@y
%    %\pgfmathparse{veclen(\pgf@xa,\pgf@ya)/28.45274} % to convert from pt to cm   
%    \pgfmathparse{veclen(\pgf@xa,\pgf@ya)}
%    %\global\let\mylength\pgfmathresult % we need a global macro
%    \global#3=\pgfmathresult%
%https://tex.stackexchange.com/questions/412899/tikz-calculate-and-store-the-euclidian-distance-between-two-coordinates?rq=1
\tikz@scan@one@point\pgfutil@firstofone($#1-#2$)\relax  
\pgfmathsetmacro{#3}{veclen(\the\pgf@x,\the\pgf@y)}
}%
%%%%%%%%%%%%%%%%%%%%%%%%%%%
% get row/column index in groupplot
\newcommand{\currentrow}{\the\pgfplots@group@current@row}
\newcommand{\currentcolumn}{\the\pgfplots@group@current@column}
\newcommand{\totalplots}{\pgfplots@group@totalplots}
%%%%%%%%%%%%%%%%%%%%%%%%%%%
% invert a colormap in pgfplots
%https://tex.stackexchange.com/questions/141181/inverting-a-colormap-in-pgfplots
\def\invertcolormap#1{%
    \pgfplotsarraycopy{pgfpl@cm@#1}\to{custom@COPY}%
    \c@pgf@counta=0
    \c@pgf@countb=\pgfplotsarraysizeof{custom@COPY}\relax
    \c@pgf@countd=\c@pgf@countb
    \advance\c@pgf@countd by-1 %
    \pgfutil@loop
    \ifnum\c@pgf@counta<\c@pgf@countb
        \pgfplotsarrayselect{\c@pgf@counta}\of{custom@COPY}\to\pgfplots@loc@TMPa
        \pgfplotsarrayletentry\c@pgf@countd\of{pgfpl@cm@#1}=\pgfplots@loc@TMPa
        \advance\c@pgf@counta by1 %
        \advance\c@pgf@countd by-1 %
    \pgfutil@repeat
%\pgfplots@colormap@showdebuginfofor{#1}%
}%
%%%%%%%%%%%%%%%%%%%%%%%%%%%
\makeatother

%% Legendes %%
\newcommand{\legenddash}[1]{%
	\raisebox{2pt}{\tikz{\draw[#1,solid,thick](0,0) -- (4mm,0);}}%
}
%%%%%%%%%%%%%%%%%%%%%%%%%%%
\def\lgdsqrsiz{1.442pt}
\newcommand{\legendsquare}[1]{\raisebox{1.2pt}{\tikz{\fill[#1] (-\lgdsqrsiz , -\lgdsqrsiz) rectangle (\lgdsqrsiz , \lgdsqrsiz);}}}
%%%%%%%%%%%%%%%%%%%%%%%%%%%
\newcommand{\legenddot}[1]{\raisebox{0.93pt}{\tikz{\fill[#1] (0.0mm,0.0mm) circle [radius=1.7pt];}}}
%%%%%%%%%%%%%%%%%%%%%%%%%%%
\newcommand{\legendtriangle}[1]{\raisebox{1.5pt}{\tikz{\fill[#1] (0.0pt,2.2pt) -- (-1.9pt,-1.1pt) -- (1.9pt,-1.1pt);}}}
%%%%%%%%%%%%%%%%%%%%%%%%%%%%%%%%%%%%%%%%%%%%%%%%%%%%%%




%% PGFPLOT SETTINGS %%%%%%%%%%%%%%%%%%%%%%%%%%%%%%%%%%
\definecolor{mycolor_1}{HTML}{68ABD9}
\definecolor{mycolor_2}{HTML}{FA7566}%FA6655}%{FAA43A}%
\definecolor{mycolor_3}{HTML}{98D45B}%AADA57}%{60BD68}%
\definecolor{mycolor_4}{HTML}{897EDA}
\definecolor{mycolor_5}{HTML}{FAA43A}%{FA6655}%
\definecolor{mycolor_6}{HTML}{46AA5B}%60BD68}%{AADA57}%
\definecolor{mycolor_7}{HTML}{F094C3}
\definecolor{mycolor_8}{HTML}{A3A3A3}
\definecolor{mycolor_9}{HTML}{000000}

\pgfplotscreateplotcyclelist{plotcycle_1}{
  	mycolor_1, mark=square*,   mark size=1.2pt\\
	mycolor_2, mark=*,         mark size=1.5pt\\
	mycolor_3, mark=triangle*, mark size=1.7pt\\
	mycolor_4, mark=diamond*,  mark size=1.7pt\\
	mycolor_5, mark=square*,   mark size=1.2pt\\
	mycolor_6, mark=*,         mark size=1.5pt\\
	mycolor_7, mark=triangle*, mark size=1.7pt\\
	mycolor_8, mark=diamond*,  mark size=1.5pt\\
	mycolor_9, mark=square*,   mark size=1.2pt\\
}


\pgfplotscreateplotcyclelist{plotcycle_BW}{
  	every mark/.append style={solid,fill=white}, mark=square*, mark size=1.2pt\\
	every mark/.append style={solid,fill=white}, mark=*, mark size=1.5pt\\
	every mark/.append style={solid,fill=white}, mark=triangle*, mark size=1.7pt\\
	every mark/.append style={solid,fill=white}, mark=diamond*, mark size=1.7pt\\
%	black, mark=square*,   mark size=1.2pt\\
%	black, mark=*,         mark size=1.5pt\\
%	black, mark=triangle*, mark size=1.7pt\\
%	black, mark=diamond*,  mark size=1.5pt\\
%	black, mark=square*,   mark size=1.2pt\\
}

% longueur des major ticks
\pgfmathsetlengthmacro\MajorTickLength{
	\pgfkeysvalueof{/pgfplots/major tick length} * 0.75
}
% longueur des minor ticks
\pgfmathsetlengthmacro\MinorTickLength{
	\MajorTickLength * 0.5
}

\pgfplotsset{
    /pgfplots/layers/Bowpark/.define layer set={
        axis background,axis grid,main,axis ticks,axis lines,axis tick labels,
        axis descriptions,axis foreground
    }{/pgfplots/layers/standard},
    /pgfplots/layers/mylayerset/.define layer set={
        axis background,axis grid,axis ticks,axis lines,main,axis tick labels,
        axis descriptions,axis foreground
    }{/pgfplots/layers/standard},
}


\pgfplotsset{
    ylabel right/.style={
        after end axis/.append code={
            \node [rotate=90, anchor=north] at (rel axis cs:1,0.5) {#1};
        }   
    },
    ylabelv right/.style={
        after end axis/.append code={
            \node [rotate=0, anchor=west] at (rel axis cs:1,0.5) {#1};
        }   
    }
}

\pgfplotsset{
	clip marker paths=true,%
	axis on top=false,%true,%
	set layers=Bowpark,%
	cycle multiindex* list={
		plotcycle_1
		\nextlist
		thick
		\nextlist
		mark options={scale=.85}%
	},%
	clip mode=individual,%
	axis line style={line width=0.5pt},%
	grid style={line width=0.3pt, draw=black!13},%
	major grid style={line width=0.4pt,draw=black!25},%
	every tick/.style={
        black,
        line width=0.5pt
      },%
    major tick length=\MajorTickLength,%
    minor tick length=\MinorTickLength,%
	legend cell align={left},%
	legend style={line width=0.5pt}%
}
%%%%%%%%%%%%%%%%%%%%%%%%%%%%%%%%%%%%%%%%%%%%%%%%%%%%%%


% controle des objets flottants (figures, tables)
\renewcommand{\topfraction}{0.7}     % autorise 70% page de graphique en haut
\renewcommand{\bottomfraction}{0.5}  % autorise 50% page de graphique en bas
\renewcommand{\floatpagefraction}{0.7}
\renewcommand{\textfraction}{0.1}
%%%%%%%%%%%%%%%%%%%%%%%%%%%%%%%%%%%%%%%%%%%%%%%%%%%%%%



\newlength{\imagewidth}
\newlength{\imageheight}
