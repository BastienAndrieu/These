\chapter{Construction de boîtes englobantes orientées}
\label{app:obb}

cf. notes ``Intersections''

\section{Définition}
On définit une boîte orientée de centre $\vrm{c}$, de demi-côtés $\family{a}{i}{1}{3}$ et d'axes (orthonormés) $\family{\vrm{e}}{i}{1}{3}$ comme l'ensemble des points $\bx \in \mathbb{R}^3$ tels que
\begin{equation}
	\left| \dotprod{\left(\bx - \vrm{c}\right)}{\vrm{e}_i} \right| \leq a_i,
	\label{eq:def_obb}
\end{equation}
pour $i = 1,\ldots,3$.\par
    Une telle boîte est un polyèdre convexe, on peut donc appliquer le \textit{théorème des axes séparateurs} afin de déterminer si l’intersection de deux boîtes orientées est vide ou non \cite{eberly2002}.
    
\section{Boîte englobant une courbe}
\ldots

\section{Boîte englobant une surface}
\ldots