\chapter{Reparamétrisation affine}%{Changement de variable affine}%
\label{app:cdv_cheb}
%
%cf. notes ``Changement de variable -- Chebyshev''
%
\def\axpb{\alpha x + \beta}

Dans cette annexe, on détaille le calcul de la reparamétrisation affine exploitée dans la \autoref{sec:calcul_intersections} pour subdiviser des courbes et surfaces paramétriques polynomiales exprimées dans la base de Chebyshev.
\par\bigskip
Soit $p$ un polynôme de degré au plus $N$ exprimé dans la base de Chebyshev
\begin{equation*}
	p(x) = \sum_{n=0}^N \hat{p}_n T_n(x).
\end{equation*}
%
Soient $-1 \leq a, b \leq 1$ et $r$ le changement de variable affine
\begin{equation*}
	r(x) = \frac{b - a}{2}x + \frac{b + a}{2} = \axpb .
\end{equation*}
qui transforme le segment $\chebinterval$ en le segment $\left[a,b\right]$.
\par
$q \equiv p \circ r$ est également un polynôme de degré au plus $N$
\begin{equation*}
	q(x) = \sum_{n=0}^N \hat{p}_n T_n(\axpb) 
	= \sum_{n=0}^N \hat{q}_n T_n(x).
\end{equation*}
La reparamétrisation consiste ainsi à déterminer les coefficients $\family{\hat{q}}{n}{0}{N}$.\par
%
On pose
\begin{equation}
	T_n(\axpb) = \sum_{k=0}^{n} \lambda_{k,n} T_k(x).
	\label{eq:combinaison_cdv}
\end{equation}
($T_n(\axpb)$ est un polynôme de degré au plus $n$ en $x$, donc pour tout $k > n$, $\lambda_{k,n} = 0$.)
\par
Il vient alors
\begin{align*}
	\sum_{n=0}^N \hat{q}_n T_n(x)
	&= \sum_{n=0}^N \sum_{k=0}^{n} \lambda_{k,n} \hat{p}_n T_k(x), \\
	&= \sum_{k=0}^N \sum_{n=n}^{N} \lambda_{k,n} \hat{p}_n T_k(x),
\end{align*}
%\begin{align*}
%\sum_{n=0}^N \hat{h}_n T_n(x) &= \sum_{n=0}^{N} \sum_{k=0}^{N} \lambda_{k,n} \hat{f}_n T_k(x) \\
%&= \sum_{k=0}^{N} \sum_{n=0}^{N} \lambda_{k,n} \hat{f}_n T_k(x) \\
%&= \sum_{n=0}^{N} \sum_{k=0}^{N} \lambda_{n,k} \hat{f}_k T_n(x), % on intervertit les indices (muets) k et n dans le membre de droite
%\end{align*}
ainsi, par unicité de la décomposition de $q$ dans la base $\family{T}{n}{0}{N}$ de $\polyspace{N}$,
\begin{equation}
	\hat{q}_n = \sum_{k=n}^{N} \lambda_{n,k} \hat{p}_k.
\end{equation}
%
%Il nous reste ainsi à déterminer les coefficients $\ffamily{\lambda}{n}{0}{N}{k}{0}{N}$.
Il nous reste ainsi à déterminer les coefficients $\left\{ \lambda_{i,j} \right\}_{0 \leq i,j \leq N}$.
\par\bigskip
On a d'abord
\begin{equation*}
	\begin{eqsys}
		T_0(\axpb) = 1 = T_0(x),\\
		T_1(\axpb) = \axpb = \beta T_0(x) + \alpha T_1(x),
	\end{eqsys}
\end{equation*}
soit $\lambda_{0,0} = 1$, $\lambda_{0,1} = \beta$ et $\lambda_{1,1} = \alpha$.
\par
D'après la relation de récurrence \eqref{eq:recurrence_chebyshev}, on a également, pour $n \geq 2$,
\begin{equation}
	T_n(\axpb) =
	2 \left(\axpb\right) T_{n-1}(\axpb) 
	- T_{n-2}(\axpb).
	\label{eq:recurrence_axpb}
\end{equation}
%
Pour tous entiers $i$ et $j$, $2 T_i T_j = T_{i+j} + T_{\left|i - j\right|}$. On a donc%, en injectant \eqref{eq:combinaison_cdv} dans \eqref{eq:recurrence_axpb},
\begin{align*}
	2 \alpha x T_{n-1}(\axpb) &=
	2 \alpha \sum_{k=0}^{n-1} \lambda_{k,n-1} T_1(x) T_k(x) \\
	&= \alpha \sum_{k=0}^{n-1} \lambda_{k,n-1} \left(T_{k+1}(x) + T_{\left|k-1\right|}(x)\right).
\end{align*}
%
On obtient ainsi, en injectant \eqref{eq:combinaison_cdv} dans \eqref{eq:recurrence_axpb}
\begin{align*}
	\sum_{k=0}^{n} \lambda_{k,n} T_k(x) = 
	&\alpha \left(
	\sum_{l=1}^{n} \lambda_{l,n-1} T_l(x) +
	\sum_{j=0}^{n-2} \lambda_{j,n-1} T_j(x) +
	\lambda_{0,n-1} T_1(x)
	\right), \\
	&+ 2 \beta \sum_{k=0}^{n-1} \lambda_{k,n-1} T_k(x)
	- \sum_{k=0}^{n-2} \lambda_{k,n-2} T_k(x).
\end{align*}
%
Soit enfin, pour $n \geq 2$ et $k \leq n$,
\begin{equation}
	\lambda_{k,n} = 
	\alpha \left(
	\lambda_{k-1,n-1} + \lambda_{k+1,n-1} + \delta_{k,1} \lambda_{0,n-1}
	\right)
	+ 2 \beta \lambda_{k,n-1} - \lambda_{k,n-2}.
\end{equation}
(On pose $\lambda_{k,n} = 0$ lorsque $k < 0$ et $\delta_{\cdot,\cdot}$ représente le symbole de Kronecker).\par
%
La complexité de cette reparamétrisation est comparable à celle de l'algorithme de Casteljau, utilisé pour subdiviser des courbes et surfaces de Bézier.