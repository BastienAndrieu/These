\chapter*{Notations}

\begin{tabular}{ll}
temps                               & $t$                        \\
intervalle/pas de temps             & $\tau$/$\Delta t$          \\
scalaire                            & $abcdefghopqrwxyz \alpha \beta \gamma \delta \epsilon \varepsilon \zeta \eta \theta \vartheta \iota \lambda \mu \xi \varpi \rho \varrho \sigma \varsigma \tau \varphi \chi \psi \omega$\\
indice                              & $ijklmn$\\ \hline
\textbf{Géométrie} & \\
courbe dans $\reals^3$ (variété)    & $\Gamma$                   \\
courbe dans $\reals^2$ (variété)    & $\Psi$                     \\
surface (variété)                   & $\Sigma$                   \\
normale unitaire                    & $\unv$                     \\
point de $\reals^d$                 & $\vit{abcdefghijklmnopqrstuvwxyz}$\\
vecteur de $\reals^d$               & $\vrm{abcdefghijklmnopqrstuvwxyz}$\\ \hline
\textbf{Enveloppes de sphères} & \\
vitesse normale                     & $\nu$                      \\ \hline
enveloppe de sphères (variété)      & $\Phi$ \\
enveloppe de sphères (paramétrisation) & $\vit{\phi}(u,v)$ \\
rayon des sphères                   & $\rho(w), \rho(u,v)$ \\
sphère                              & $\mathscr{S}$ \\
boule                               & $\mathscr{B}$ \\ 
cercle                              & $\mathscr{C}$ \\ 
plan                                & $\mathscr{P}$ \\ \hline
\textbf{Entités \brep} & \\
coquille (\eng{shell})              & \brepshell, ensemble : $\mathcal{S}$ \\
face                                & \brepface, ensemble : $\mathcal{F}$ \\
arête (\eng{edge})                  & \brepedge, ensemble : $\mathcal{E}$ \\
co-arête (\eng{half-edge})          & $\mathrm{H}$, ensemble : $\mathcal{H}$, ou $\brepedge^{1,2}$ \\
sommet (\eng{vertex})               & \brepvertex, ensemble : $\mathcal{V}$ \\
contour (\eng{wire})                & \brepwire, ensemble : $\mathcal{W}$ ($\brepwire^{\mathrm{ext}}, \brepwire^{\mathrm{int},i}$) \\ \hline
\textbf{Éléments de maillage} & \\
triangle                            & $K$ ou $\tau$\\
face                                & $F$\\
arête (\eng{edge})                  & $?$\\
demi-arête (\eng{half-edge})        & $?$\\
sommet/n\oe ud                      & $S$\\ \hline
\textbf{Polynômes} & \\
espace des polynômes de degré $\leq N$ & $\mathcal{P}_N$ ou $\reals_N[X]$, $\reals_N[x_1, \ldots, x_d]$\\
polynômes de Chebyshev              & $T_n$                      \\
polynômes de Bernstein              & $B_n^{N}$                  \\
\end{tabular}

\begin{tabular}{ll}
\textbf{Méthode (pseudo-)spectrale} & \\
degré                               & $L,M,N,P,Q$                \\
somme partielle de Chebyshev        & $\truncseries{f}{N}$       \\
polynôme d'interpolation            & $\interpolant{f}{N}$       \\
dérivées du polynôme d'interpolation & $\interpderiv{f}{N}{k}$   \\
coefficients de $\truncseries{f}{N}$ & $\hat{f}_n$               \\
coefficients de $\interpolant{f}{N}$ & $\tilde{f}_n$             \\
coefficients de $\interpderiv{f}{N}{k}$ & $\deriv{\tilde{f}}{k}_n$ \\
matrice de différentiation          & $\mathbf{D}_N$             \\ \hline
\textbf{Algèbre, calcul différentiel} & \\
fonctions continues sur $[a,b]$     & $\matholdcal{C}[a,b]$ \\
fonctions régulières sur $[a,b]$    & $\matholdcal{C}^k[a,b]$, $k \in \integers$ ou $k= \infty$ \\
matrice                             & $\mathbf{ABCDEFGKLMNOPQRSTUVWXYZ \Theta \Lambda}$\\
matrice identité                    & $\mathbf{I_d}$\\
matrice Jacobienne                  & $\jacobian{\!f}$, $\jacobian{\bs}$, $\jacobian{\bg}$\\
matrice Hessienne                   & $\hessian{f}$              \\ \hline
\textbf{Géométrie différentielle} & \\
paramètre de courbe                 & $w$                        \\
paramètres de surface               & $u,v$                      \\
courbe paramétrique dans $\reals^3$ & $\bg(w)$                   \\
courbe paramétrique dans $\reals^2$ & $\vit{\psi}(w)$            \\
surface paramétrique                & $\bs(u,v)$                 \\
tenseur métrique                    & $\fff
= \begin{pmatrix}E & F\\ F & G\end{pmatrix}
= \begin{pmatrix}I_{1,1} & I_{1,2}\\ I_{1,2} & I_{2,2}\end{pmatrix}$\\
matrice de la 2nde forme fondamentale & $\sff
= \begin{pmatrix}L & M\\ M & N\end{pmatrix}
= \begin{pmatrix}\sffc_{1,1} & \sffc_{1,2}\\ \sffc_{1,2} & \sffc_{2,2}\end{pmatrix}$\\
courbures principales               & $\kappa_{1,2}$             \\
courbure moyenne                    & $H = (\kappa_1 + \kappa_2)/2$\\
courbure de Gauss                   & $K = \kappa_1 \kappa_2$    \\
domaine paramétrique                & $\Omega$, \ie $\Sigma = \bs(\Omega)$\\
\end{tabular}

%\bigskip
\vfill
\pagebreak

%\textsf{abcdefghijklmnopqrstuvwxyz}\par
%abcdefghijklmnopqrstuvwxyz\par
%$\mathrm{abcdefghijklmnopqrstuvwxyz}$\par
%$abcdefghijklmnopqrstuvwxyz$\par
%$\mathbf{abcdefghijklmnopqrstuvwxyz}$\par
%$\boldsymbol{abcdefghijklmnopqrstuvwxyz}$\par
%$\mathscr{abcdefghijklmnopqrstuvwxyz}$\par

\begin{tabular}{ll}
abcdefghijklmnopqrstuvwxyz & text \\
$abcdefghijklmnopqrstuvwxyz$ & math \\
$\mathrm{abcdefghijklmnopqrstuvwxyz}$ & mathrm \\
$\mathbf{abcdefghijklmnopqrstuvwxyz}$ & mathbf \\
$\boldsymbol{abcdefghijklmnopqrstuvwxyz}$ & boldsymbol \\
$\mathcal{abcdefghijklmnopqrstuvwxyz}$ & mathcal \\
$\matholdcal{abcdefghijklmnopqrstuvwxyz}$ & matholdcal \\
$\mathscr{abcdefghijklmnopqrstuvwxyz}$ & mathscr
\end{tabular}


\bigskip

\begin{tabular}{ll}
ABCDEFGHIJKLMNOPQRSTUVWXYZ & text \\
$ABCDEFGHIJKLMNOPQRSTUVWXYZ$ & math \\
$\mathrm{ABCDEFGHIJKLMNOPQRSTUVWXYZ}$ & mathrm \\
$\mathbf{ABCDEFGHIJKLMNOPQRSTUVWXYZ}$ & mathbf \\
$\boldsymbol{ABCDEFGHIJKLMNOPQRSTUVWXYZ}$ & boldsymbol \\
$\mathcal{ABCDEFGHIJKLMNOPQRSTUVWXYZ}$ & mathcal \\
$\matholdcal{ABCDEFGHIJKLMNOPQRSTUVWXYZ}$ & matholdcal \\
$\mathscr{ABCDEFGHIJKLMNOPQRSTUVWXYZ}$ & mathscr
\end{tabular}
\bigskip

$\alpha \beta \gamma \delta \epsilon \varepsilon \zeta \eta \theta \vartheta \iota \kappa \lambda \mu \nu \xi \pi \varpi \rho \varrho \sigma \varsigma \tau \upsilon \phi \varphi \chi \psi \omega$\par
$\boldsymbol{\alpha \beta \gamma \delta \epsilon \varepsilon \zeta \eta \theta \vartheta \iota \kappa \lambda \mu \nu \xi \pi \varpi \rho \varrho \sigma \varsigma \tau \upsilon \phi \varphi \chi \psi \omega}$\par
$\Gamma \Delta \Theta \Lambda \Xi \Pi \Sigma \Upsilon \Phi \Psi \Omega$\par

%$\boldsymbol{\Gamma \Delta \Theta \Lambda \Xi \Pi \Sigma \Upsilon \Phi \Psi \Omega}$\par
\bigskip
\begin{tabular}{ccccccccccc}
$\gamma$ & $\delta$ & $\theta$ & $\lambda$ & $\xi$ & $\pi$ & $\sigma$ & $\upsilon$ & $\phi$ & $\psi$ & $\omega$\\
$\vit{\gamma}$ & $\vit{\delta}$ & $\vit{\theta}$ & $\vit{\lambda}$ & $\vit{\xi}$ & $\vit{\pi}$ & $\vit{\sigma}$ & $\vit{\upsilon}$ & $\vit{\phi}$ & $\vit{\psi}$ & $\vit{\omega}$\\
$\Gamma$ & $\Delta$ & $\Theta$ & $\Lambda$ & $\Xi$ & $\Pi$ & $\Sigma$ & $\Upsilon$ & $\Phi$ & $\Psi$ & $\Omega$
\end{tabular}

\par\bigskip
