\chapter{Généralisation aux surfaces régulières par morceaux de topologie quelconque}
\chaptermark{Généralisation aux surfaces de topologie quelconque}

\section{Construction des nouvelles faces}
\cite{peternell1997}

paramétrisation et limites du domaine paramétrique

\subsection{Arêtes convexes}
paramétrisation approchée des courbes d'intersection (choix du paramètre, fitting LS série de Chebyshev univariée), portion de \eng{canal surface} approchée 

\subsection{Sommets convexes}
deux variantes :

\subsubsection{Une face trimmée}
$(u,v) \equiv$ coordonnées sphériques\\
+ : 1 seule face, spectre Chebyshev étroit


\subsubsection{Plusieurs faces non trimmées}
polygone sphérique découpé en quads \cite{hahn1989}\\
+ : domaine paramétrique non trimmé


\section{Résolution des intersections entre faces}
\subsection{État de l'art}
état de l'art méthodes d'intersection de surfaces paramétriques :
review complète \cite{patrikalakis2009}\\
subdivision \cite{houghton1985}, implicitisation approchée \cite{dokken2001}, suivi (marching) \cite{barnhill1990}, Hohmeyer \cite{hohmeyer1992} (critère de détection de boucles sur les enveloppes de normales, paramétrisation monotone et tracé des branches d'intersection)


\subsection{Approche retenue}
\cite{peternell1997}
adaptation de l'approche de Hohmeyer \cite{hohmeyer1992} (full Chebyshev / Chebyshev-Bernstein)
\begin{itemize}
	\item changement de base \cite{rababah2003}
	\item subdivision : changement de variable (détail en annexe) / algorithme de Casteljau (ref?) (remarques complexité et conditionnement)
	\item volumes englobants convexes : oriented bounding box \cite{fournier1994} (détail en annexe) / enveloppe convexe (comparer volumes)
	\item test de séparation : théorème de séparation des convexes \cite{eberly2002} / optimisation linéaire \cite{seidel1991}
	\item traitement des intersections tangentielles
\end{itemize}



\section{Validation de la méthode}

\subsection{Propagation suivant un champ de vitesse continu}
sphère/cube dans un écoulement tourbillonnaire incompressible analytique de période temporelle $2T$\\
calcul (exact) de volume avec quadrature de Clenshaw-Curtis (étanchéité (continuité $G^0$ partout) garantie car les marqueurs de bord coïncident tout au long de la déformation)\\
convergence de l'erreur d'approximation sur la position, l'aire et le volume à $t = 0$ et $t = T$ pour différents niveaux de discrétisations spatiale et temporelle\\
+ convergence de la variation de volume au cours de la déformation

\subsection{Propagation à vitesse normale uniforme}
cube en expansion

\subsection{Propagation à vitesse normale non uniforme}
?
