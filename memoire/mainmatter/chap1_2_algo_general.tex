\chapter[Algorithme de propagation d'interfaces régulières \piecewise]{Algorithme général pour la propagation d'interfaces régulières \piecewise}
\label{chap:algo_general}

Objectif : proposer un algorithme de construction de l'EdB sous la forme d'un nouveau modèle \brep\ à partir du modèle \brep\ de l'interface courante. 
Pour cela, on tire d'abord des observations sur lesquelles on se basera pour construire dans un premier temps une pseudo-EdS, que l'on traitera dans un second temps afin d'obtenir l'EdB. 
Dans ce chapitre, on explicitera une partie des algorithmes utilisés mais les détails de la mise en \oe uvre numérique feront l'objet du chapitre suivant.

\setlength{\imagewidth}{74mm}
\setlength{\imageheight}{\imagewidth}
\def\trmask{0.82}
\begin{figure}
  \centering
  \tikzset{x=\imagewidth, y=\imageheight,
  	img/.style={anchor=south west, inner sep=0}}
  %
  \hspace*{\fill}
  \subbottom[Modèle \brep\ initial.]{
	\begin{tikzpicture}
		\figEoBBrep{1}{0}{0}
%		\draw[blue, dashed] 
%			(current bounding box.south west) --
%			(current bounding box.south east) --
%			(current bounding box.north east) --
%			(current bounding box.north west) -- cycle;
	\end{tikzpicture}
  }
  \hfill%
  \subbottom[Modèle \brep\ de l'enveloppe des boules.]{
	\begin{tikzpicture}
		\figEoBBrep{2}{0}{0}
%		\draw[blue, dashed] 
%			(current bounding box.south west) --
%			(current bounding box.south east) --
%			(current bounding box.north east) --
%			(current bounding box.north west) -- cycle;
	\end{tikzpicture}
  }
  \hspace*{\fill}
  \caption{???}
  %
\end{figure}

\section{Observations/Motivations}
Convexité des courbes singulières (foreshadowing Hohmeyer), EdB $\subset$ EdS ($\to$ \autoref{section:principe_huygens}?), notion d'EdS partielle suffisante (ou \guillemets{pseudo-EdS}, \cf notes Huygens)





\section{Construction d'une \guillemets{proto-\brep} de l'EdS partielle suffisante}
\label{section:def_canal_surface}
Construction des EdS propres des faces (carreaux restreints), arêtes convexes vives (arc/segment de courbe) et sommets vifs convexes, paramétrisation d'une EdS (propre) à 1 paramètre/2 paramètres ($\to$ \autoref{section:principe_huygens}?)




\section{Conversion en \brep\ de l'EdB}

\subsection{Construction du graphe des intersections}
Conservation/création des intersections tangentielles, calcul des intersections transverses entre paire de carreaux non-restreints, segmentation des courbes d'intersection en segments quasi-disjoints (intersection de 3 carreaux non-restreints ou plus), \guillemets{clipping} par le domaine paramétrique de chaque carreau restreint
\par\bigskip
pour un carreau de surface, il s'agit d'un graphe planaire orienté dont un plongement dans $\reals^2$ est donné par la trace des courbes d'intersections dans son espace paramétrique

\subsection{Construction des faces, arêtes et sommets \brep}
pour chaque carreau : extraction des cycles du graphe d'intersection, caractérisations des contours (extérieurs/intérieurs), détermination des faces $\to$ insertion des contours dans la \brep\ $\to$ insertion des (co-)arêtes et sommets
\begin{figure}
	\centering
	\newcommand{\arc}{\mathcal{A}}%
\newcommand{\subin}{\ensuremath{_{\mathrm{in}}}}%
\newcommand{\subout}{\ensuremath{_{\mathrm{out}}}}%
\newcommand{\noeud}{\mathcal{N}}%
\newcommand{\cycle}{\mathcal{C}}%
\newcommand{\graph}{\mathcal{G}}%
\begin{tikzpicture}[%
	scale=0.5,
	>={Latex[length=4pt]},      % Arrow style
    start chain=going below,    % General flow is top-to-bottom
    node distance=6mm and 50mm, % Global setup of box spacing
    every join/.style={flow},   % Default linetype for connecting boxes
    ]
% ------------------------------------------------- 
% A few box styles 
% <on chain> *and* <on grid> reduce the need for manual relative
% positioning of nodes
\tikzset{
  base/.style={draw, on chain, on grid, align=center, minimum height=4ex, minimum width=5em},
  proc/.style={base, rectangle},
  test/.style={base, diamond, aspect=2},
  term/.style={proc, rounded corners=2ex},
  % coord node style is used for placing corners of connecting lines
  coord/.style={coordinate, on chain, on grid, node distance=6mm and 25mm},
  % -------------------------------------------------
  % Connector line styles for different parts of the diagram
  flow/.style={->, draw}
}
% -------------------------------------------------
\node [term, join] (start) {Début};
\node [proc, join] (p1) {Éliminer les branches pendantes de $\graph$};
\node [test, join] (t1) {$A = \emptyset$ ?};

\node [proc] (startarc) {Choisir un arc $\arc_*$ de $\graph$};

\node [proc, join] {Démarrer un nouveau cycle $\cycle$ à partir de $\arc_*$};
\node [proc, join] {$\arc \leftarrow \arc_*$ et\\ $\noeud \leftarrow \dest(\arc)$};


\node [test, join] (t2) {$\noeud = \orig(\arc_*)$?};

\node [proc] (maxangles) {Identifier $\hi{\alpha}\subin$, $\hi{\alpha}\subout$ et $\hi{\arc}\subout$};


\node [test, join] (t3) {$\hi{\alpha}\subin > \hi{\alpha}\subout$ ?};
\node [proc] (abortcycle) {Abandonner $\cycle$};

\node [term, left=of t1, text width=3em] (end) {Fin};
\node [proc, right=of t2] (completecycle) {Rajouter $\cycle$\\à la liste des cycles\\et l'extraire de $\graph$};
\node [proc, left=of t3] (appendcycle) {$\arc \leftarrow \hi{\arc}\subout$,\\ $\noeud \leftarrow \dest(\arc)$ et \\ajouter $\arc$ à $\cycle$};


\draw [flow] (t1.west) -- node[above] {oui} (end);
\draw [flow] (t1.south) -- node[left] {non} (startarc);

\draw [flow] (t2.east) -- node[above] {oui} (completecycle);
\draw [flow] (t2.south) -- node[left] {non} (maxangles);

\draw [flow] (t3.west) -- node[above] {oui} (appendcycle);
\draw [flow] (t3.south) -- node[left] {non} (abortcycle);


\draw [flow] (appendcycle.north) |- (t2);

\node [coord, left=of appendcycle] (c2)  {};
\draw [flow] (abortcycle.west) -| (c2) |- (p1);

\draw [flow] (completecycle.north) |- (p1);
% -------------------------------------------------
\end{tikzpicture}
	\caption{Organigramme de l'algorithme d'extraction des cycles d'un graphe orienté plongé dans $\reals^2$.}
\end{figure}