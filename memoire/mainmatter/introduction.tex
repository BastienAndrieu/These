\chapter*{Introduction}
\chaptermark{Introduction}
\addcontentsline{toc}{chapter}{Introduction}


\section*{Contexte}
\begin{itemize}
	\item Combustion/érosion/ablation de solides
	\item dépôt (e.g. accrétion de givre, fabrication additive)
	\item interaction fluide-structure (e.g. aéroélasticité)
	\item décalage (offset) de surface
	\begin{itemize}
		\item définition de tolérances
		\item usinage par machine-outil à commande numérique
	\end{itemize}	 
	\item génération de maillage volumique par avancée de front (e.g. couche limite pour écoulements NS à haut Reynolds)
\end{itemize}

\section*{Cadre/Problématiques de la thèse}
\begin{enumerate}
	\item Résoudre la propagation d'interfaces \troisD \emph{géométriquement régulières par morceaux} et de \emph{genre topologique quelconque} 
	\begin{itemize}
		\item on ne s'intéresse pas aux interfaces fluides (e.g. écoulements multi-phasiques, vésicules en suspension, …) dont la propagation est régie par la tension de surface, un flot de courbure ou de Willmore et qui subissent de grands changements de topologie
	\end{itemize}
	
	\item Réaliser des simulations numériques (de type EF/VF) de phénomènes multi-physiques complexes [mettant en jeu des interfaces en propagation]/[tenant compte des déformations dynamiques de le géométrie résultant de la propagation d'interfaces]
\end{enumerate}



\section*{Contributions}

\begin{enumerate}
	\item Utilisation du \emph{formalisme \brep} pour représenter une interface \textit{régulière par morceaux} en propagation en \troisD (dans des applications de type interaction fluide-structure, combustion de solide dans un fluide).
	
	\item Mise au point d’un algorithme basé sur le \emph{principe de Huygens} (avec condition d’entropie) pour \emph{adapter dynamiquement la géométrie et la topologie} du modèle \brep de l’interface au cours de la propagation.
	
	\item Mise en \oe uvre d’une méthode pseudo-spectrale (i.e. d’ordre élevé) (utilisant les polynômes de Chebyshev comme fonctions de base) pour suivre \emph{efficacement et avec une grande précision} le mouvement de l’interface.
	
	\item Mise en œuvre d’une méthodologie pour adapter un \emph{maillage dynamique géométriquement fidèle au modèle \brep dynamique} de l’interface, dans le but de réaliser des simulations EF/VF.
	
	\item (Intégration de l'outil de suivi de surface dans une chaîne de calcul multi-physique)
\end{enumerate}



\section*{Organisation du manuscrit}
2 parties
\begin{enumerate}
	\item Développement d'un outil de suivi de surface adapté aux géométries régulières par morceaux
	\begin{enumerate}
		\item Formulation mathématique du problème de propagation d'interfaces régulières par morceaux en trois dimensions
		\item Formulation d'un algorithme basé sur le PHCE pour adapter dynamiquement la géomérie et la topologie du modèle BRep de l'interface au cours de la propagation
		\item Mise en \oe uvre numérique et validation de l'algorithme
	\end{enumerate}
	
	\item Intégration de l'outil de suivi de surface dans une chaîne de calcul multi-physique
	\begin{enumerate}
		\item Méthodologie pour adapter un maillage dynamique géométriquement fidèle au modèle \brep dynamique de l'interface
		\item Couplage de l'outil de suivi de surface avec des solveurs EF/VF -- Application à la simulation de la combustion de propergol solide dans les moteurs de fusée
	\end{enumerate}
\end{enumerate}