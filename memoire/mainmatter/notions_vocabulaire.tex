\newcommand{\interface}{\ensuremath{\mathcal{S}}}
\newcommand{\interfacebrep}{\tilde{\interface}}

\newcommand{\solide}{\ensuremath{\Omega}}
\newcommand{\region}{\ensuremath{\matholdcal{R}}}
\newcommand{\regionB}{\ensuremath{\matholdcal{Q}}}

\newcommand{\nappe}{\ensuremath{\matholdcal{N}}}
\newcommand{\crete}{\ensuremath{\matholdcal{C}}}
\newcommand{\pic}{\matholdcal{P}}
\newcommand{\adherentsDimSup}[1]{\ensuremath{\mathcal{H}\!(#1)}}

\newcommand{\point}{\ensuremath{\bm{p}}}

\newcommand{\carreau}{\ensuremath{\Sigma}}

\newcommand{\maillage}{\mathcal{M}}
\newcommand{\noeudxyz}{\vit{v}}
\newcommand{\noeud}{\matholdcal{V}}%\bm{v}}
\newcommand{\chaine}{\mathcal{K}}

%\chapter*{Notions et vocabulaire}
%
%\section*{Carreau paramétrique}
%Carreau $\carreau = (\fullUVdomain = \intervalLoHi{u} \times \intervalLoHi{v}, \bs)$
%\begin{itemize}
%	\item Arc/courbe iso-paramétrique du bord de $\carreau$ ($u = \lo{u}$ ou $u = \hi{u}$ ou $v = \lo{v}$ ou $v = \hi{v}$) :
%	\begin{itemize}
%		\item côté de $\carreau$
%	\end{itemize}
%	
%	\item intersection de deux côtés de $\carreau$ ($(u,v) = (\lo{u}, \lo{v})$ ou $(u,v) = (\hi{u}, \lo{v})$ ou $(u,v) = (\hi{u}, \hi{v})$ ou $(u,v) = (\lo{u}, \hi{v})$) :
%	\begin{itemize}
%		\item coin de $\carreau$
%	\end{itemize}
%\end{itemize}
%
%\section*{Arc paramétrique}
%Arc
%\begin{itemize}
%	\item dans $\reals^3$ : $\Gamma = (\fullWdomain = \intervalLoHi{w}, \bg)$
%	\item dans $\reals^2$ : $\Psi = (\fullWdomain, \bp)$
%\end{itemize}
%
%\begin{itemize}
%	\item Points $\bg(\lo{w})$ et $\bg(\hi{w})$ (resp. $\bp(\lo{w})$ et $\bp(\hi{w})$) :
%	\begin{itemize}
%		\item extrémités de $\Gamma$ (resp. $\Psi$)
%	\end{itemize}
%\end{itemize}
%
%
%\section*{Description et représentation des surfaces régulières \piecewise}
%Surface (variété de dimension 2) $\interface$, $\contgeom{1}$ \piecewise
%\begin{itemize}
%	\item sous-variété de dimension 2 de $\interface$, connexe, de continuité $\contgeom{1}$ :
%	\begin{itemize}
%		\item nappe (régulière/$\contgeom{1}$) (de $\interface$), notée $\nappe$
%		\item[+] \textit{maximale} si n'est contenue dans aucune autre ($\Rightarrow$ quasi-disjointe avec toutes les autres)
%	\end{itemize}
%	\item sous-variété de dimension 1 de $\interface$, connexe, de continuité $\contgeom{1}$, formée exclusivement de points irréguliers de $\interface$ :
%	\begin{itemize}
%		\item crête (de $\interface$), notée $\crete$
%		\item courbe singulière (de $\interface$)
%		\item[+] \textit{maximale} si ne peut être contenue dans aucune autre ($\Rightarrow$ quasi-disjointe avec toutes les autres)
%	\end{itemize}
%	\item point irrégulier de $\interface$ qui n'est à l'intérieur d'aucune nappe ou crête
%	\begin{itemize}
%		\item pic (de $\interface$)
%		\item coin (de $\interface$)
%		\item point singulier (de $\interface$)
%		\item pointe (de $\interface$)
%	\end{itemize}
%	\item (unique) décomposition de $\interface$ en nappes maximales, crêtes maximales et coins :
%	\begin{itemize}
%		\item modèle maximal de $\interface$ (MM)
%		\item décomposition maximale de $\interface$ (DM)
%		\item représentation maximale de $\interface$ (MRep?)
%	\end{itemize}
%\end{itemize}
%
%\section*{Propagation d'interface}
%\begin{itemize}
%	\item interface $\interface = \boundary{\solide}$
%	\item domaine intérieur ($\sim$ solide) $\solide$ (ouvert)
%	\item domaine extérieur (complémentaire) $\complement{\solide}$
%\end{itemize}
%
%\subsection*{Principe de Huygens}
%$\region \subset \interface$
%\begin{itemize}
%	\item sphère centrée en $\point$ et de rayon $r$ notée $\sphere[\point][r]$
%	\item $\bigcup_{\point \in \region} \sphere[\point][\rho(\point)]$ : ?, noté $\sphere[\region][\rho]$
%	\item boule ouverte centrée en $\point$ et de rayon $r$ notée $\ball{\point}{r}$
%	\item $\bigcup_{\point \in \region} \ball{\point}{\rho(\point)}$ : ?, noté $\ball{\region}{\rho}$
%	\item $\boundary{\ball{\region}{\rho}}$ : enveloppe des boules (EdB), notée $\EdB{\region}{\rho}$
%	\item surface tangente à toutes les sphères de $\sphere[\region][\rho]$ et en tout point tangente à une sphère de $\sphere[\region][\rho]$ : enveloppe des sphères (EdS), notée $\EdS{\region}{\rho}$
%	\item $\closure{ {\EdS{\region}{\rho} \setminus \EdS{\boundary{\region}}{\rho}} }$ : EdS propre de $\region$, notée $\EdSpropre{\region}{\rho}$
%	\item $\EdB{\interface}{\rho} \cap \sphere[\region][\rho]$ : zone d'influence sur l'EdB, notée $\influEdB{\region}{\rho}$
%	%\item $\left\{ \regionB \subseteq \Sigma \mid \region \in \closure{\regionB} \text{ (et } \dim \regionB > \dim \region \text{)} \right\}$ : ?, noté $\adherentsDimSup{\region}$
%	\item $\left\{ \regionB \subseteq \Sigma \mid \region \in \boundary{\regionB}  \right\}$ : ?, noté $\adherentsDimSup{\region}$
%	\item $\EdS{\region}{\rho} \bigcap_{\regionB \in \adherentsDimSup{\region}} \EdS{\regionB}{\rho}$ (\ie lieu des points de tangence entre l'EdS de $\region$ et les EdS des entités de dimension $> \dim \region$ adhérentes à $\region$) : pseudo-EdS de $\region$, notée $\pseudoEdS{\region}{\rho}$
%\end{itemize}
%
%\subsection*{\brep}
%\begin{enumerate}
%	\item modèle \brep, noté \brepbody
%	\item sommet vif/régulier, noté $\brepvertex$
%	\item arête vive/régulière, notée $\brepedge$
%	\item (co-)arête, notée $\brepedge^i$, ($i=1,2$)
%	\item face, notée $\brepface$
%	\item contour, noté $\brepwire$
%\end{enumerate}
%
%\subsection*{Maillage}
%\begin{itemize}
%	\item n\oe ud, noté $\bm{v}$
%	\item arête, notée $\mathrm{e}$
%	%\item (face(tte), notée $\mathrm{f}$)
%	\item triangle, noté $\mathrm{t}$
%	\item suite de n\oe uds connectés par des arêtes : chaîne, notée $\mathrm{C}$ 
%\end{itemize}
%
%\subsection*{Maillage basé sur un modèle \brep}
%\begin{itemize}
%	\item maillage carreau-par-carreau
%	\item maillage trans-carreaux (indépendant des frontières des carreaux du modèle \brep)
%\end{itemize}