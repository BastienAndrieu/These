\chapter{Méthode d'ordre élevé pour le suivi d'un patch rectangulaire de surface}

(méthode pseudo-spectrale : solution développée dans une base de fonctions globales et régulières, résidu annulé exactement en un nombre discret de points de collocation, donnant une EDO en temps par point de collocation)\\

suivi lagrangien de marqueurs/points de collocation situés sur la surface\\
\cite{peternell1997}

\section{Discrétisation spatiale}

Choix des fonctions de base et des marqueurs/points de collocation\\
\subsection{État de l'art}
\begin{itemize}
	\item harmoniques sphériques \cite{veerapaneni2011}, polynômes trigonométriques \cite{gueyffier2015} $\to$ contraintes topologiques (genre 0, sans bord ou périodique) (méthode de continuation \cite{bruno2007} pour s'affranchir de cette contrainte, mais complexe (POUs, \ldots) et jamais utilisé pour des surfaces en mouvement)
	\item CAO : courbes/surfaces polynomiales (algébriques)/rationnelles (en produit tensoriel) par morceaux (B-splines/NURBS), utilisant la base de Bernstein
	\begin{equation}
		B_n^N(x) = \binom{N}{n} \left( 1 - x \right)^{N-n} x^n.
	\end{equation}
	\begin{itemize}
		\item[+] coefficients = points de contrôle dans l'espace physique
		\item[+] partition de l'unité $\Rightarrow$ propriété d'enveloppe convexe
		\item[-] algorithme d'évaluation (de Casteljau) numériquement stable mais couteux $\bigO{N^2}$
		\item[-] points de contrôle pas \emph{sur} la courbe/surface $\Rightarrow$ pas exploitables comme marqueurs lagrangiens
		\item[-] peu pratiques pour réduire/élever le degré des polynômes
	\end{itemize}
\end{itemize}
\bigskip
on choisit les polynômes de Chebyshev, couramment employés dans les méthodes spectrales dans le cas non-périodique
\subsection{Polynômes de Chebyshev}
\cite{peternell1997}

\subsubsection{Définition}
	\begin{itemize}
		\item cos, récurrence, pour $x \in \chebinterval$ 
		\begin{equation*}
			T_0(x) = 1, \quad T_1(x) = x,
		\end{equation*}				
		et pour $n \geq 2$,
		\begin{equation}
			T_n(x) = 2x T_{n-1}(x) - T_{n-2}(x).
			\label{eq:recurrence_chebyshev}
		\end{equation}
		\item extrema (CGL) (minimise phénomène de Runge pour l'interpolation)
	\end{itemize}
	
\subsubsection{Développement en série de Chebyshev, approximation de fonctions}
fonctions d'une variable
\begin{itemize}
	\item orthogonalité $\Rightarrow$ aspect multirésolution, élévation/réduction de degré directe et quasi-optimale (d'ailleurs utilisé en CAO \cite{lachance1988})
	\item série tronquée $\truncseries{f}{N}$, erreur de troncature
	\item interpolant $\interpolant{f}{N}$, erreur d'aliasing
	\item CGL $\to$ DCT, transformation rapide
\end{itemize}
généralisation à plusieurs variables (produit-tensoriel)

\paragraph{Représentation de courbes et de surfaces.}
courbe $\bg : \chebinterval \to \mathbb{R}^3$
\begin{equation}
	\truncseries{\bg}{N}(t) = \sum_{n=0}^{N} \hat{\bg}_{n} T_n(t).
\end{equation}
surface $\bs : \chebinterval^2 \to \mathbb{R}^3$
\begin{equation}
	\truncseries{\bs}{MN}(u,v) = \sum_{m=0}^{M} \sum_{n=0}^{N} \hat{\bs}_{mn} T_m(u) T_n(v).
\end{equation}

\subsubsection{Évaluation}
algorithme de sommation de Clenshaw exploitant la relation de récurrence \eqref{eq:recurrence_chebyshev}) \cite{clenshaw1955}, numériquement stable et efficace ($\bigO{N}$)

\subsubsection{Dérivation}
\begin{itemize}
	\item matrice de dérivation (espace physique)
	\item formule de récurrence (espace spectral), remarque de \cite[Section 2.3, p~.94]{wengle1978}%, \cite[p.~46]{peyret2013}
\end{itemize}

\subsubsection{Quadrature}
Clenshaw-Curtis\\
calcul (approché) longueur de courbe/aire de surface

\subsubsection{Précision spectrale}
Si $f \in C^p(\chebinterval)$, alors $\max_{\chebinterval} \left| \truncseries{f}{N} - f \right| = \bigO{N^{1-p}}$ lorsque $N \to \infty$ \cite[Théorème 5.14]{mason2002}.\\
illustration convergence exponentielle pour $f$ analytique et décroissance exponentielle des $\hat{f}_n$



\section{Discrétisation temporelle}
\cite{peternell1997}
\begin{itemize}
	\item champ de vitesse connu : intégration classique, schéma explicite (Euler, RK)
	\item vitesse normale connue : approximation EdS
	\begin{itemize}
		\item calcul dérivées par différentiation spectrale
		\item calcul normale et composante tangentielle
		\item non-linéarités $\to$ aliasing, peut causer instabilité \cite{rahimian2015}
	\end{itemize}
	\item auto-intersections
	\begin{itemize}
		\item globales (traitées au chapitre 3)
		\item locales \cite{farouki1986}
	\end{itemize}
\end{itemize}