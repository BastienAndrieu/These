\chapter{Déformation de maillage surfacique}

\textit{on veut mettre au point une méthodologie pour déformer un maillage de la surface en propagation en utilisant le modèle \brep\ dynamique comme support géométrique, afin de pouvoir réaliser des simulations EF/VF dans des domaines de géométrie déformables.}

\section{Méthodes de simulation dans des géométries déformables}
État de l'art :
\begin{enumerate}
	\item maillage volumique conforme à l'interface
	\begin{enumerate}
		\item\label{item:ale} un seul maillage \eng{body-fitted} avec formulation ALE : frontière = maillage de l'interface, intérieur déformé de façon arbitraire (nécessite généralement de préserver la connectivité)
		\item plusieurs maillages \eng{body-fitted} qui se superposent (méthode Chimère \cite{meakin1989, wang2000}, FLUSEPA \cite{brenner1991} ) 
		\begin{itemize}
			\item[+] facilite la génération du maillage en volume lorsque la géométrie est complexe (\eg hyper-sustentateurs), et évite de déformer un maillage 3d
			\item[-] nécessite de traiter les intersections entre les blocs de maillage, limité aux mouvements rigides (?)
		\end{itemize}
	\end{enumerate}
	\item maillage volumique non conforme à l'interface (\eng{immersed/embedded boundary methods} \cite{peskin2002}) : fluide traité de façon eulérienne (maillage fixe), soit explicitement \cite{wang2012, hovnanian2012}{\color{gray},  soit implicitement \cite{bruchon2009}}
	\begin{itemize}
		\item[+] évite de générer et déformer un maillage 3d autour d'une géométrie complexe
		\item[-] application des conditions aux limites
	\end{itemize}
\end{enumerate}

ici, on se concentre sur les applications utilisant la méthodologie \ref{item:ale}, et on ne s'occupe que du maillage de la frontière/interface. On se limite également aux maillages triangulaires, linéaires par morceaux (mais extension aux maillages hybrides, courbes envisageable)

\section{Objectifs et enjeux}
\subsection{Objectifs}
\begin{enumerate}
	\item préserver la connectivité du maillage autant que faire se peut ($\Rightarrow$ déformation pure)
	\item maintenir une bonne qualité de maillage ($\Rightarrow$ métrique à définir, lissage/optimisation par déplacements tangentiels)
	\item le maillage doit représenter fidèlement l'interface ($\Rightarrow$ sommets localisés exactement sur la surface \brep) et ses caractéristiques géométriques (arêtes vives, coins, \ldots) ($\Rightarrow$ contraintes sur les n\oe uds et arêtes du maillage dans ces régions) ($\to$ persistance des entités \brep?)
\end{enumerate}

Déformation = mouvement induit par la propagation + lissage/optimisation

\subsection{Enjeux}
\begin{itemize}
	\item ``localiser'' les entités du maillage sur le modèle \brep\ (face \brep\ et coordonnées $uv$) 
	\item associer des n\oe uds/arêtes du maillage aux sommets/arêtes vifs du modèle \brep
%	\item persistance des entités \brep\  au cours de la déformation
\end{itemize}

\section{Maillage conforme aux carreaux}
Maillage conforme aux frontières des faces \brep\ \cite{andrieu2017}
\begin{itemize}
	\item un maillage par face \brep
	\item maillage unique des arêtes \brep\ pour garantir la conformité globale
	\item[+] lissage/optimisation dans l'espace $uv$ 2d, les caractéristiques géométriques sont naturellement représentées par le maillage
	\item[-] contraintes excessives sur le maillage (détailler \ldots), persistance des entités \brep\  pas flexible
\end{itemize}
%	\item persistance des entités \brep\  au cours de la déformation
%	\item respect des caractéristiques géométriques (sommets et arêtes vifs) $\to$ contraintes sur les n\oe uds et arêtes du maillage
%\end{itemize}

\section{Maillage trans-carreaux}
Hypergraphe \cite{foucault2008}\par
Projection d'un déplacement $xyz$ sur une surface paramétrique\par
Projection d'un déplacement $xyz$ sur une surface \brep\ composite (traversée des arêtes \brep\ douces, ``trajectoire trans-carreaux'' \cite[Section~5.5]{foucault2008}), \cite{thompson2005}, \cite[p.42 et Section~4.4.1]{crozet2017}\par
Régénération des chemins contraints\par
Pré-déformation\par
Optimisation, pondération des triangles

\par\bigskip
n\oe uds du maillage contraints sur :
\begin{itemize}
	\item un sommet \brep\ : fixe;
	\item une arête \brep\ : déplacement $xyz$ projeté sur la direction tangente locale de la courbe qui supporte l'arête
	\item une face \brep\ : déplacement $xyz$ projeté sur le plan tangent local de la surface qui supporte la face. 
	Si le n\oe ud est \textit{proche} d'un bord de la face, tester si son déplacement traverse le bord
	\begin{itemize}
		\item \ldots
	\end{itemize}
\end{itemize}
