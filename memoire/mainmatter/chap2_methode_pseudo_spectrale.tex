\chapter[Méthode d'ordre élevé pour le suivi d'un carreau de surface]{Méthode d'ordre élevé pour le suivi d'un seul carreau de surface}
\label{chap:methode_ps}

\section{Discrétisation spectrale en espace}
\subsection{État de l'art}
\subsection{Polynômes de Chebyshev}
\subsection{Représentation de surfaces}

\section{Intégration temporelle}
\subsection{Advection dans un champ de vecteurs vitesse donné}
Intégration explicite de la vitesse aux marqueurs lagrangiens (typiquement Runge-Kutta à l'ordre 4)

\subsection{Approximation de l'enveloppe des sphères}% partielle}
Entrée : vecteur position $\bx_{i,j}$ et vitesse normale $\nu_{i,j}$ de chaque marqueur lagrangien , pas de temps $\Delta t$
\begin{enumerate}
	\item transformation directe (de l'espace physique vers l'espace spectral) pour construire les polynômes d'interpolation du vecteur position et de la vitesse normale
	\item construction des polynômes dérivés 
	\item transformation inverse pour évaluer les dérivées aux n\oe uds CGL $(u_i,v_j)$
	\item calcul de la normale 
	\begin{equation}
		\unv = \frac{1}{\sqrt{\determinant{\fff}}} \crossprod{\bsu}{\bsv}
	\end{equation}
	\item calcul de la composante tangentielle du déplacement vers l'EdS
	\begin{equation}
		\vrm{w} = \frac{1}{\determinant{\fff}} 
		\left(
			\left( \nu_v I_{2,1} - \nu_u I_{2,2} \right)\bsu + 
			\left( \nu_u I_{2,1} - \nu_v I_{1,1} \right)\bsv
		\right)
	\end{equation}
	\item on pose $\tau = \min\left\{\Delta t, \displaystyle\frac{\lambda}{\displaystyle\max_{i,j} \normtwo{\vrm{w}_{i,j}^{(k)} }} \right\}$ ($\lambda \leq 1$) et on avance dans le temps d'un pas $\tau$
	\begin{equation}
		\bx_{i,j}^{(k+1)} = \bx_{i,j}^{(k)} + \tau \nu_{i,j}^{(k)} 
		\left( 
			%\tau \vrm{w}_{i,j} + \sqrt{1 - \tau^2 \vrm{w}_{i,j}^2} \unv_{i,j}^{(k)}
			\tau \vrm{w}_{i,j}^{(k)} + \sqrt{1 - \tau^2 \normtwo{\vrm{w}_{i,j}^{(k)}}^2} \unv_{i,j}^{(k)}
		\right)
	\end{equation}
\end{enumerate}

\section{Amélioration de la stabilité}
\subsection{Réduction de l'erreur d'aliasing}
méthode proposée par \cite{rahimian2015} difficile à appliquer dans notre cas car 
\begin{enumerate}
	\item les carreaux de surface ont un bord, 
	\item l'espacement non-uniforme des marqueurs lagrangiens (images des n\oe uds CGL) impose une forte contrainte CFL sur leurs déplacements
	
\end{enumerate}

\subsection{Prévention des singularités géométriques}

%singularité $\Leftrightarrow \determinant{\fff} = 0$ ($\Rightarrow$ normale indéfinie)
%Définition singularité
%\begin{itemize}
%	\item paramétrique : $\determinant{\fff}(u,v) = 0 \Leftrightarrow \rank{\jacobian{\bs}(u,v)} < 2$
%	\item géométrique : 
%\end{itemize}
2 types de singularités (\cite[p.320]{patrikalakis2009}) :
\begin{itemize}
	\item points irréguliers (plan tangent non défini)
	\item auto-intersections (non-injectivité de la paramétrisation) : ne pose pas de problème de stabilité numérique mais viole la définition de variété
\end{itemize}

\begin{itemize}
	\item \cite{jiao2001} (en 2D, \ie l'interface est une courbe) : 
	\item \cite{farouki1986} donne les conditions pour qu'une interface (représentée par une mosaïque de carreaux paramétriques) propagée à vitesse normale uniforme devienne localement singulière
\end{itemize}

pistes de résolution
\begin{enumerate}
	\item approximation non dégénérée \cite{farouki1986}
	\item tracé des courbes iso-courbure critique 
\end{enumerate}


