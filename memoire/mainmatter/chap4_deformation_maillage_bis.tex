\chapter{Déformation de maillage surfacique}

\textit{Objectif du chapitre: on veut mettre au point une méthodologie pour déformer un maillage de l'interface en propagation en utilisant le modèle \brep\ dynamique comme support géométrique, afin de pouvoir réaliser des simulations EF/VF dans des domaines de géométrie déformables.}

\par\bigskip

Motivation
\begin{enumerate}
	\item la précision et la vitesse de convergence du calcul dépendent fortement de la qualité (forme et taille) des éléments du maillage
\end{enumerate}


\section{Simulation numérique avec géométrie déformable}% [état de l'art]}
\begin{enumerate}
	\item maillage volumique (fluide) conforme à l'interface
	\begin{enumerate}
		\item \label{item:methodo_bodyfitted_ALE} 1 seul maillage \anglais{body-fitted} avec formulation ALE (ref)
		\begin{itemize}
			\item principe : frontière = maillage de l'interface, intérieur déformé de façon arbitraire
			\item intérêt/avantages : \ldots
			\item contraintes/inconvénients : 
			\begin{itemize}
				\item la qualité du maillage volumique dépend fortement de celle du maillage surfacique, surtout dans les régions proches de l'interface, où ont généralement lieu les phénomènes physiques les plus pertinents
				\item la connectivité du maillage doit rester fixe
			\end{itemize}
		\end{itemize}
		
		\item plusieurs maillages \anglais{body-fitted} qui se superposent
		\begin{itemize}
			\item méthode Chimère \cite{meakin1989, wang200}, \item FLUSEPA \cite{brenner1991}
			\item intérêt/avantages : 
			\begin{itemize}
				\item facilite la génération du maillage volumique lorsque la géométrie est complexe (\eg hyper-sustentateurs)
				\item évite de déformer un maillage 3d
			\end{itemize}						
			\item contraintes/inconvénients : 
			\begin{itemize}
				\item nécessite de traiter les intersections entre les blocs de maillage
				\item limité aux mouvements rigides (?)
			\end{itemize}
		\end{itemize}
	\end{enumerate}
	
	\item maillage volumique non conforme à l’interface
	\begin{itemize}
		\item méthode des frontières immergées \cite{peskin2002, hovnanian2012, wang2012} : interface représentée explicitement, volume (fluide) traité de façon eulérienne (\ie maillage fixe)
		\item intérêt/avantages : évite de générer et déformer un maillage 3d autour d’une géométrie complexe
		\item contraintes/inconvénients : application indirecte des conditions aux limites
	\end{itemize}
\end{enumerate}

Dans cette thèse,
\begin{enumerate}
	\item on se concentre sur les applications utilisant la méthodologie décrite à l'\autoref{item:methodo_bodyfitted_ALE}
	\item on ne traite que le maillage (surfacique) de l'interface
	\item (on se limite pour le moment à des maillages simpliciaux, mais une future extension aux maillages hybrides et courbes envisageable)
\end{enumerate}


\section{Optimisation de maillage surfacique}% [état de l'art]}
\begin{enumerate}
	\item changements locaux de connectivité
	\begin{enumerate}
		\item bascule d'arête
		\item contraction d'arête
	\end{enumerate}
	\item bouger de n\oe ud (direct, \ie $xyz$ ou indirect, \ie $uv$)
	\begin{enumerate}
		\item méthodes heuristiques (lissage laplacien, analogies physiques \cite{farhat1998}, interpolation (IDW, RBF, \ldots) \ldots)
		\item lissage basé sur l'optimisation d'une métrique de qualité \cite{freitag1995, canann1998, jiao2008, gargallo2014}
	\end{enumerate}
\end{enumerate}

Dans cette thèse,
\begin{enumerate}
	\item \autoref{item:methodo_bodyfitted_ALE} $\Rightarrow$ on se concentre essentiellement sur le bouger de n\oe ud (déformation pure)
	\item on effectuera des reconnections locales qu'en dernier recours (remaillage)
\end{enumerate}

\section{Génération de maillage surfacique basé sur un modèle \brep}
