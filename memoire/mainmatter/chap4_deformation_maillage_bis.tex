\chapter{Déformation de maillage surfacique}

\textit{Objectif du chapitre: on veut mettre au point une méthodologie pour déformer un maillage de l'interface en propagation en utilisant le modèle \brep\ dynamique comme support géométrique, afin de pouvoir réaliser des simulations EF/VF dans des domaines de géométrie déformables.}

\par\bigskip

Motivation
\begin{enumerate}
	\item la précision et la vitesse de convergence du calcul dépendent fortement de la qualité (forme et taille) des éléments du maillage
\end{enumerate}

\section{Introduction/état de l'art/problématique}

\subsection{Simulation numérique avec une géométrie déformable}
\begin{enumerate}
	\item maillage volumique (fluide) conforme à l'interface
	\begin{enumerate}
		\item \label{item:methodo_bodyfitted_ALE} 1 seul maillage \anglais{body-fitted} avec formulation ALE \emph{(ref)}
		\begin{itemize}
			\item principe : frontière = maillage de l'interface, intérieur déformé de façon arbitraire
			\item intérêt/avantages : \ldots
			\item contraintes/inconvénients : 
			\begin{itemize}
				\item la qualité du maillage volumique dépend fortement de celle du maillage surfacique, surtout dans les régions proches de l'interface, où ont généralement lieu les phénomènes physiques les plus pertinents
				\item la connectivité du maillage doit rester fixe \emph{(à vérifier)}
			\end{itemize}
		\end{itemize}
		
		\item plusieurs maillages \anglais{body-fitted} qui se superposent
		\begin{itemize}
			\item méthode Chimère \cite{meakin1989, wang2000}, FLUSEPA \cite{brenner1991}
			\item intérêt/avantages : 
			\begin{itemize}
				\item facilite la génération du maillage volumique lorsque la géométrie est complexe (\eg hyper-sustentateurs)
				\item évite de déformer un maillage 3d
			\end{itemize}						
			\item contraintes/inconvénients : 
			\begin{itemize}
				\item nécessite de traiter les intersections entre les blocs de maillage
				\item limité aux mouvements rigides \emph{(à vérifier)}
			\end{itemize}
		\end{itemize}
	\end{enumerate}
	
	\item maillage volumique non conforme à l’interface
	\begin{itemize}
		\item méthode des frontières immergées \cite{peskin2002, hovnanian2012, wang2012} : interface représentée explicitement, volume (fluide) traité de façon eulérienne (\ie maillage fixe)
		\item intérêt/avantages : évite de générer et déformer un maillage 3d autour d’une géométrie complexe
		\item contraintes/inconvénients : application indirecte des conditions aux limites
	\end{itemize}
\end{enumerate}

Dans cette thèse,
\begin{enumerate}
	\item on se concentre sur les applications utilisant la méthodologie décrite à l'\autoref{item:methodo_bodyfitted_ALE}
	\item on ne traite que le maillage (surfacique) de l'interface
	\item (on se limite pour le moment à des maillages simpliciaux, mais une future extension aux maillages hybrides et courbes envisageable)
\end{enumerate}


\subsection{Optimisation de maillage surfacique}
\begin{enumerate}
	\item changements locaux de connectivité
	\begin{enumerate}
		\item bascule d'arête
		\item contraction d'arête
	\end{enumerate}
	\item bouger de n\oe ud (direct, \ie $xyz$ ou indirect, \ie $uv$)
	\begin{enumerate}
		\item méthodes heuristiques (lissage laplacien, analogies physiques \cite{farhat1998}, interpolation (IDW, RBF, \ldots) \ldots)
		\item lissage basé sur l'optimisation d'une métrique de qualité \cite{freitag1995, canann1998, jiao2008, gargallo2014}
	\end{enumerate}
\end{enumerate}

Dans cette thèse,
\begin{enumerate}
	\item \autoref{item:methodo_bodyfitted_ALE} $\Rightarrow$ on se concentre essentiellement sur le bouger de n\oe ud (déformation pure)
	\item on effectuera des reconnections locales qu'en dernier recours (remaillage)
\end{enumerate}

\subsection{Génération de maillage surfacique basé sur un modèle \brep}
Essentiellement extension de méthodes standard (\ie quadtree, Delaunay, avancée de front) 2d plan à des surfaces immergées/plongées dans $\reals^3$
\begin{enumerate}
	\item méthodes indirectes (Riemanniennes) : on travaille dans l'espace paramétrique en tenant compte de la métrique (anisotrope, Riemannienne) induite par la paramétrisation de façon à ce que le plongement du maillage dans $\reals^3$ respecte les critères prescrits
	\begin{enumerate}
		\item conforme à la topologie \brep\ : on exploite directement les paramétrisations locales (carreaux de surface) du modèle \brep\ \cite{borouchaki2000} (on maille d'abord les sommets, puis les arêtes et enfin les faces afin de garantir la conformité du maillage)
		\begin{itemize}
			\item intérêt/avantages : utilisation de méthodes 2d plan robustes et efficaces
			\item contraintes/inconvénients : les arêtes \brep\ douces introduisent des contraintes supplémentaires sur le maillage, sans avoir de signification du point du vue du calcul EF/VF $\Rightarrow$ éléments de mauvaise qualité
		\end{itemize}
		\item trans-carreaux par (re-)paramétrisation globale : 
			\begin{enumerate}
				\item \cite{marcum1999} : %on part d'un maillage de référence conforme à la topologie \brep\ (\anglais{Physical Space Approximation} ou PSA). on plonge ce maillage de référence dans $\reals^2$ de façon à définir une paramétrisation globale linéaire par morceaux. on génère ensuite un maillage en ui
				\item \cite{noel2002} : 
				\item \cite{jones2004} : %on part d'un maillage de référence conforme à la topologie \brep. On choisit une face \brep\ de \textit{base}, dont l'espace paramétrique local sert de base pour construire un espace paramétrique global. Les faces \brep\ adjacentes sont ensuite plongées dans l'espace paramétrique global . Cette approche ne fonctionne que lorsque les domaines paramétriques des faces adjacentes ont des dimensions similaires
			\end{enumerate}
		\begin{itemize}
			\item intérêt/avantages : lève les contraintes topologiques du modèle \brep\ qui ne sont pas pertinentes pour le calcul EF/VF
			\item contraintes/inconvénients : 
			\begin{itemize}
				\item topologie : limité aux variétés ouvertes (avec un ou plusieurs bords)
				\item géométrie : limité aux surfaces quasi-planes et régulières, on perd le lien avec la géométrie de la surface \brep
				\item (méthodes pas suffisamment automatisées)
			\end{itemize}
			
		\end{itemize}
	\end{enumerate}
	\item 
\end{enumerate}