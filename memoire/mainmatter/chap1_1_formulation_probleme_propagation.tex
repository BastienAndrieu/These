%\chapter[Formulation du problème de propagation d'interfaces $\contgeom{1}$ \piecewise\ en \troisD]{Formulation mathématique du problème de propagation d'interfaces régulières \piecewise\ en trois dimensions}
\chapter[Formulation du problème de propagation d'interfaces $\contgeom{1}$ \piecewise]{Formulation mathématique du problème de propagation d'interfaces régulières \piecewise\ en trois dimensions}
\chaptermark{Formulation du problème de propagation d'interfaces $\contgeom{1}$ \piecewise}%\ en \troisD}
\label{chap:formulation_probleme_propagation}

Ce premier chapitre a pour objectif de formaliser le problème central de la thèse : la propagation d'interfaces régulières \piecewise\ en trois dimensions. 
Pour cela, on rappelle dans un premier temps les éléments de la géométrie différentielle des courbes et surfaces auxquels on aura recours tout au long du manuscrit.
On accorde un intérêt particulier aux surfaces de régularité $\contgeom{1}$ \piecewise, ainsi qu'à leur description par décomposition cellulaire via la représentation par les frontières.
On donne ensuite la formulation lagrangienne traditionnelle du problème de propagation d'interface et met en évidence ses limites dans le contexte de la thèse.
On conclut ce chapitre en introduisant une formulation plus adéquate du problème, basée sur le principe de Huygens.


\section{Géométrie différentielle des (courbes et) surfaces}
(annexe?) introduire notions : 
\begin{enumerate}
	\item plan tangent, continuité géométrique $\contgeom{1}$
	\item \textit{carreau paramétrique} : $(\uvdomain, \bs)$
	\begin{itemize}
		\item $\uvdomain \subset \reals^2$ est le \textit{domaine paramétrique}
		\item $\Sigma = \bs(\uvdomain)$ est le \textit{support} du carreau (surface)
		\item $\bs : \uvdomain \to \Sigma \subset \reals^3$ est une \textit{paramétrisation} de $\Sigma$
		\item vecteurs tangents $\bsu$ et $\bsv$
		\item vecteur normal $\unv = \unitized{\crossprod{\bsu}{\bsv}}$
		\item tenseur métrique, première forme fondamentale $\fff$, structure Riemannienne
		\item seconde forme fondamentale $\sff$, courbures normale, principales, moyenne et gaussienne
	\end{itemize}
	\item \textit{arc/courbe paramétrique} : $(\wdomain, \bg)$
	\begin{itemize}
		\item $\wdomain = \left[\lo{ł}, \hi{w}\right]$ est l'\textit{intervalle paramétrique}
		\item $\Gamma = \bg(\wdomain)$ est le \textit{support} de l'arc (courbe)
		\item $\bg : \wdomain \to \Gamma \subset \reals^3$ est une \textit{paramétrisation} de $\Gamma$
		\item vecteur tangent $\bg'$
		\item courbure
		\item courbe sur un carreau de surface
		\begin{itemize}
			\item géométrie différentielle de l'intersection de deux carreaux
		\end{itemize}
	\end{itemize}
\end{enumerate}

\input{figures/fig_diffgeom.tex}


\section{Description et représentation des surfaces régulières \piecewise}
%[décomposition en nappes régulières ($\contgeom{1}$), courbes singulières (crêtes?) et points irréguliers (coins?) (\cf notes Huygens)]
%\par\bigskip
%(Inspiré par \cite[p.65]{rossignac1985} et \cite[Section 2.6]{rossignac1986}.)\par

%[courbes et surfaces paramétriques : vocabulaire (arc, carreau, espace et domaine paramétriques, carreau restreint, courbes de restriction), vecteur(s) et plan tangents, vecteur (pseudo-)normal, notion de continuités paramétrique/géométrique, tenseur métrique, première et seconde formes fondamentales, courbures principales, gaussienne et moyenne]
%\par\bigskip
%Dans ce manuscrit, on appelle \textit{carreau de surface paramétrique} de classe $\contdiff{k}$ tout couple $(\uvdomain, \bs)$ où $\uvdomain \subset \reals^2$, $\bs : \uvdomain \to \reals^3$ est une fonction de classe $\contdiff{k}$ sur $\uvdomain$, et $k \in \integers$. 
%On appelle $\uvdomain$ le \textit{domaine paramétrique} d'un tel carreau et on dit que $\bs$ en est une \textit{paramétrisation}. 
%Par commodité, on désignera parfois le carreau simplement par $\bs$.
%\par
%Le support $\bs(\uvdomain)$ 
introduire notions \cite{rossignac1985} : 
\begin{enumerate}
	\item \textit{nappe régulière} : variété de dimension 2, connexe, de continuité $\contgeom{1}$ (direction normale continue)
	\item \textit{courbe singulière} : variété de dimension 1, connexe, de continuité $\contgeom{1}$ (direction tangente continue)
	\item \textit{point singulier} : point qui n'est à l'intérieur d'aucune nappe régulière ou courbe singulière
\end{enumerate}

\begin{itemize}
	\item la frontière d'une nappe régulière est composée de courbes et de points singuliers (si elle n'est pas vide)
	\item la frontière d'une courbe singulière (ses extrémités) est composée de points singuliers (si elle n'est pas vide)
	\item tout point d'une surface $\contgeom{1}$ \piecewise\ est
	\begin{itemize}
		\item soit situé à l'intérieur d'une nappe régulière ;
		\item soit situé à l'intérieur d'une courbe singulière ;
		\item soit un point singulier.
	\end{itemize}
\end{itemize}

\begin{figure}
	\centering
	\setlength{\imagewidth}{80mm}%
\setlength{\imageheight}{\imagewidth}%
\DTLsetseparator{,}%
\DTLloaddb[noheader,keys={x,y,a}]{dbcorners}{figures/data/piecewise_smooth_surface/corners_xya.dat}%
\begin{tikzpicture}[%
	x=\imagewidth, y=\imageheight,
	img/.style={anchor=south west, inner sep=0}]
	%%% SURFACE
	\node[img] at (0,0) {\includegraphics[width=\imagewidth]{piecewise_smooth_surface/surface}};
	%%% HIDDEN EDGES
	\node[img] at (0,0) {\includegraphics[width=\imagewidth]{piecewise_smooth_surface/edges_hidden}};
	%%% HIDDEN CORNERS
	\DTLforeach*{dbcorners}{\locx=x, \locy=y, \loca=a}{%
		\ifnum \loca = 0
			\fill[black] (\locx,\locy) circle (1.0pt);
		\fi
	}%
	%%% SURFACE (semi-transparent to mask hidden edges & corners)
	{\transparent{0.75}
		\node[img] at (0,0) {\includegraphics[width=\imagewidth]{piecewise_smooth_surface/surface}};
	}%
	%%% VISIBLE EDGES
	\node[img] at (0,0) {\includegraphics[width=\imagewidth]{piecewise_smooth_surface/edges_visible}};
	%%% VISIBLE CORNERS
	\DTLforeach*{dbcorners}{\locx=x, \locy=y, \loca=a}{%
		\ifnum \loca = 1
			\fill[black] (\locx,\locy) circle (1.0pt);
		\fi
	}%
	%
	%%% ANOTATIONS
	\DTLassign{dbcorners}{1}{\locxa=x,\locya=y,\loca=a}% 	
	\DTLassign{dbcorners}{4}{\locxb=x,\locyb=y,\loca=a}% 
	\DTLassign{dbcorners}{8}{\locxc=x,\locyc=y,\loca=a}% 
	\node[anchor=west] (anotIrrPts) at (\locxb,{0.5*(\locya + \locyb)}) {points irréguliers};
	\draw[shorten >=5pt, ->] (anotIrrPts.south) to [bend left=20] (\locxa, \locya);
	\draw[shorten >=5pt, ->] (anotIrrPts.north) to [bend right=20] (\locxb, \locyb);
	\draw[shorten >=5pt, ->] (anotIrrPts.west) to [bend left=10] (\locxc, \locyc);
\end{tikzpicture}
\DTLgdeletedb{dbcorners}%

	\caption{Surface régulière \piecewise\ dont les courbes et points singuliers sont mis en évidence.}
	\label{fig:piecewise_smooth_surface_decomposition}
\end{figure}

%La frontière $\boundary{\Omega}$ du solide $\Omega$ est une surface de continuité $\contgeom{1}$ (\ie possède une direction normale continue) \piecewise. 
%Les singularités géométriques de $\boundary{\Omega}$ peuvent être de dimension 1 (courbes) ou 0 (points).
%Les courbes singulières de $\boundary{\Omega}$ sont elle-mêmes des variétés différentielles de continuité $\contgeom{1}$ (\ie possède une direction tangente continue) excepté aux points singuliers.

\subsection{Représentation par les frontières (\brep)}
%Origine, utilisation, concept, définitions formelles des entités, lien avec la décomposition d'une surface $\contgeom{1}$ \piecewise, (géométrie différentielle des courbes et surfaces paramétriques), structures de données (DCEL)
La \textit{représentation par les frontières} (ou \brep\ pour \anglais{Boundary Representation}) est un formalisme très répandu dans le domaine de la Conception Assistée par Ordinateur (CAO) \textit{(développer \ldots)}.
Elle consiste à décrire un solide $\Omega$ à l'aide d'une décomposition cellulaire de la surface $\Sigma$ qui matérialise sa frontière. 
Cette décomposition --- illustrée sur la \autoref{fig:BRep} --- est constituée de \textit{faces}, d'\textit{arêtes} et de \textit{sommets} dont on donne une définition dans les paragraphes suivants. 

\begin{figure}
	\centering
	\setlength{\imagewidth}{80mm}%
\setlength{\imageheight}{\imagewidth}%
\DTLsetseparator{,}%
\DTLloaddb[noheader,keys={x,y,a}]{dbverts}{figures/data/BRep/verts_xya.dat}%
\begin{tikzpicture}[%
	x=\imagewidth, y=\imageheight,
	img/.style={anchor=south west, inner sep=0}]
	%%%%%%%%%%%%%%%% SHELL %%%%%%%%%%%%%%%%
	%%% FACES
	\node[img] at (0,0) {\includegraphics[width=\imagewidth]{BRep/shell}};
	%%% HIDDEN EDGES
	\node[img] at (0,0) {\includegraphics[width=\imagewidth]{BRep/edges_hidden}};
	%%% HIDDEN VERTICES
	\DTLforeach*{dbverts}{\locx=x, \locy=y, \loca=a}{%
		\ifnum \loca = 0
			\fill[black] (\locx,\locy) circle (1.0pt);
		\fi
	}%
	%%% FACES (semi-transparent to mask hidden edges & verts)
	{\transparent{0.75}
		\node[img] at (0,0) {\includegraphics[width=\imagewidth]{BRep/shell}};
	}%
	%%% VISIBLE EDGES
	\node[img] at (0,0) {\includegraphics[width=\imagewidth]{BRep/edges_visible}};
	%%% VISIBLE VERTICES
	\DTLforeach*{dbverts}{\locx=x, \locy=y, \loca=a}{%
		\ifnum \loca = 1
			\fill[black] (\locx,\locy) circle (1.0pt);
		\fi
	}%
\end{tikzpicture}
\DTLgdeletedb{dbverts}%
%%%%%%%%%%%%%%%%%%%%%%%%%%%%%%%%%%%%%%
%
%%%%%%%%%%%%%%%% FACES %%%%%%%%%%%%%%%%
\def\imfacew{44mm}
\def\ngriduv{6}
\def\vertsep{0.05}
\def\edglabsepuv{0.17}
\def\wirlabsepuv{0.18}
\def\edglabsepxyz{0.06}
\def\iniwclr{0.3}
\def\decwclr{0.3}%{0.2}
\def\uvscale{0.34}
\def\uvyshift{-0.7}
\pgfmathsetmacro\sepyshift{0.5 * (\uvyshift+\uvscale)}%
%
\begin{tikzpicture}[%
	x = \imfacew, y = \imfacew,
	gridtick/.style={red, fill=white, font=\tiny, inner sep=0.5pt},
	img/.style={anchor=south west, inner sep=0},
	label/.style={inner sep=1pt, font=\scriptsize},
	uvgrid/.style={black!10!white},
	curv/.style={line width=0.8pt, line cap=round},
	spacelabel/.style={anchor=north, rotate=90, inner sep=0, font=\bfseries},
	]
	\foreach \jfa/\ifa in {-1/007, 0/008, 1/002}{%
		\figbrepface{\ifa}{{1.05*\jfa - 0.5}}{{-\sepyshift}}%\hfill
	}%
	\draw[very thick, gray, dashed] 
	({-0.5*\textwidth},0) -- ++ 
	(\textwidth,0);
	\node[spacelabel] (xyzspace) at 
	({-0.5*\textwidth},{-\sepyshift+0.5}) {Espace euclidien\vphantom{Espace paramétrique}};
	\node[spacelabel] (uvspace) at 
	({-0.5*\textwidth},{\sepyshift-\uvscale}) {Espace paramétrique\vphantom{Espace euclidien}};
\end{tikzpicture}

	\caption{Modèle \brep\ d'une surface régulière par morceaux.}
	\label{fig:BRep}
\end{figure}

\subsubsection{Faces}
Une face du modèle \brep\ est une variété de dimension 2 connexe délimitée par des arêtes et des sommets. 
Géométriquement, une face $\brepface$ est décrite par un carreau paramétrique restreint.
La topologie des courbes de restriction du domaine paramétrique est décrite à l'aide de \textit{contours} (un par composante connexe du bord de $\brepface$). 
Une face possède ainsi un contour extérieur $\brepwire^{\mathrm{ext}}$ et, éventuellement un ou plusieurs contours intérieurs $\brepwire^{\mathrm{int},i}$ si celle-ci comporte des \guillemets{trous}.
Les faces, qui sont quasi-disjointes deux-à-deux (\ie ne s'intersectent qu'en des arêtes ou des sommets du modèle \brep), sont regroupées en \textit{coquilles}, qui représentent chacune une composante connexe de $\Sigma$. 


\subsubsection{Arêtes}
Une arête du modèle \brep\ est une variété de dimension 1 connexe délimitée par des sommets.
Puisque $\Sigma$ est une variété sans bord, chaque arête du modèle \brep\ est incidente à exactement deux faces $\brepface_1$ et $\brepface_2$. 
Géométriquement, l'arête $\brepedge$ est représentée par une branche de la courbe d'intersection entre les carreaux $\Sigma_1$ et $\Sigma_2$ qui décrivent respectivement $\brepface_1$ et $\brepface_2$. 
Cette courbe peut également être représentée par sa trace dans l'espace paramétrique de chaque carreau. 
On peut donc la représenter à l'aide des trois courbes paramétriques $\bg$, $\bp_1$ et $\bp_2$ telles que
\begin{equation}
	\bg = \bs_1 \circ \bp_1 = \bs_2 \circ \bp_2.
\end{equation}

\begin{figure}
	\centering
	\newlength{\locimw}
\setlength{\locimw}{67.5mm}
\newlength{\locimh}
\setlength{\locimh}{\locimw * \real{0.75}}
%
\def\uvscale{0.28}
\def\fracaxeoffset{0.0}
\def\distanceaxe{0.1}
\def\psisep{-0.42}
%
\DTLsetseparator{,}%
\DTLloaddb[noheader,keys={x,y}]{dbpoint}{figures/data/fig_simple_intersection_point.dat}%
\DTLassign{dbpoint}{1}{\xloc=x, \yloc=y}% 
%
\begin{tikzpicture}[
	x=\locimw, y=\locimh, 
	axe/.style={-stealth, line width=0.5pt},
	uvdomain/.style={thin}, 
	image/.style={anchor=south west, inner sep=0},
	curve/.style={thick, line cap=round},
	label/.style={font=\normalsize},
	axelabel/.style={font=\small},
	axeuvlabel/.style={axelabel, inner sep=0},
	point/.style={fill=black, circle, scale=0.3},
	map/.style={-{Classical TikZ Rightarrow[length=4pt,width=4pt]}}]
	%
	\node[image] (img) at (0,0) {\includegraphics[width=\locimw]{figures/fig_simple_intersection}};
	\node[image] (img) at (0,0) {\includegraphics[width=\locimw]{figures/fig_simple_intersection_border_hid}};
	{\transparent{0.75}%
		\node[image] (img) at (0,0) {\includegraphics[width=\locimw]{figures/fig_simple_intersection}};
	}%
	\node[image] (img) at (0,0) {\includegraphics[width=\locimw]{figures/fig_simple_intersection_border_vis}};
	\draw[curve] plot file {figures/data/fig_simple_intersection_curve.dat};
	\node[point] (xyz) at (\xloc, \yloc) {};
	\node[label, anchor=east] at (\xloc, \yloc) {$\bg(w)$};
	%
	\node[label] at (0.56, 0.91) {$\Sigma_1$};
	\node[label] at (0.93, 0.75) {$\Sigma_2$};
	%
%	\foreach \igrid in  {0,0.1,...,1.01}{
%		\draw[red, thin] (0,\igrid) -- (1,\igrid)
%		                 (\igrid,0) -- (\igrid,1);
%	}
	% 
	% trièdre
	\def\scaletriedre{0.8}
	\coordinate (o) at (0.13209545612335205 , 0.14773482084274292);
	\coordinate (x) at (0.23464959859848022 , 0.02938912808895111);
	\coordinate (y) at (0.26169726252555847 , 0.2535654902458191);
	\coordinate (z) at (0.1061694324016571 , 0.31903284788131714);
	\draw[axe] (o) -- ($(o)!\scaletriedre!(x)$) node[axelabel, anchor=west] {$x$};
	\draw[axe] (o) -- ($(o)!\scaletriedre!(y)$) node[axelabel, anchor=west] {$y$};
	\draw[axe] (o) -- ($(o)!\scaletriedre!(z)$) node[axelabel, anchor=south] {$z$};
	%
	\begin{scope}[scale=\uvscale, x=\locimh, y=\locimh, shift={(img.west)}]
		\begin{scope}[shift={(-1.5,0.7)}]
			\DTLloaddb[noheader,keys={r,g,b}]{dbsurfacecolor}{figures/data/fig_brep_faces/facecolor_002.dat}%
			\DTLassign{dbsurfacecolor}{1}{\rfai=r,\gfai=g,\bfai=b}% 
			\definecolor{surfacecolor}{RGB}{\rfai,\gfai,\bfai}
			\draw[uvdomain, fill=surfacecolor] (-1,-1) -- (1,-1) -- (1,1) -- (-1,1) -- cycle;
			\DTLgdeletedb{dbsurfacecolor}
			%
			\draw[curve] plot file {/d/bandrieu/GitHub/FFTsurf/test/demo_intersection/simple/curve_uv1.dat};
			%
			\DTLassign{dbpoint}{2}{\uloc=x, \vloc=y}% 
			\DTLassign{dbpoint}{3}{\duloc=x, \dvloc=y}% 
			\node[point] (uv1) at (\uloc, \vloc) {};
			%\node[label] at ({\uloc + \psisep*\duloc}, {\vloc + \psisep*\dvloc}) {$\bp_1(w)$};
			\node[label, anchor=north east, inner sep=0] at (\uloc, \vloc) {$\bp_1(w)$};
			% Axes
			\coordinate (o) at ({-1-\distanceaxe},{-1-\distanceaxe});
			\draw[axe] (o) -- ++ ({\fracaxeoffset+\distanceaxe+0.5},0) node [axeuvlabel, anchor=north west] {$u_1$};
			\draw[axe] (o) -- ++ (0,{\fracaxeoffset+\distanceaxe+0.5}) node [axeuvlabel, anchor=south east] {$v_1$};
		\end{scope}
	\end{scope}
	%
	\begin{scope}[scale=\uvscale, x=\locimh, y=\locimh, shift={(img.east)}]
		\begin{scope}[shift={(1.6,-0.9)}]
			\DTLloaddb[noheader,keys={r,g,b}]{dbsurfacecolor}{figures/data/fig_brep_faces/facecolor_008.dat}%
			\DTLassign{dbsurfacecolor}{1}{\rfai=r,\gfai=g,\bfai=b}% 
			\definecolor{surfacecolor}{RGB}{\rfai,\gfai,\bfai}
			\draw[uvdomain, fill=surfacecolor] (-1,-1) -- (1,-1) -- (1,1) -- (-1,1) -- cycle;
			\DTLgdeletedb{dbsurfacecolor}
			%
			\draw[curve] plot file {/d/bandrieu/GitHub/FFTsurf/test/demo_intersection/simple/curve_uv2.dat};
			%
			\DTLassign{dbpoint}{4}{\uloc=x, \vloc=y}% 
			\DTLassign{dbpoint}{5}{\duloc=x, \dvloc=y}% 
			\node[point] (uv2) at (\uloc, \vloc) {};
			%\node[label] at ({\uloc + \psisep*\duloc}, {\vloc + \psisep*\dvloc}) {$\bp_2(w)$};
			\node[label, anchor=north east, inner sep=0] at (\uloc, \vloc) {$\bp_2(w)$};
			% Axes
			\coordinate (o) at ({-1-\distanceaxe},{-1-\distanceaxe});
			\draw[axe] (o) -- ++ ({\fracaxeoffset+\distanceaxe+0.5},0) node [axeuvlabel, anchor=north west] {$u_2$};
			\draw[axe] (o) -- ++ (0,{\fracaxeoffset+\distanceaxe+0.5}) node [axeuvlabel, anchor=south east] {$v_2$};
		\end{scope}
	\end{scope}
	%
	% mappings
	\draw [map, shorten <= 5mm, shorten >= 5mm] (uv1) to [bend left =40] node [label, anchor=south west] {$\bs_1$} (xyz);
	\draw [map, shorten <= 5mm, shorten >= 5mm] (uv2) to [bend right=40] node [label, anchor=south west] {$\bs_2$} (xyz);
	%
%	\draw[red, thin, dashed] (current bounding box.south west) rectangle (current bounding box.north east);
\end{tikzpicture}
\DTLgdeletedb{dbpoint}
	\caption{Description géométrique d'une arête \brep.}
\end{figure}

Les arêtes, qui sont quasi-disjointes deux-à-deux (\ie ne s'intersectent qu'en des sommets du modèle \brep), sont regroupées pour former les contours des faces. 
Puisque ces derniers sont orientés, chaque arête $\brepedge$ est formée de deux \textit{co-arêtes} jumelles $\brepedge^1$ et $\brepedge^2$, chacune associée à une face incidente à $\brepedge$. 
Chaque co-arête possède une orientation, conforme à celle du contour qui la contient.


\subsubsection{Sommets}
Un sommet du modèle \brep\ matérialise l'intersection de deux arêtes et a pour support géométrique un point de $\reals^3$.

\subsubsection{Structure de données}
\begin{enumerate}
	\item séparation entre topologie (structure) / géométrie (forme)
	\item connectivité entre les entités capturée à l'aide d'une structure de graphe (homéomorphe à un polyèdre)
	\item structure de données adaptée $\to$ \anglais{Doubly Connected Edge List} (DCEL) (\anglais{Halfedges}) \textit{(donner réfs.)}
\end{enumerate}

La \autoref{fig:BRep_hierarchy} présente un diagramme de la hiérarchie des entités qui constituent le modèle \brep.
\begin{figure}
	\centering
	\input{figures/code/BRep_hierarchy}
	\caption{Hiérarchie des éléments constituant un modèle \brep.}
	\label{fig:BRep_hierarchy}
\end{figure}


\subsubsection{Représentation de surfaces régulières par morceaux}
Le formalisme de la représentation par les frontières est particulièrement adapté pour décrire des surfaces régulières par morceaux. 
En effet, 
\begin{itemize}
	\item l'ensemble des points singuliers de $\Sigma$ est un sous-ensemble des sommets de son modèle \brep\ ;
	\item l'ensemble des courbes singulières de $\Sigma$ est un sous-ensemble des arêtes de son modèle \brep\ ;
	\item les nappes régulières de $\Sigma$ sont formées par la réunion des faces de son modèle \brep.
%	\item chaque arête \brep\ vive est contenue dans une courbe singulière
\end{itemize}











\section{Formulation lagrangienne du problème de propagation d'interface}
On considère une interface $\Sigma$ entre deux milieux distincts (\eg un solide et un fluide).
Dans cette thèse, on se concentre sur des problèmes en trois dimensions. 
$\Sigma$ représente donc une surface (\ie une variété de dimension 2) que l’on supposera orientable et fermée (\ie sans bord).
De cette manière, l'interface sépare un domaine \textit{intérieur} $\Omega$ --- que l'on supposera ouvert --- et un domaine \textit{extérieur} $\complement{\Omega} = \reals^3 \setminus \Omega$.
\par
%[EDP avec vecteur vitesse ou vitesse normale, problème de définition au niveau des courbes singulières et points irréguliers]
%\par\bigskip
Le problème que l'on cherche à résoudre consiste à déterminer l’évolution au cours du temps de l'interface $\Sigma$ étant données sa position actuelle ainsi que sa vitesse de propagation.
Dans la formulation lagrangienne traditionnelle, ce problème est exprimé sous la forme d'une équation aux dérivées partielles (EDP) pour le vecteur position $\bx$ d'un point de l'interface.
L'équation décrivant la propagation suivant un champ de vecteur vitesse $\vrm{u} : \Sigma \times \reals \to \reals^3$ est
\begin{equation}
	\frac{\partial \bx}{\partial t} = \vrm{u}(\bx,t).
	\label{eq:lagrange_vecteur_vitesse}
\end{equation}
On peut également considérer que chaque point de $\Sigma$ se déplace le long de la direction normale à l'interface suivant un champ de vitesse normale $\nu : \Sigma \times \reals \to \reals$. 
L'équation décrivant la propagation est alors
\begin{equation}
	\frac{\partial \bx}{\partial t} = \nu(\bx,t) \unv(\bx,t),
	\label{eq:lagrange_vitesse_normale}
\end{equation}
où $\unv(\bx,t)$ désigne la direction normale à $\Sigma$ en $\bx$ à l'instant $t$, pointant vers l'extérieur de $\Omega$.
\par\bigskip
En principe, les formulations \eqref{eq:lagrange_vecteur_vitesse} et \eqref{eq:lagrange_vitesse_normale} sont équivalentes puisque la composante tangentielle du vecteur vitesse n'affecte pas la forme de l'interface. 
\par\bigskip
La formulation lagrangienne ne permet d'obtenir une solution au problème de propagation d'interface que dans le cas où cette dernière est \textit{globalement} régulière et le reste tout au long de sa propagation. 
En effet, puisque la direction normale $\unv$ n'est pas définie au niveau des courbes et points singuliers de $\Sigma$, le déplacement de ces points est ambigu. 
On peut notamment distinguer deux cas \cite{jiao2007}.
\par
Premièrement, si l'interface subit un mouvement d'advection (comme le transport d'un fluide ou encore la déformation d'un solide sous l'effet de contraintes mécaniques) alors ses singularités géométriques sont préservées au cours de la propagation.
\par
En revanche, si l'interface se propage à la manière d'un front d'onde (comme la progression d'une flamme, d'un dépôt de matière ou encore l'ablation d'un solide) alors les singularités sont soit préservées soit régularisées, suivant la convexité locale de l'interface.
C'est essentiellement sur ce deuxième type de propagation que l'on se concentre dans cette thèse.
\par\bigskip
Puisque l'on s'intéresse ici à la propagation d'interfaces régulières seulement \piecewise, il est nécessaire d'obtenir une meilleure formulation du problème qui s'affranchisse des ambiguïtés ainsi mises en évidence. 
Plutôt que de poser le problème sous la forme d'une équation différentielle, cette nouvelle formulation se présente comme une construction géométrique.

\section{Principe de Huygens avec condition d'entropie}
\label{section:principe_huygens}
\def\p{\vit{p}}
\def\q{\vit{q}}
%
%[Histoire, formulation, notion d'enveloppe de sphères/boules, construction géométrique au lieu de EDP, équations définissant l'EdS propre à 1/2 paramètre(s), différence d'ordre 2 avec transport suivant la normale, définitions implicites (\cf notes Huygens)]
%\par\bigskip
Alors qu'il développait un modèle ondulatoire de la propagation de la lumière, Christiaan Huygens proposa le principe suivant : chaque point de l'espace atteint par une onde lumineuse se comporte comme la source d'une ondelette secondaire émise dans toutes les directions. 
Si le milieu de propagation est homogène et isotrope alors les ondelettes sont sphériques. 
Le front d'onde, qui se propage ainsi de proche en proche, est alors formé par l'\textit{enveloppe} de ces ondelettes, \ie la surface qui est tangente à chacune d'elles et dont chaque point est un point de tangence avec une ondelette.
\par
Le principe de Huygens permet de décrire de nombreux phénomènes analogues à la propagation d'une onde dans un milieu tels que la progression d'une flamme. 
\textit{(développer \ldots)}

\par\bigskip
\begin{enumerate}
	\item distinguer propagation d'une onde et d'une interface entre deux milieux matériels (l'une peut s'auto-intersecter, l'autre non) $\to$ condition d'entropie
	%\item formaliser la notion d'enveloppe des sphères $\EdS{\Sigma}{\rho}$ (surface tangente à toutes les sphères et en tout point tangente à une sphère) $\to$ système d'équations avec $\implicitsphere_{\rho} : (\bx,\p) \mapsto \normtwo{\bx - \p}^2 - \rho^2(\p)$
	%\item formaliser la notion d'enveloppe des boules $\EoB{\Sigma}{\rho} := \boundary{\left\{\Omega \cup \sphere[\Sigma][\rho]\right\}}$
	\item formaliser la notion d'enveloppe des sphères (EdS) $\EdS{H \subseteq \Sigma}{\rho}$ (surface tangente à toutes les sphères et en tout point tangente à une sphère)
	\item formaliser la notion d'enveloppe des boules $\EoB{\Sigma}{\rho} := \boundary{\left\{\Omega \cup \sphere[\Sigma][\rho]\right\}}$
	\item donner définition implicite de $\EoB{\Sigma}{\rho}$ :
	\[ \EoB{\Sigma}{\rho} = \phi^{-1}(0) \]
	avec 
	\[ \phi(\bx) = \min_{\p \in \Sigma} \normtwo{\bx - \p}^2 - \rho(\p)^2 \]
	\item après une propagation à vitesse normale $\nu$ constante pendant un intervalle de temps $\tau$ tel que $\nu \tau = \rho$, la nouvelle interface est $\EoB{\Sigma}{\rho}$
	\item à condition que $\rho$ soit suffisamment régulière (continuité à déterminer), $\EoB{\Sigma}{\rho} \subset \EdS{\Sigma}{\rho}$
\end{enumerate}


\begin{figure}
	\centering
	\includegraphics[width=10cm]{EdS_EdB.JPG}
	\caption{Schéma pour distinguer EdS et EdB (à revoir!).}
	\label{fig:EdS_EdB}
\end{figure}

%Dans la suite, étant donné un sous-ensemble (ouvert ou fermé) $E$ de $\reals^d$, on notera $\complement{E} := \reals^d \setminus E$ son \textit{complémentaire}, $\boundary{E}$ sa \textit{frontière}, $\interior{E}$ son \textit{intérieur} et $\closure{E} = E \cup \boundary{E}$ son \textit{adhérence}.
%\par\bigskip
%
%\def\p{\vit{p}}
%\def\q{\vit{q}}
%Soient $r> 0$ et $\p$ un point de $\reals^3$. 
%On note $\sphere[\p][r]$ la sphère de rayon $r$ centrée en $\p$
%\begin{equation}
%    \sphere[\p][r] := \left\{ 
%        \bx \in \reals^3 \mid \normtwo{\bx - \p} = r
%    \right\},
%\end{equation}
%et $\ball{\p}{r}$ la boule \textit{ouverte} de rayon $r$ centrée en $\p$
%\begin{equation}
%    \ball{\p}{r} := \left\{ 
%        \bx \in \reals^3 \mid \normtwo{\bx - \p} < r
%    \right\}.
%\end{equation}
%
%Soit $\rho : \Sigma \to \reals_{+*}$ une fonction $\alpha$-lipschitzienne sur $\Sigma$ et à valeurs strictement positives. 
%On note $\ball{\Sigma}{\rho}$ la réunion des boules centrées sur $\Sigma$ et de rayon $\rho$
%\begin{equation}
%    \ball{\Sigma}{\rho} := \bigcup_{\p \in \Sigma} \ball{\p}{\rho(\p)}
%    = \left\{
%        \bx \in \reals^3 \mid \exists \p \in \Sigma \mid \normtwo{\bx - \p} < \rho(\p)
%    \right\}.
%\end{equation}
%
%Son complémentaire est
%\begin{equation}
%    \complement{\ball{\Sigma}{\rho}} = \bigcap_{\p \in \Sigma} \complement{\ball{\p}{\rho(\p)}}
%    = \left\{
%        \bx \in \reals^3 \mid \forall \p \in \Sigma, \normtwo{\bx - \p} \geq \rho(\p)
%    \right\},
%\end{equation}
%
%et sa frontière est
%\begin{align}
%    \boundary{\ball{\Sigma}{\rho}} 
%    &= \boundary{ \left(\complement{\ball{\Sigma}{\rho}}\right) } ,\nonumber\\
%    &= \closure{\ball{\Sigma}{\rho}} \cap 
%        \closure{ \left(\complement{\ball{\Sigma}{\rho}}\right) } ,\nonumber\\
%%    &= \left\{
%%        \bx \in \reals^3 
%%        \mid \forall \p \in \Sigma, \normtwo{\bx - \p} \geq \rho(\p) 
%%        \text{ et }
%%        \exists \p \in \Sigma \mid \normtwo{\bx - \p} \leq \rho(\p)
%%    \right\} ,\nonumber\\
%     &= \left\{
%        \bx \in \reals^3 
%        \mid \forall \p \in \Sigma, \normtwo{\bx - \p} \geq \rho(\p) 
%        \text{ et }
%        \exists \p \in \Sigma \mid \normtwo{\bx - \p} = \rho(\p)
%    \right\}.
%\end{align}
%
%\par\bigskip
%
%On note $\dilation{\Omega}{\rho}$ la \guillemets{dilatation} de $\Omega$ par $\rho$
%\begin{equation}
%    \dilation{\Omega}{\rho} = \Omega \cup \ball{\Sigma}{\rho}.
%\end{equation}
%
%On note $\EoB{\Sigma}{\rho}$ sa frontière
%\begin{align}
%    \EoB{\Sigma}{\rho} 
%    &:= \boundary{ \!\left(\dilation{\Omega}{\rho}\right) }, \\
%    &=  \boundary{ \!\left(\complement{ \left(\dilation{\Omega}{\rho}\right) }\right) } , \nonumber\\
%    &=  \boundary{ \!\left(\complement{\Omega} \cap \complement{\ball{\Sigma}{\rho}}\right) },\nonumber\\
%    &= 
%    \left\{
%        \bx \in \interior{ \left(\complement{ \Omega }\right) }
%        \mid \forall \q \in \Sigma, \normtwo{\bx - \q} \geq \rho(\q) 
%        \text{ et }
%        \exists \p \in \Sigma \mid \normtwo{\bx - \p} = \rho(\p)
%    \right\}.
%\end{align}
%Il s'agit de l'\textit{enveloppe des boules}\footnote{plus exactement des demi-boules à l'extérieur de $\Omega$.} (EdB) centrées sur $\Sigma$ et de rayon $\rho$.

%
%\def\p{\vit{p}}
%\def\q{\vit{q}}
%Soient $r > 0$ et $\p$ un point de $\reals^3$.
%On note $\sphere[\p][r]$ la sphère de rayon $r$ centrée en $\p$
%\begin{equation}
%    \sphere[\p][r] := \left\{ 
%        \bx \in \reals^3 \mid \normtwo{\bx - \p} = r
%    \right\},
%\end{equation}
%et $\ball{\p}{r}$ la boule \textit{ouverte} de rayon $r$ centrée en $\p$
%\begin{equation}
%    \ball{\p}{r} := \left\{ 
%        \bx \in \reals^3 \mid \normtwo{\bx - \p} < r
%    \right\}.
%\end{equation}
%
%Soit $\rho : \Sigma \to \reals_{+*}$ une fonction $\alpha$-lipschitzienne sur $\Sigma$ et à valeurs strictement positives. 
%On définit
%\begin{align}
%  \implicitsphere_{\rho} \colon \interior{ \left(\complement{ \Omega }\right) } \times \Sigma & \to \reals \nonumber\\
%  (\bx, \p) &\mapsto \normtwo{\bx - \p}^2 - \rho^2(\p).
%\end{align}
%La sphère $\sphere[\p][\rho(\p)]$ est alors le lieu des points $\left\{ \bx \in \reals^3 \mid \implicitsphere_{\rho}(\bx, \p) = 0 \right\}$, alors que la boule $\ball{\p}{r}$ est le lieu des points $\left\{ \bx \in \reals^3 \mid \implicitsphere_{\rho}(\bx, \p) < 0 \right\}$.
%
%On note $\sphere[\p][\rho(\p)]$ la famille des sphères 
%\begin{equation}
%    \sphere[\Sigma][\rho] := \left\{ \sphere[\p][\rho(\p)] \mid \p \in \Sigma \right\}.
%\end{equation}
%L'enveloppe de cette famille est la surface qui est tangente à chacun de ses membres et qui est en tout point tangente à une sphère $\sphere[\p][\rho(\p)]$.
%Si $\bs$ est une paramétrisation de $\Sigma$ alors l'enveloppe de $\sphere[\Sigma][\rho]$ est le lieu des points $\bx \in \reals^3$ qui vérifient%
%\def\sysvspace{1.5ex}
%\begin{equation}
%    \left\{\begin{matrix}
%        \implicitsphere_{\rho}(\bx, \bs(u,v)) &= 0, \\[\sysvspace]
%        \frac{\partial }{\partial u}\implicitsphere_{\rho}(\bx, \bs(u,v)) &= 0, \\[\sysvspace]
%        \frac{\partial }{\partial v}\implicitsphere_{\rho}(\bx, \bs(u,v)) &= 0.
%    \end{matrix}\right.
%    \label{eq:sys_envelope_of_spheres}
%\end{equation}
%En considérant $\rho$ comme une fonction des paramètres $u$ et $v$, on peut réécrire le système \eqref{eq:sys_envelope_of_spheres} 
%\begin{equation}
%    \left\{\begin{matrix}
%        \normtwo{\bx - \bs(u,v)}^2 - \rho(u,v)^2 &= 0, \\
%        \dotprod{\left( \bx - \bs(u,v) \right)}{\bsu(u,v)} + \rho(u,v) \rho_u(u,v) &= 0, \\
%        \dotprod{\left( \bx - \bs(u,v) \right)}{\bsv(u,v)} + \rho(u,v) \rho_v(u,v) &= 0.
%    \end{matrix}\right.
%\end{equation}

